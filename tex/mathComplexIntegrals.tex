\باب{مخلوط تکملات}
مخلوط تکملات دو وجوہات کی بنا اہم ہیں۔عملی وجہ یہ ہے کہ حقیقی تکملات حل کرنے کی تراکیب سے کئی حقیقی تکملات حل کرنا ناممکن ہے جبکہ ان کو مخلوط تکملات کی ترکیب سے حل کیا جا سکتا ہے۔دوسری وجہ نظریاتی ہے۔ جہاں مخلوط تکملات کی ترکیب سے تحلیلی تفاعل کی چند بنیادی خصوصیات دریافت ہوتی ہیں (بالخصوص بلند درجی تفرق کی موجودگی) جن کا ثبوت  تکمل استعمال کیے بغیر انتہائی مشکل ہو گا۔یہ صورت حال حقیقی اور مخلوط علم الاحصاء میں بنیادی فرق کی نشاندہی کرتی ہے۔

اس باب میں ہم پہلے مخلوط تکملات کی تعریف پیش کرتے ہیں۔سب سے بنیادی نتیجہ  کوشی مخلوط تکمل کا مسئلہ حاصل ہو گا جس سے  سے کوشی تکمل کی کلیات حاصل ہوں گی جو بہت اہم  ہیں۔ ہم ثابت کریں گے کہ اگر کوئی تفاعل تحلیلی ہو تب اس کے ہر درجہ کے تفرق موجود ہوں گے۔اس نقطہ نظر سے مخلوط تحلیلی تفاعل حقیقی متغیر کی حقیقی تفاعل سے زیادہ سادہ رویہ رکھتے ہیں۔

%=====================
\حصہ{مخلوط مستوی میں خطی تکمل}
حقیقی علم الاحصاء کی طرح ہم قطعی تکمل اور غیر قطعی تکمل میں تمیز کرتے ہیں۔ایک غیر قطعی تکمل ایسا تفاعل ہوتا ہے جس کا تفرق خطے میں دیا گیا تحلیلی تفاعل ہو گا۔تفاعل کی تفرق کو الٹ لکھتے ہوئے ہم کئی غیر قطعی تکمل دریافت کر سکتے ہیں۔

آئیں اب مخلوط تفاعل \عددی{f(z)}، جہاں \عددی{z=x+iy} ہے، کی قطعی تکمل یا خطی تکمل کی تعریف پیش کرتے ہیں۔ہم دیکھیں گے کہ حقیقی قطعی تکمل کی تصور کو وسعت دیتے ہوئے  مخلوط قطعی تکمل کا تصور پیدا ہوتا ہے۔یوں موجودہ بحث عین حصہ \حوالہ{حصہ_سمتی_تکمل_خطی_تکمل} کی طرح ہو گی۔قطعی تکمل کی صورت میں حقیقی محور پر کوئی وقفہ تکمل کی راہ  ہو گی۔مخلوط قطعی تکمل کی صورت میں ہم مخلوط مستوی پر کسی منحنی\حاشیہد{درحقیقت منحنی کے کسی حصے یا قوس پر تکمل لیا جائے گا۔اپنی آسانی کی خاطر ہم  "منحنی" کی اصطلاح کو  پوری منحنی کے لئے اور منحنی کے چھوٹے حصہ کے لئے بھی استعمال کریں گے۔} پر چلتے ہوئے تکمل حاصل کریں گے۔

فرض کریں کہ  مخلوط \عددی{z} مستوی میں \عددی{C} ایک ہموار منحنی (حصہ \حوالہ{حصہ_نقش_محافظ_زاویہ_نقش}) ہے۔تب ہم \عددی{C} کو درج ذیل روپ میں لکھ سکتے ہیں
\begin{align}\label{مساوات_مخلوط_تکمل_راہ_الف}
z(t)=x(t)+iy(t)\quad \quad \quad (a\le t\le b)
\end{align}
جہاں تمام \عددی{t} کے لئے \عددی{z(t)} کا استمراری تفرق \عددی{\dot{z}(t)\ne 0} پایا جاتا ہے، اور یوں \عددی{C} قابل تصحیح (حصہ \حوالہ{حصہ_الاحصاء_لمبائی_قوس}) ہو گی جس کا ہر نقطہ پر یکتا مماس ہو گا۔آپ کو یاد ہو گا کہ \عددی{C} پر مثبت رخ سے مراد  \عددی{t} کی بڑھتی قیمت کا مطابقتی رخ  ہے۔

فرض کریں کہ  \عددی{f(z)} ایک استمراری تفاعل ہے جو (کم از کم)  \عددی{C} کی ہر نقطہ پر معین ہے۔ہم  مساوات \حوالہ{مساوات_مخلوط_تکمل_راہ_الف} میں دیے گئے وقفہ \عددی{a\le t\le b} کو درج ذیل ٹکڑوں میں تقسیم کرتے ہیں
\begin{align*}
t_0(=a),t_1,\cdots,t_{n-1},t_n(=b)
\end{align*} 
جہاں \عددی{t_0<t_1<\cdots<t_n} ہے۔اس کے مطابق \عددی{C} کے ٹکڑے (شکل \حوالہ{شکل_مخلوط_تکمل_ٹکڑے_راہ})
\begin{align*}
z_0,z_1,\cdots,z_{n-1},z_n()=Z
\end{align*}
 پائے جاتے ہیں جہاں \عددی{z_j=z(t_j)} ہے۔
\begin{figure}
\centering
\begin{tikzpicture}
%\draw[thick](0,0) grid (3,2);
%\draw[thin,gray,step=0.1] (0,0) grid (3,2);
\draw(0,0) node[ocirc]{}node[right]{$z_0$} to [out=135,in=-135] coordinate[pos=0.3](kA)coordinate[pos=0.5](kB) coordinate[pos=0.7](kC)coordinate[pos=0.9](kD)  (0.5,1.5) to [out=45,in=80] coordinate[pos=0.2](kE)coordinate[pos=0.5](kF)coordinate[pos=0.7](kG) coordinate[pos=0.8](kH)coordinate[pos=0.9](kI)coordinate[pos=1,ocirc](kJ)node[pos=1,below]{$Z$}(3,1);
\draw(kA)++(0:0.1)--++(0:-0.2)node[left]{$z_1$};
\draw(kB)++(-10:0.1)--++(-10:-0.2)node[left]{$z_2$};
\draw(kC)++(-40:0.1)--++(-40:-0.2)++(-40:-0.25)coordinate(kkA);
\draw(kD)++(-40:0.1)--++(-40:-0.2)++(-40:-0.25)coordinate(kkB);
\draw($(kkA)!0.5!(kkB)$)node[rotate=50]{$\cdots$};
\draw(kE)++(-60:0.1)--++(-60:-0.2)node[above]{$z_{m-1}$};
\draw(kF)node[ocirc]{}node[above]{$\zeta_m$};
\draw(kG)++(-115:0.1)--++(-115:-0.2)node[above]{$z_m$};
\draw(kH)++(-130:0.1)--++(-130:-0.2);
\draw(kI)++(-170:0.1)--++(-170:-0.2);
\draw(kE)--(kG)node[pos=0.5,below]{$\abs{\Delta z_m}$};
\end{tikzpicture}
\caption{مخلوط خطی تکمل}
\label{شکل_مخلوط_تکمل_ٹکڑے_راہ}
\end{figure}
ہم \عددی{C} کے ہر ٹکڑے پر کوئی اختیاری نقطہ منتخب کرتے ہیں، مثلاً ہم \عددی{z_0} اور \عددی{z_1} کے درمیان نقطہ \عددی{\zeta_1} منتخب کرتے ہیں 
(یعنی \عددی{\zeta_1=z(t)} جہاں \عددی{t_0\le t\le t_1} ہے) اور \عددی{z_1} اور \عددی{z_2} کے درمیان نقطہ \عددی{\zeta_2} منتخب کرتے ہیں، وغیرہ وغیرہ۔ہم اب مجموعہ
\begin{align}\label{مساوات_مخلوط_تکمل_راہ_ب}
S_n=\sum_{m=1}^{n}f(\zeta-m)\Delta z_m
\end{align}
لیتے ہیں جہاں
\begin{align*}
\Delta z_m=z_m-z_{m-1}
\end{align*}
ہے۔ہم ایسے مجموعے \عددی{n=2,3,\cdots} کے لئے مکمل بے قاعدگی سے حاصل کرتے ہیں پس اتنا دھیان رکھتے ہیں کہ جب \عددی{n} لامتناہی کے قریب پہنچے تب \عددی{\abs{\Delta z_m}} کی زیادہ سے زیادہ قیمت صفر کے قریب پہنچتی ہو۔یوں ہمیں مخلوط قیمتوں کا سلسلہ \عددی{S_2,S_3,\cdots} ملتا ہے۔اس سلسلے کی حد، راہ \عددی{C} پر  \عددی{f(z)} کا \اصطلاح{خطی تکمل}\فرہنگ{تکمل!خطی}\حاشیہب{line integral}\فرہنگ{integral!line} (یا صرف \ترچھا{تکمل}) کہلاتا ہے جس کو درج ذیل لکھا جاتا ہے۔
\begin{align}\label{مساوات_مخلوط_تکمل_راہ_پ}
\int\limits_C f(z)\dif z
\end{align}
منحنی \عددی{C} کو تکمل کی راہ کہتے ہیں۔

ہم درج ذیل پوری بحث میں فرض کرتے ہیں کہ مخلوط خطی تکمل کی تمام راہ \موٹا{ٹکڑوں میں ہموار} ہیں یعنی ہر راہ محدود تعداد کی ہموار منحنیات پر مشتمل ہے۔

ہمارے مفروضوں کی مد نظر  خطی تکمل مساوات \حوالہ{مساوات_مخلوط_تکمل_راہ_پ} موجود ہو گا، بلکہ \عددی{f(z)=u(x,y)+iv(x,y)} لکھتے ہوئے اور
\begin{align*}
\zeta_m=\xi_m+i\eta_m\quad \text{اور}\quad \Delta z_m=\Delta x_m+i\Delta y_m
\end{align*}
لیتے ہوئے مساوات \حوالہ{مساوات_مخلوط_تکمل_راہ_ب} کو 
\begin{align}\label{مساوات_مخلوط_تکمل_راہ_ت}
S_n=\sum(u+iv)(\Delta x_m+i\Delta y_m)
\end{align}
لکھا جا سکتا ہے جہاں \عددی{u=u(\zeta_m,\xi_m)} اور \عددی{v=v(\zeta_m,\xi_m)} ہیں اور ہم \عددی{m} کو \عددی{1} تا \عددی{n} لیتے ہوئے مجموعہ حاصل کرتے ہیں۔ہم اب \عددی{S_n} کو چار مجموعوں میں تقسیم کر سکتے ہیں۔
\begin{align*}
S_n=\sum u\Delta x_m-\sum v\Delta y_m+i[\sum u\Delta y_m+\sum v\Delta x_m]
\end{align*}
یہ مجموعے حقیقی ہیں۔چونکہ \عددی{f} استمراری ہے لہٰذا \عددی{u} اور \عددی{v} بھی استمراری ہوں گے۔یوں اگر ہم \عددی{n} کی قیمت کو متذکرہ بالا طریقے سے بڑھا کر لامتناہی کے قریب کریں تب \عددی{\Delta x_m} اور \عددی{\Delta y_m} کی زیادہ سے زیادہ قیمت صفر کے قریب ہو  گی اور دائیں ہاتھ ہر مجموعہ حقیقی تکمل کی صورت اختیار کرے گا:
\begin{align}\label{مساوات_مخلوط_تکمل_راہ_ٹ}
\lim_{n\to \infty} S_n =\int\limits_C f(z)\dif z=\int\limits_C u\dif x-\int\limits_C  v\dif y+i\big[\int\limits_C u\dif y+\int\limits_C v\dif x\big]
\end{align} 
اس سے ظاہر ہوتا ہے کہ خطی تکمل مساوات \حوالہ{مساوات_مخلوط_تکمل_راہ_پ} موجود ہو گا اور اس تکمل کی قیمت پر راہ ٹکڑے کرنے کی ترکیب اور ہر ٹکڑے کے بیچ نقطہ \عددی{\zeta_m} کی انتخاب کا کوئی اثر نہیں ہو گا۔

مزید، حصہ \حوالہ{حصہ_خطی_تکمل_کا_حل} کی طرح، منحنی \عددی{C} کی مساوات \حوالہ{مساوات_مخلوط_تکمل_راہ_الف} استعمال کرتے ہوئے ہم ان میں سے ہر حقیقی تکمل کو قطعی تکمل میں تبدیل کر سکتے ہیں:
\begin{align}\label{مساوات_مخلوط_تکمل_راہ_ث}
\int\limits_C f(z)\dif z=\int\limits_a^b u \dot{x}\dif t-\int\limits_a^b v\dot{y}\dif t+i\big[\int\limits_a^b u \dot{y} \dif t+\int\limits_a^b v\dot{x}\dif t\big]
\end{align}
جہاں \عددی{u=u[x(t),y(t)]}، \عددی{v=v[x(t),y(t)]} ہیں جبکہ \عددی{t} کے ساتھ تفرق کو نقطہ سے ظاہر کیا گیا ہے۔

ہم اس کو عموماً
\begin{align*}
\int\limits_C f(z)\dif z=\int_a^b (u+iv)(\dot{x}+i\dot{y})\dif t
\end{align*}
یا مختصراً
\begin{align}\label{مساوات_مخلوط_تکمل_راہ_ج}
\int\limits_C f(z)\dif z=\int\limits_a^b f[z(t)]\dot{z}(t)\dif t
\end{align}
لکھتے ہیں۔

آئیں چند سادہ مثالیں دیکھیں۔

%============================
\ابتدا{مثال}\شناخت{مثال_مخلوط_تکمل_دائرے_پر_تکمل}\quad \موٹا{اکائی دائرے پر \عددی{\tfrac{1}{z}} کا تکمل}\\
اکائی دائرہ \عددی{C} پر گھڑی کی الٹ رخ \عددی{z=1} سے شروع کر کے ایک چکر لگاتے ہوئے \عددی{f(z)=\tfrac{1}{z}} کا تکمل حاصل کریں۔ہم \عددی{C} کو درج ذیل روپ میں لکھ سکتے ہیں۔
\begin{align}\label{مساوات_مخلوط_تکمل_اکائی_رداس_الف}
z(t)=\cos t+i\sin t\quad \quad \quad (0\le t\le 2\pi)
\end{align}
یوں
\begin{align*}
\dot{z}(t)=-\sin t+i\cos t
\end{align*}
ہو گا لہٰذا مساوات \حوالہ{مساوات_مخلوط_تکمل_راہ_ج} کے تحت درکار تکمل
\begin{align*}
\int_C\frac{\dif z}{z}=\int_0^{2\pi} \frac{1}{\cos t+i\sin t} (-\sin t+i\cos t)\dif t=i\int_0^{2\pi} \dif t=i2\pi
\end{align*}
ہو گا۔یہ بنیادی نتیجہ ہے جو ہم بار بار استعمال کریں گے۔

ظاہر ہے کہ ہم مساوات \حوالہ{مساوات_مخلوط_تکمل_اکائی_رداس_الف} کو مختصراً
\begin{align}\label{مساوات_مخلوط_تکمل_اکائی_رداس_ب}
z(t)=e^{it}\quad \quad \quad (0\le t\le 2\pi)
\end{align}
لکھ سکتے ہیں۔یوں تفرق لیتے ہوئے
\begin{align*}
\dot{z}(t)=ie^{it},\quad \dif z=ie^{it}\dif t
\end{align*}
لکھ کر یہی نتیجہ
\begin{align}
\int_C \frac{\dif z}{z}=\int_0^{2\pi} \frac{1}{e^{it}}ie^{it}\dif t=i\int_0^{2\pi}\dif t=i2\pi
\end{align}
دوبارہ حاصل کرتے ہیں۔ 
\انتہا{مثال}
%=============================
\ابتدا{مثال}\شناخت{مثال_مخلوط_تکمل_غیر_تحلیلی}\quad \موٹا{غیر تحلیلی تفاعل کا تکمل}\\
سیدھی راہ \عددی{C_1} پر \عددی{z_0=0} تا \عددی{z=1+i} تفاعل \عددی{f(z)=\text{\RL{حقیقی $z\,$}}=x} کا تکمل تلاش کریں 
(شکل \حوالہ{شکل_مثال_مخلوط_تکمل_غیر_تحلیلی}-الف)۔\\
اس راہ کو درج ذیل روپ میں لکھا جا سکتا ہے۔
\begin{align*}
z(t)=x(t)+iy(t)=(1+i)t\quad \quad (0\le t\le 1)
\end{align*} 
یوں 
\begin{align*}
f[z(t)]=\text{\RL{حقیقی $z\,$}}=x(t)=t,\quad \dif z=(1+i)\dif t
\end{align*}
ہو گا جس سے ہم تکمل حاصل کرتے ہیں:
\begin{align*}
\int_{C_1} \text{\RL{حقیقی $z\,$}}\dif z=\int_0^1 t(1+i)\dif t=(1+i)\int_0^1 t\dif t=\frac{1}{2}(1+i)
\end{align*}

آئیں اب حقیقی محور پر \عددی{0} تا \عددی{1} چل کر، یہاں سے خیالی محور کے متوازی چلتے ہوئے  \عددی{1+i} تک اسی تفاعل \عددی{f(z)=\text{\RL{حقیقی $z\,$}}=x} کا تکمل حاصل کرتے ہیں (شکل \حوالہ{شکل_مثال_مخلوط_تکمل_غیر_تحلیلی}-الف میں راہ \عددی{C_2})۔ ہم اس راہ  کے پہلے حصے کو
\begin{align*}
z=z(t)=t\quad \quad\quad (0\le t\le 1)
\end{align*}
اور دوسرے حصے کو
\begin{align*}
z(t)=1+i(t-1)\quad \quad \quad (1\le t\le 2)
\end{align*}
لکھ سکتے ہیں۔یوں پوری راہ وقفہ \عددی{0\le t\le 2} کی مطابقتی ہو گی۔پہلے حصے پر \عددی{\text{\RL{حقیقی $z\,$}}=t,\,\dif z=\dif t} اور دوسرے حصے پر \عددی{\text{\RL{حقیقی $z\,$}}=1,\,\dif z=i\dif t} ہو گا۔یوں پورا تکمل دو ٹکڑوں میں حاصل ہو گا:
\begin{align*}
\int_{C_2} \text{\RL{}}\dif z=\int_0^1t\dif t+\int_1^2 i\dif t=\frac{1}{2}+i
\end{align*}
آپ دیکھ سکتے ہیں کہ راہ کے دوسرے حصے کو درج ذیل بھی لکھا جا سکتا ہے
\begin{align*}
z(t)=1+it\quad\quad\quad(0\le t\le 1)
\end{align*}
جس کو استعمال کرتے ہوئے تکمل کے حدود \عددی{0} اور \عددی{1} ہوں گے اور تکمل کی قیمت وہی رہے گی۔

اس مثال سے آپ دیکھ سکتے ہیں کہ غیر تحلیلی تفاعل کے تکمل کی قیمت  نا صرف راہ کی آخری حدود بلکہ راہ کی جیومیٹریائی شکل  پر بھی منحصر ہوتی ہے۔
\begin{figure}
\centering
\begin{subfigure}{0.5\textwidth}
\centering
\begin{tikzpicture}
\draw(0,0)--(2.5,0)node[below]{$x$};
\draw(0,0)--(0,1.75)node[left]{$y$};
\draw[thick,dashed,->-=0.5](0,0)--(1.5,1.5)node[pos=0.6,above left]{$C_1$};
\draw[thick,->-=0.5](0,0)--(1.5,0)node[below]{$1$}node[pos=0.5,below]{$C_2$};
\draw[thick,->-=0.5](1.5,0)--(1.5,1.5)node[ocirc]{}node[right]{$z=1+i$};
\end{tikzpicture}
\caption*{(الف) مثال \حوالہ{مثال_مخلوط_تکمل_غیر_تحلیلی} میں تکمل کی راہ}
\end{subfigure}%
\begin{subfigure}{0.5\textwidth}
\centering
\begin{tikzpicture}
\draw(0,0)--(2,0)node[below]{$x$};
\draw(0,0)--(0,1.75)node[left]{$y$};
\draw[->-=0.125](1,0.825) circle (0.75);
\draw[-latex](1,0.825)node[ocirc]{}node[below]{$z_0$}--++(135:0.75)node[pos=0.4,shift={(45:0.25)}]{$\rho$};
\draw(1,0.825)++(42:0.95)node[]{$C$};
\end{tikzpicture}
\caption*{(ب) مثال \حوالہ{مثال_مخلوط_تکمل_طاقت} میں تکمل کی راہ}
\end{subfigure}
\caption{تکملات کی راہ}
\label{شکل_مثال_مخلوط_تکمل_غیر_تحلیلی}
\end{figure}
\انتہا{مثال}
%==========================
\ابتدا{مثال}\شناخت{مثال_مخلوط_تکمل_طاقت}\quad \موٹا{عدد صحیح طاقت کے تکمل}\\
مان لیں کہ \عددی{f(z)=(z-z_0)^m} ہے جہاں \عددی{m} عدد صحیح اور \عددی{z_0} مستقل ہیں۔گھڑی کی الٹ رخ رداس \عددی{\rho} کے دائرہ \عددی{C} پر تکمل حاصل کریں۔دائرے کا مرکز \عددی{z_0} ہے (شکل \حوالہ{شکل_مثال_مخلوط_تکمل_غیر_تحلیلی}-ب)۔ \\
ہم \عددی{C} کو 
\begin{align*}
z(t)=z_0+\rho(\cos t+i\sin t)=z_0+\rho e^{it} \quad \quad (0\le t\le 2\pi)
\end{align*}
لکھ سکتے ہیں۔یوں 
\begin{align*}
(z-z_0)^m=\rho^m e^{imt},\quad \dif z=i\rho e^{it}\dif t
\end{align*}
ہو گا لہٰذا تکمل درج ذیل ہو گا۔
\begin{align*}
\int_C (z-z_0)^m\dif z=\int_0^{2\pi} \rho^m e^{imt} i\rho e^{it}\dif t=i\rho^{m+1}\int_0^{2\pi} e^{i(m+1)t} \dif t
\end{align*}
\عددی{m=-1} کی صورت مثال \حوالہ{مثال_مخلوط_تکمل_دائرے_پر_تکمل} میں دیکھی گئی ہے جبکہ \عددی{m\ne -1} کی صورت میں درج ذیل  ہو گا (مساوات \حوالہ{مساوات_تحلیلی_قوت_نمائی_خیالی_دوری_عرصہ} دیکھیں):
\begin{align*}
\int_0^{2\pi} e^{i(m+1)t}=\big[\frac{e^{i(m+1)t}}{i(m+1)}\big]_0^{2\pi}=0\quad \quad (m\ne -1, \text{\RL{عدد صحیح}})
\end{align*}
یوں تکمل کا حل درج ذیل ہو گا۔
\begin{align}
\int_C (z-z_0)^m\dif z=
\begin{cases}
i2\pi& (m=-1)\\
0& (m\ne -1, \text{\RL{عدد صحیح}})
\end{cases}
\end{align}
\انتہا{مثال}
%============================
\ابتدا{مثال}\quad \موٹا{تکمل کی تعریف کی عملی استعمال}\\
مان لیں کہ \عددی{f(z)=\text{مستقل}=k} ہے جبکہ ابتدائی نقطہ \عددی{z_0} اور اختتامی نقطہ \عددی{Z} کے درمیان \عددی{C} کوئی راہ ہے۔اس صورت میں ہم تکمل کی تعریف، یعنی مساوات \حوالہ{مساوات_مخلوط_تکمل_راہ_ب} میں دیے گئے مجموعہ \عددی{S_n} کی حد، استعمال کرتے ہیں۔یوں
\begin{align*}
S_n=\sum_{m=1}^{n} k\Delta z_m=k[(z_1-z_0) +(z_2-z_1)+\cdots+(Z-z_{n-1})]=k(Z-z_0)
\end{align*}
ہو گا جس سے تکمل کی قیمت درج ذیل حاصل ہو گی۔
\begin{align*}
\int_C k\dif z=\lim_{n\to \infty}S_n=k(Z-z_0)
\end{align*}
ہم دیکھتے ہیں کہ اس تکمل کی قیمت صرف ابتدائی اور اختتامی نقطوں \عددی{z_0} اور \عددی{Z} پر منحصر ہے نا ان نقطوں کے مابین راہ پر۔بالخصوص اگر راہ \عددی{C} بند ہو  تب \عددی{Z=z_0} ہو گا لہٰذا تکمل کی قیمت صفر ہو گی۔
\انتہا{مثال}
%=========================
\ابتدا{مثال}\quad \موٹا{تکمل کی تعریف کی دوسری مثال}\\
فرض کریں کہ \عددی{f(z)=z} ہے جبکہ  ابتدائی نقطہ \عددی{z_0} اور اختتامی نقطہ \عددی{Z} کے مابین \عددی{C} کوئی راہ ہے۔ہم دوبارہ  مساوات \حوالہ{مساوات_مخلوط_تکمل_راہ_ب}  استعمال کرتے ہیں۔\عددی{\zeta_m=z_m} لیتے ہوئے
\begin{align*}
S_n=\sum_{m=1}^n z_m\Delta z_m=z_1(z_1-z_0)+z_2(z_2-z_1)+\cdots+Z(Z-z_{n-1})
\end{align*}
حاصل ہو گا۔اسی طرح \عددی{\zeta_m=z_{m-1}} لیتے ہوئے
\begin{align*}
S^*_n=\sum_{m=1}^n z_{m-1} \Delta z_m=z_0(z_1-z_0)+z_1(z_2-z_1)+\cdots+z_{n-1}(Z-z_{n-1})
\end{align*}
حاصل ہو گا۔ان دونوں کو جمع کرتے ہوئے \عددی{S_n+S^*_n=Z^2-z_0^2} ملتا ہے۔یوں
\begin{align*}
\lim_{n\to \infty} (S_n+S^*_n)=2\int_{z_0}^Z z\dif z=Z^2-z_0^2
\end{align*}
ہو گا  جس سے ان نقطوں کے مابین ہر راہ پر تکمل کی قیمت
\begin{align*}
\int_{z_0}^Z z\dif z=\frac{1}{2} (Z^2-z_0^2)
\end{align*}
حاصل ہوتی ہے۔بالخصوص اگر \عددی{C} بند راہ ہو تب \عددی{Z=z_0} ہو گا لہٰذا
\begin{align}
\oint_C z\dif z=0
\end{align}
ہو گا۔یہی نتیجہ مسئلہ \حوالہ{خطی_تکمل_سطحی_مسئلہ_گرین} سے  مساوات \حوالہ{مساوات_مخلوط_تکمل_راہ_ث} میں دیے گئے کلیہ کی مدد سے  بھی حاصل کیا جا سکتا ہے۔
\انتہا{مثال}
%============================

\حصہء{سوالات}
سوال \حوالہ{سوال_مخلوط_تکمل_راہ_کی_روپ_الف} تا سوال \حوالہ{سوال_مخلوط_تکمل_راہ_کی_روپ_الف} میں \عددی{A} تا \عددی{B} قطع کو \عددی{z=z(t)} روپ میں لکھیں۔

%================
\ابتدا{سوال}\شناخت{سوال_مخلوط_تکمل_راہ_کی_روپ_الف}\quad
$A:z=0,\quad B:z=1+i2$\\
\انتہا{سوال}
%=======================
