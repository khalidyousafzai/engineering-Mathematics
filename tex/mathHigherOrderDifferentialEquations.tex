\باب{بلند درجی خطی سادہ تفرقی مساوات}
دو درجی خطی سادہ تفرقی مساوات کو حل کرنے کے طریقے بلند درجی خطی سادہ ترفقی مساوات کے لئ بھی قابل استعمال ہیں۔ہم دیکھیں گے کہ بلند درجی صورت میں مساوات زیادہ پیچیدہ ہوں گے،  امتیازی مساوات کے جذر بھی تعداد میں زیادہ اور حصول میں نسبتاً مشکل ہوں گے اور ورونسکی زیادہ اہم کردار ادا کرے گا۔ 

\حصہ{متجانس خطی سادہ تفرقی مساوات}
\عددی{n} درجی سادہ تفرقی مساوات سے مراد ایسی مساوات ہے جس میں نا معلوم متغیرہ \عددی{y(x)} کا \عددی{y^n=\tfrac{\dif^{\, n} y}{\dif x^n}} سب سے بلند درجی تفرق ہو۔ایسی سادہ تفرقی مساوات کو
\begin{align*}
F(x,y,y',\cdots, y^{(n)})=0
\end{align*}
لکھا جا سکتا ہے جس میں \عددی{y} اور کم درجی تفرق موجود یا غیر موجود ہو سکتے ہیں۔ایسی مساوات کو \اصطلاح{خطی}\فرہنگ{خطی}\فرہنگ{linear} کہتے ہیں اگر اس کو 
\begin{align}\label{مساوات_سادہ_بلند_خطی_الف}
y^{(n)}+p_{n-1}(x)y^{(n-1)}+\cdots+p_1(x)y'+p_0(x)y=r(x)
\end{align}
لکھنا ممکن ہو۔صفحہ \حوالہصفحہ{مساوات_سادہ_دو_درجی_تعریف} پر دو درجی خطی سادہ تفرقی مساوات کی بات کی گئی۔موجودہ مساوات میں \عددی{n=2}، \عددی{p_1=p} اور \عددی{p_0=q} پر کرنے سے دو درجی مساوات حاصل ہو گی۔عددی سر \عددی{p_0(x)} تا \عددی{p_n(x)}  اور جبری تفاعل \عددی{r(x)} غیر تابع متغیرہ  \عددی{x} کے کوئی بھی تفاعل ہو سکتے ہیں جبکہ \عددی{y(x)} نا معلوم متغیرہ ہے۔خطی مساوات کو معیاری صورت میں لکھا گیا ہے جہاں \عددی{y^{(n)}} کا عددی سر اکائی \عددی{1} ہے۔ تفرقی مساوات میں  \عددی{p_n(x)y^{(n)}} موجود ہونے کی صورت میں پوری مساوات کو \عددی{p_n(x)} سے تقسیم کرتے ہوئے معیاری صورت حاصل کریں۔جو تفرقی مساوات درج بالا صورت میں لکھنا ممکن نہ ہو \اصطلاح{غیر خطی}\فرہنگ{غیر خطی}\فرہنگ{non linear} کہلاتی ہے۔

کسی کھلے وقفے \عددی{I} پر \عددی{r(x)} \اصطلاح{مکمل صفر}\فرہنگ{مکمل صفر} \عددی{r \equiv 0} ہونے کی صورت میں  مساوات \حوالہ{مساوات_سادہ_بلند_خطی_الف} سے \اصطلاح{متجانس مساوات}\فرہنگ{متجانس مساوات}\فرہنگ{homogeneous}
\begin{align}\label{مساوات_سادہ_بلند_خطی_ب}
y^{(n)}+p_{n-1}(x)y^{(n-1)}+\cdots+p_1(x)y'+p_0(x)y=0
\end{align}
حاصل ہوتی ہے۔کھلے وقفے پر \عددی{r(x)} کے مکمل صفر ہونے سے مراد یہ ہے کہ اس وقفے پر ہر \عددی{x} کے لئے \عددی{r(x)} کی قیمت صفر کے برابر ہے۔دو درجی تفرقی مساوات کی طرح  اگر \عددی{r(x)} مکمل صفر نہ ہو تب مساوات \اصطلاح{غیر متجانس}\فرہنگ{غیر متجانس}\فرہنگ{nonhomogeneous} کہلائے گی۔

کھلے وقفہ \عددی{I} پر \عددی{n} درجی خطی یا غیر خطی سادہ تفرقی مساوات کے حل \عددی{y=h(x)} سے مراد ایسا تفاعل ہے جو \عددی{I} پر معین ہو،  کھلے وقفے پر اس کا \عددی{n} درجی تفرق موجود ہو اور تفرقی مساوات میں \عددی{y} اور اس کے تفرقات کی جگہ \عددی{h} اور اس کے تفرقات پر کرنے سے مساوات کے دونوں اطراف بالکل یکساں حاصل ہوں۔ 
%=======================

\جزوحصہء{متجانس خطی سادہ تفرقی مساوات:خطی میل اور عمومی حل}
\اصطلاح{خطی میل}\فرہنگ{خطی میل} یا \اصطلاح{اصول خطیت}\فرہنگ{اصول خطیت} جس کا ذکر صفحہ \حوالہصفحہ{مسئلہ_دو_درجی_خطی_میل} مسئلہ \حوالہ{مسئلہ_دو_درجی_خطی_میل} میں کیا گیا بلند درجی خطی متجانس سادہ تفرقی مساوات کے لئے بھی درست ہے۔
%===============

\ابتدا{مسئلہ}\quad بنیادی مسئلہ برائے متجانس خطی سادہ بلند درجی تفرقی مساوات\فرہنگ{مسئلہ!بنیادی۔متجانس خطی}\\
کھلے وقفہ \عددیء{I} پر متجانس خطی بلند درجی تفرقی مساوات \حوالہ{مساوات_سادہ_بلند_خطی_ب} کے حل کا خطی میل بھی \عددیء{I} پر اس مساوات کا حل ہو گا۔بالخصوص ان حل کو مستقل مقدار سے ضرب دینے سے بھی مساوات کے حل حاصل ہوتے ہیں۔(یہ اصول غیر خطی اور  غیر متجانس مساوات پر لاگو نہیں ہوتا۔)
\انتہا{مسئلہ}
%==========================

اس کا ثبوت گزشتہ باب میں دئے گئے ثبوت کی طرح ہے جس کو یہاں پیش نہیں کیا جائے گا۔

ہماری بقایا گفتگو ہو بہو دو درجی تفرقی مساوات کی طرح ہو گی لہٰذا یہاں بلند درجی خطی متجانس مساوات کی عمومی حل کی بات کرتے ہیں۔ایسا کرنے کی خاطر \عددی{n} عدد تفاعل کی  \اصطلاح{خطی طور غیر تابع}\فرہنگ{غیر تابع!خطی طور}\فرہنگ{خطی طور!غیر تابع} ہونے کی تصور کو وسعت دیتے ہیں۔

%====================
\ابتدا{تعریف}\quad عمومی حل، اساس اور مخصوص حل\\
کھلے وقفے \عددی{I} پر مساوات \حوالہ{مساوات_سادہ_بلند_خطی_ب} کا \اصطلاح{عمومی حل}\فرہنگ{عمومی حل}\فرہنگ{general solution}
\begin{align}
y(x)=c_1y_1(x)+c_2y_2(x)+\cdots +c_ny_n(x)
\end{align}
ہے جہاں \عددی{y_1(x)} تا \عددی{y_n(x)} حل کی اساس اور \عددی{c_1} تا \عددی{c_2} اختیاری مستقل ہیں۔یوں \عددی{y_1} تا \عددی{y_n} کھلے وقفے پر خطی طور غیر تابع ہیں۔ 

عمومی حل کے مستقل کی قیمتیں مقرر کرنے سے \اصطلاح{مخصوص حل}\فرہنگ{مخصوص حل}\فرہنگ{particular solution} حاصل ہو گا۔
\انتہا{تعریف}
%=========================

\ابتدا{تعریف}\quad خطی طور تابع تفاعل اور خطی طور غیر تابع تفاعل\\
تصور کریں کہ کھلے وقفے \عددی{I} پر  \عددی{n} عدد تفاعل \عددی{y_1(x)} تا \عددی{y_n(x)} معین ہیں۔

 وقفہ \عددی{I} پر معین \عددی{y_1} تا \عددی{y_n}،  اس وقفے   پر اس صورت \اصطلاح{خطی طور غیر تابع}\فرہنگ{خطی طور! غیر تابع}\حاشیہب{linearly independent}\فرہنگ{linearly independent} کہلاتے ہیں جب پورے وقفے پر
\begin{align}\label{مساوات_سادہ_بلند_خطی_طور_غیر_تابع_الف}
k_1 y_1(x)+k_2 y_2(x)+\cdots+k_ny_n(x)=0
\end{align}
سے مراد 
\begin{align*}
k_1=k_2= \cdots =k_n=0 
\end{align*}
ہو۔\عددی{k_1} تا \عددی{k_n} میں  کم از کم ایک کی قیمت صفر نہ ہونے کی صورت میں مساوات \حوالہ{مساوات_سادہ_بلند_خطی_طور_غیر_تابع_الف} پر پورا اترتے ہوئے حل \عددی{y_1} تا \عددی{y_n} \اصطلاح{خطی طور تابع}\فرہنگ{خطی طور!تابع}\حاشیہب{linearly dependent}\فرہنگ{linearly dependent} کہلاتے ہیں۔
\انتہا{تعریف}
%==========================

\عددی{y_1} تا \عددی{y_n} میں (کم از کم ایک) تفاعل کو اس صورت بقایا تفاعل کے \اصطلاح{خطی میل}\فرہنگ{خطی میل} کے طرز پر لکھا جا سکتا ہے جب اس وقفے پر \عددی{y_1} تا \عددی{y_n} خطی طور تابع ہوں۔ یوں اگر \عددی{k_1 \ne 0} ہو تب ہم مساوات \حوالہ{مساوات_سادہ_بلند_خطی_طور_غیر_تابع_الف} کو \عددی{k_1} سے تقسیم کرتے ہوئے
\begin{align*}
y_1=-\frac{1}{k_1}(k_2 y_2+k_3y_3+\cdots+k_ny_n)
\end{align*}
 لکھ سکتے ہیں جو تناسبی رشتہ ہے۔یہ مساوات کہتی ہے کہ \عددی{y_1} کو بقایا تفاعل کے خطی میل کی صورت میں لکھا جا سکتا ہے۔اسی کو خطی طور تابع کہتے ہیں۔آپ دیکھ سکتے ہیں کہ \عددی{n=2} کی صورت میں ہمیں حصہ \حوالہ{حصہ_سادہ_دو_وجودیت_یکتائی_ورونسکی} میں بیان کئے گئے تصورات ملتے ہیں۔
%=======================

\ابتدا{مثال}\quad خطی طور تابع\\
ثابت کریں کہ تفاعل \عددی{y_1=2\sin x}، \عددی{y_2=1.5x^2}، \عددی{y_3=5\cos x+\sin x} اور \عددی{y_4=4\cos x} کسی بھی کھلے وقفے پر خطی طور تابع ہیں۔

حل:ہم \عددی{y_3=\tfrac{1}{2}y_1+0y_2+\tfrac{5}{4}y_4} لکھ سکتے ہیں لہٰذا \عددی{y_1} تا \عددی{y_4} خطی طور تابع تفاعل ہیں۔
\انتہا{مثال}
%============================
\ابتدا{مثال}\شناخت{مثال_سادہ_بلند_خطی_طور_غیر_تابع}\quad خطی طور غیر تابع\\
ثابت کریں کہ \عددی{y_1=x}، \عددی{y_2=x^3} اور \عددی{y=x^4}  کسی بھی کھلے وقفے پر خطی طور غیر تابع ہیں۔

حل:ہم  مساوات \عددی{k_1y_1+k_2y_2+k_3y_3=0} میں مختلف \عددی{x} کی قیمتیں پر کرتے ہوئے \عددی{k_1} تا \عددی{k_3} دریافت کرتے ہیں۔کھلے وقفے پر نقطہ \عددی{x=1}، \عددی{x=-1} اور \عددی{x=2} چنتے ہوئے درج ذیل ہمزاد مساوات ملتے ہیں۔
\begin{align*}
k_1+k_2+k_3&=0\\
-k_1-k_2+k_3&=0\\
2k_1+8k_2+16k_3&=0
\end{align*}
ان ہمزاد مساوات کو حل کرتے ہوئے \عددی{k_1=0}، \عددی{k_2=0} اور \عددی{k_3=0} ملتا ہے جو خطی طور غیر تابع ہونے کا ثبوت ہے۔
\انتہا{مثال}
%========================
\ابتدا{مثال}\شناخت{مثال_سادہ_بلند_اساس_عمومی_حل_الف}\quad اساس۔عمومی حل\\
تین درجی سادہ تفرقی مساوات \عددی{y^{(3)}-y'=0} کا عمومی حل تلاش کریں۔ \عددی{y^{(3)}} سے مراد \عددی{\tfrac{\dif^{\,3} y}{\dif x^3}} ہے۔

حل:حصہ \حوالہ{حصہ_سادہ_دو_درجی_مستقل_عددی_سر} کی طرح ہم اس متجانس مساوات کا حل \عددی{y=e^{\lambda x}} تصور کرتے  ہوئے امتیازی مساوات
\begin{align*}
\lambda^3-\lambda=0
\end{align*}
حاصل کرتے ہیں۔اس کو \عددی{\lambda(\lambda^2-1)=0} لکھتے ہوئے \عددی{\lambda=0} اور \عددی{\lambda=\mp 1} ملتے ہیں جن سے اساس \عددی{y_1=c}، \عددی{y_2=e^x} اور \عددی{y_3=e^{-x}} ملتا ہے۔جیسا مثال \حوالہ{مثال_سادہ_بلند_اساس_عمومی_حل_ب} میں ثابت کیا جائے گا، یہ اساس کسی بھی کھلے وقفے پر خطی طور غیر تابع ہیں لہٰذا کسی بھی کھلے وقفے پر  عمومی حل 
\begin{align*}
y=c_1+c_2e^x+c_3e^{-x}
\end{align*}
ہو گا۔
\انتہا{مثال}
%=======================

\جزوحصہء{ابتدائی قیمت مسئلہ۔وجودیت اور یکتائی}
مساوات \حوالہ{مساوات_سادہ_بلند_خطی_ب} پر مبنی ابتدائی قیمت مسئلہ مساوات \حوالہ{مساوات_سادہ_بلند_خطی_ب} اور درج ذیل \عددی{n} \اصطلاح{ابتدائی شرائط}\فرہنگ{ابتدائی!شرائط} پر مشتمل ہو گا
\begin{align}\label{مساوات_سادہ+بلند_ابتدائی_شرائط}
y(x_0)=K_0, y'(x_0)=K_1,\cdots , y^{(n-1)}(x_0)=K_{n-1}
\end{align}
جہاں \عددی{x_0} کھلے وقفے \عددی{I} پر ایک نقطہ اور \عددی{K_0} تا \عددی{K_{n-1}} اس نقطے پر  دیے گئے مقدار ہیں۔

صفحہ \حوالہصفحہ{مسئلہ_سادہ_دو_درجی_یکتا_مخصوص_حل} پر مسئلہ \حوالہ{مسئلہ_سادہ_دو_درجی_یکتا_مخصوص_حل} کو وسعت دیتے ہیں جس سے درج ذیل ملتا ہے۔
%============================

\ابتدا{مسئلہ}\شناخت{مسئلہ_سادہ_بلند_درجی_یکتا_مخصوص_حل}\quad مسئلہ وجودیت اور یکتائی برائے ابتدائی قیمت بلند درجی  تفرقی مساوات\فرہنگ{مسئلہ!وجودیت اور یکتائی}\\
کھلے وقفہ \عددی{I} پر مساوات \حوالہ{مساوات_سادہ_بلند_خطی_ب} کے عددی سر \عددی{p_0} تا \عددی{p_{n-1}} استمراری ہونے کی صورت میں اگر \عددی{x_0} کھلے وقفے پر پایا جاتا ہو تب مساوات \حوالہ{مساوات_سادہ_بلند_خطی_ب} اور مساوات \حوالہ{مساوات_سادہ+بلند_ابتدائی_شرائط} پر مبنی ابتدائی قیمت مسئلے کا \عددی{I} پر  \اصطلاح{یکتا حل}\فرہنگ{یکتا حل} \عددی{y(x)} \اصطلاح{موجود}\فرہنگ{حل!موجود}\فرہنگ{موجود!حل} ہے۔
\انتہا{مسئلہ}
%============================

حل کی موجودگی اور یکتائی کا ثبوت اس کتاب میں نہیں دیا جائے گا۔

%================
\ابتدا{مثال}\quad تین درجی یولر کوشی مساوات کا ابتدائی قیمت مسئلہ\\
درج ذیل ابتدائی قیمت مسئلے کو حل کریں۔
\begin{align*}
x^3y'''-5x^2y''+12xy'-12y=0,\quad y(1)=1, \quad y'(1)=-1, \quad y''(1)=0
\end{align*}

حل:ہم تفرقی مساوات میں آزمائشی تفاعل \عددی{y=x^m} پر کرتے ہوئے امتیازی مساوات
\begin{align*}
m^3-8m^2+19m-12=0
\end{align*}
حاصل کرتے ہیں جس کے جذر \عددی{m=1}، \عددی{m=3} اور \عددی{m=4} ہیں۔جذر  کو مختلف طریقوں سے حاصل کیا جاتا ہے البتہ یہاں جذر حاصل کرنے پر بحث نہیں کی جائے گی۔ یوں حل کی اساس \عددی{y_1=x}، \عددی{y_2=x^3} اور \عددی{y_3=x^4} ہیں جنہیں مثال \حوالہ{مثال_سادہ_بلند_خطی_طور_غیر_تابع} میں خطی طور غیر تابع ثابت کیا گیا۔اس طرح عمومی حل
\begin{align*}
y=c_1x+c_2x^3+c_3x^4
\end{align*}
ہو گا۔دیے گئے تفرقی مساوات کو  \عددی{x^3} سے تقسیم کرتے ہوئے \عددی{y'''} کا عددی سر اکائی حاصل کرتے ہوئے تفرقی مساوات کی معیاری صورت حاصل ہوتی ہے۔معیاری صورت میں مساوات کے دیگر عددی سر \عددی{x=0} پر غیر استمراری ہیں۔اس کے باوجود درج بالا عمومی حل تمام \عددی{x} بشمول  \عددی{x=0} کے لئے درست ہے۔ 

عمومی حل اور اس کے تفرقات \عددی{y'=c_1+3c_2x^2+4c_3x^3} اور \عددی{y''=6c_2x+12c_3x^2} میں ابتدائی معلومات پر کرتے ہوئے درج ذیل ہمزاد مساوات ملتے ہیں
\begin{align*}
c_1+c_2+c_3&=1\\
c_1+3c_2+4c_3&=-1\\
6c_2+12c_3&=0
\end{align*}
جن کا حل \عددی{c_1=3}، \عددی{c_2=-4} اور \عددی{c_3=2} ہے۔اس طرح مخصوص حل درج ذیل ہو گا۔
\begin{align*}
y=3x-4x^3+2x^4
\end{align*}
\انتہا{مثال}
%==========================

\جزوحصہء{خطی طور غیر تابع حل۔ورونسکی}
عمومی حل کے حصول کے لئے ضروری ہے کہ حل خطی طور غیر تابع ہوں۔اگرچہ عموماً حل کو دیکھ کر ہی اندازہ ہو جاتا ہے کہ  وہ خطی طور غیر تابع ہیں یا نہیں ہیں، البتہ ایسا معلوم کرنے کا منظم طریقہ زیادہ بہتر ہو گا۔صفحہ \حوالہصفحہ{مسئلہ_سادہ_دو_حل_تابع_غیر_تابع} پر مسئلہ \حوالہ{مسئلہ_سادہ_دو_حل_تابع_غیر_تابع} دو درجی  \عددی{n=2} مساوات کے علاوہ بلند درجی مساوت کے لئے بھی درست ہے۔ بلند درجی مساوات کی صورت میں ورونسکی درج ذیل ہو گی۔
\begin{align}\label{مساوات_سادہ_بلند_ورونسکی_الف}
W(y_1,\cdots, y_n)=
\begin{vmatrix}
y_1 & y_2 & \cdots & y_n\\
y'_1 & y'_2 & \cdots & y'_n\\
\vdots & & \\
y_1^{(n-1)} & y_2^{(n-1)} & \cdots & y_n^{(n-1)}
\end{vmatrix}
\end{align}
ورونسکی تفرقی مساوات کے حل \عددی{y_1} تا \عددی{y_n} پر مبنی ہے جو از خود \عددی{x} پر مبنی ہیں۔ورونسکی غیر صفر ہونے کی صورت میں  \عددی{y_1} تا \عددی{y_n} خطی طور غیر تابع ہوں گے۔
%========================

\ابتدا{مسئلہ}\شناخت{مسئلہ_سادہ_بلند_حل_تابع_غیر_تابع}\quad خطی طور تابع اور غیر تابع حل\فرہنگ{مسئلہ!تابع اور غیر تابع حل}\\
کھلے وقفہ \عددی{I} پر استمراری  \عددی{p_0(x)} تا  \عددی{p_{n-1}(x)} عددی سر والے سادہ تفرقی  مساوات \حوالہ{مساوات_سادہ_بلند_خطی_ب} کے \عددی{I}  پر حل \عددی{y_1} تا \عددی{y_n} اس صورت \اصطلاح{خطی طور تابع}\فرہنگ{خطی طور تابع}\فرہنگ{linearly dependent} ہوں گے جب ان کے \اصطلاح{ورونسکی}\فرہنگ{ورونسکی}\حاشیہب{Wronskian}\فرہنگ{Wronskian} کی قیمت  کسی \عددی{x_0} پر صفر کے برابر  ہو، جہاں \عددی{x_0} کھلے وقفے \عددی{I} پر پایا جاتا ہے۔مزید اگر  نقطہ \عددی{x=x_0} پر \عددی{W=0} ہو تب پورے \عددی{I} پر \عددی{W} \اصطلاح{مکمل صفر}\فرہنگ{مکمل صفر}\فرہنگ{صفر!مکمل}\حاشیہب{identically zero}\فرہنگ{identically zero} ہو گا۔یوں اگر \عددی{I} پر کوئی ایسا \عددی{x} پایا جاتا ہو جس پر \عددی{W} صفر کے برابر نہ ہو تب \عددی{I} پر  \عددی{y_1} تا \عددی{y_n} \اصطلاح{خطی طور غیر تابع}\فرہنگ{خطی طور غیر تابع}\فرہنگ{linearly independent} ہوں گے  اور یہ حل کی اساس ہوں گے۔
\انتہا{مسئلہ}
%==============================

\ابتدا{ثبوت}

(الف) \quad تصور کریں کہ کھلے وقفہ \عددی{I} پر  \عددی{y_1} تا \عددی{y_n} مساوات \حوالہ{مساوات_سادہ_بلند_خطی_ب} کے حل ہیں۔یوں خطی طور غیر تابع کی تعریف سے 
\begin{align}\label{مساوات_سادہ_بلند_ورونسکی_ب}
k_1y_1+k_2y_2+\cdots+k_ny_n=0
\end{align}
لکھا جا سکتا ہے۔ \عددی{I} پر اس مساوات کی \عددی{n-1} تفرقات لیتے ہیں۔
\begin{gather}
\begin{aligned}\label{مساوات_سادہ_بلند_ورونسکی_پ}
k_1y'_1+k_2y'_2+\cdots+k_ny'_n&=0\\
k_1y''_1+k_2y''_2+\cdots+k_ny''_n&=0\\
\vdots &\\
k_1y^{(n-1)}_1+k_2y^{(n-1)}_2+\cdots+k_ny^{(n-1)}_n&=0
\end{aligned}
\end{gather}
مساوات \حوالہ{مساوات_سادہ_بلند_ورونسکی_ب} اور مساوات \حوالہ{مساوات_سادہ_بلند_ورونسکی_پ} \عددی{n} عدد خطی متجانس ہمزاد  الجبرائی مساوات کا نظام ہے جس کا \اصطلاح{غیر صفر حل}\فرہنگ{غیر صفر حل}\فرہنگ{حل!غیر صفر}\حاشیہب{non trivial solution}\فرہنگ{non trivial solution} \عددی{k_1} تا \عددی{k_n} ہے لہٰذا \عددی{I} پر تمام \عددی{x} کے لئے، اس نظام کی عددی سر قالب کی حتمی قیمت، \اصطلاح{مسئلہ کریمر}\فرہنگ{مسئلہ!کریمر}\فرہنگ{کریمر!مسئلہ}\حاشیہب{Cramer's theorem}\فرہنگ{Cramer's!theorem} [جسے باب-7 میں پیش کیا گیا ہے] کے تحت ،  صفر کے برابر ہو گی۔اب قالب کی حتمی قیمت ہی ورونسکی ہے لہٰذا \عددی{I} پر تمام \عددی{x} کے لئے \عددی{W} صفر کے برابر ہے۔

(ب) \quad مسئلہ کریمر کو استعمال کرتے ہوئے ہم یوں بھی کہہ سکتے ہیں کہ \عددی{W=0} کی صورت میں مساوات \حوالہ{مساوات_سادہ_بلند_ورونسکی_ب} اور مساوات \حوالہ{مساوات_سادہ_بلند_ورونسکی_پ} خطی متجانس ہمزاد  الجبرائی مساوات کے نظام کا \عددی{x=x_0} پر غیر صفر حل \عددی{k^*_1} تا \عددی{k^*_n} پایا جاتا ہے جس کو استعمال کرتے ہوئے، \عددی{I} پر مساوات \حوالہ{مساوات_سادہ_بلند_خطی_ب} کا عمومی حل\عددی{y^*=k^*_1y_1+\cdots+k^*_ny_n} لکھا جا سکتا ہے۔مساوات \حوالہ{مساوات_سادہ_بلند_ورونسکی_ب} اور مساوات \حوالہ{مساوات_سادہ_بلند_ورونسکی_پ}  کے تحت  \عددی{y^*} ابتدائی شرائط \عددی{y^*(x_0)=0} تا \عددی{y^{*(n-1)}(x_0)=0} پر پورا اترتا ہے۔انہیں ابتدائی شرائط پر حل \عددی{y \equiv 0} بھی پورا اترتا ہے اور یوں مسئلہ \حوالہ{مسئلہ_سادہ_بلند_درجی_یکتا_مخصوص_حل} کے تحت، چونکہ  مساوات \حوالہ{مساوات_سادہ_بلند_ورونسکی_ب} کے عددی سر \عددی{I} پر استمراری ہیں، لہٰذا \عددی{y^*=y} ہو گا۔اس طرح \عددی{y^*=k^*_1y_1+\cdots+k^*_ny_n \equiv 0} پورے \عددی{I} پر ہو گا جس کا مطلب ہے کہ  \عددی{I} پر \عددی{y_1} تا \عددی{y_n} خطی طور تابع ہیں۔

(پ)\quad اگر \عددی{W} کی قیمت \عددی{x_0} پر صفر ہو جہاں \عددی{x_0} کھلے وقفہ \عددی{I} پر پایا جاتا ہو، تب ثبوت (ب) کے تحت خطی طور تابع ہونا ثابت ہوتا ہے اور یوں ثبوت (الف) کے تحت \عددی{W \equiv 0} ہو گا۔اس طرح اگر \عددی{I} پر نقطہ \عددی{x_1} پر \عددی{W} صفر نہ ہو تب \عددی{y_1} تا \عددی{y_n} کھلے وقفہ \عددی{I} پر خطی طور غیر تابع ہوں گے۔
\انتہا{ثبوت}
%============================

\ابتدا{مثال}\شناخت{مثال_سادہ_بلند_اساس_عمومی_حل_ب} \quad اساس۔ ورونسکی\\
ثابت کریں کہ مثال \حوالہ{مثال_سادہ_بلند_اساس_عمومی_حل_الف} میں حاصل کردہ حل \عددی{y_1=c}، \عددی{y_2=e^x} اور \عددی{y_3=e^{-x}} خطی طور غیر تابع ہیں۔

حل: مساوات \حوالہ{مساوات_سادہ_بلند_ورونسکی_الف} کے طرز پر  ورونسکی لکھ ک
\begin{align*}
W=
\begin{vmatrix}
c& e^x& e^{-x}\\
0& e^x & -e^{-x}\\
0&e^x&e^x
\end{vmatrix}
=c e^{x} e^{-x}
\begin{vmatrix}
1& 1& 1\\
0& 1 & -1\\
0&1&1
\end{vmatrix}=
c
\begin{vmatrix}
1& -1\\
1&1
\end{vmatrix}
=2c
\end{align*}
حل کیا گیا ہے جہاں پہلی قطار سے \عددی{c}، دوسری قطار سے \عددی{e^x} اور تیسری قطار سے \عددی{e^{-x}} باہر نکال کر قالب کی سادہ صورت حاصل کی گئی اور اس کے بعد پہلی قطار سے قالب کو پھیلا کر اس کی حتمی قیمت حاصل کی گئی ہے۔چونکہ \عددی{x} کی کسی بھی قیمت کے لئے  \عددی{W \ne 0} ہے لہٰذا کسی بھی کھلے وقفے پر \عددی{y_1} تا \عددی{y_3} خطی طور غیر تابع ہیں۔
\انتہا{مثال}
%========================

\جزوحصہء{مساوات \حوالہ{مساوات_سادہ_بلند_خطی_ب} کے عمومی حل میں تمام حل شامل ہیں}
