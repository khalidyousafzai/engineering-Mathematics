\باب{تکمل بذریعہ ترکیب بقیہ}
چونکہ مساوات \حوالہ{مساوات_ٹیلر_لوغوں_تسلسل_ب} کے تکمل استعمال کیے بغیر لوغوں تسلسل (مساوات \حوالہ{مساوات_ٹیلر_لوغوں_تسلسل_الف}) کے عددی سر حاصل کرنے کے کئی تراکیب پائے جاتے ہیں لہٰذا ہم \عددی{c_1} کا کلیہ استعمال کرتے ہوئے مخلوط تکمل کی قیمت کو با آسانی اور  نفاست کے ساتھ حاصل کر سکتے ہیں۔ \عددی{c_1} کو \عددی{z=a} پر \عددی{f(z)} کا \اصطلاح{بقیہ} کہا جائے گا۔جیسا ہم حصہ میں دیکھیں گے، اس طاقتور ترکیب کو استعمال کرتے ہوئے کئی اہم حقیقی تکمل بھی حل کیے جاتے ہیں۔

\حصہ{بقیہ}
تفاعل \عددی{f(z)} جو نقطہ \عددی{z=0} کی پڑوس میں تحلیلی ہو کے لئے کوشی مسئلہ تکمل سے اس پڑوس میں کسی بھی خط ارتفاع  پر 
\begin{align}\label{مساوات_بقیہ_تکمل_الف}
\int_C f(z)\dif z=0
\end{align}
ہو گا۔البتہ اگر \عددی{C} کے اندر نقطہ  \عددی{z=a} پر \عددی{f(z)} کا تنہا ندرت پایا جاتا ہو تب  مساوات \حوالہ{مساوات_بقیہ_تکمل_الف} میں دیا گیا تکمل عموماً غیر صفر ہو گا۔ایسی صورت میں \عددی{f(z)} کو لوغوں تسلسل
\begin{align}\label{مساوات_بقیہ_تکمل_ب}
f(z)=\sum\limits_{n=0}^{\infty} b_n(z-a)^n+\frac{c_1}{z-a}+\frac{c_2}{(z-a)^2}+\cdots
\end{align}
سے ظاہر کیا جا سکتا ہے جو دائرہ کار \عددی{0<\abs{z-a}<R} میں مرتکز ہو گا جہاں \عددی{a} سے \عددی{f(z)} کی قریب ترین ندرت کا فاصلہ \عددی{R} ہے۔مساوات \حوالہ{مساوات_ٹیلر_لوغوں_تسلسل_ب} سے ہم دیکھتے ہیں کہ عددی سر \عددی{c_1} درج ذیل ہو گا
\begin{align*}
c_1=\frac{1}{i2\pi}\int_C f(z)\dif z
\end{align*}
لہٰذا
\begin{align}\label{مساوات_بقیہ_تکمل_پ}
\int_C f(z)\dif z=i2\pi c_1
\end{align}
لکھا جا سکتا ہے جہاں تکمل کو گھڑی کے الٹ رخ، دائرہ کار \عددی{0<\abs{z-a}<R} میں سادہ بند راہ \عددی{C}  پر حاصل کیا جاتا ہے۔مساوات \حوالہ{مساوات_بقیہ_تکمل_ب} میں \عددی{c_1} کو نقطہ \عددی{z=a} پر \عددی{f(z)} کا \اصطلاح{بقیہ}\فرہنگ{بقیہ}\حاشیہب{residue}\فرہنگ{residue} کہتے ہیں جس کو ہم درج ذیل لکھ کر ظاہر کرتے ہیں۔
\begin{align}
c_1=\underset{z=a\hfill}{\Res f(z)}
\end{align}

ہم دیکھ چکے ہیں کہ لوغوں تسلسل کے عددی سر کو، عددی سر کی تکمل کلیات کو استعمال کیے بغیر، مختلف تراکیب سے حاصل کیا جا سکتا ہے۔ان میں سے  کسی ایک ترکیب سے \عددی{c_1} حاصل کرتے ہوئے \اصطلاح{ارتفاعی تکمل}\فرہنگ{تکمل!ارتفاعی}\حاشیہب{contour integral}\فرہنگ{integral!contour} کی قیمت حاصل کی جا سکتی ہے۔  

%========================
\ابتدا{مثال}\quad \موٹا{تکمل کی قیمت کا حصول بذریعہ بقیہ}\\
تفاعل \عددی{f(z)=z^{-4}\sin z} کا اکائی دائرے پر گھڑی کی رخ تکمل حاسل کریں۔\\
مساوات \حوالہ{مساوات_ٹیلر_تکونیاتی_تفاعل} سے ہم لوغوں تسلسل
\begin{align*}
f(z)=\frac{\sin z}{z^4}=\frac{1}{z^3}-\frac{1}{3!z}+\frac{z}{5!}-\frac{z^3}{7!+-\cdots}
\end{align*}
حاصل کرت ہیں۔ہم دیکھتے ہیں کہ \عددی{z=0} پر \عددی{f(z)} کا تین درجی قطب پایا جاتا ہے جس کا مطابقتی بقیہ \عددی{c_1=-\tfrac{1}{3!}} ہے لہٰذا مساوات \حوالہ{مساوات_بقیہ_تکمل_پ} سے درج ذیل حاصل ہو گا۔
\begin{align*}
\int_C \frac{\sin z}{z^4}\dif z=i2\pi c_1=-\frac{i\pi}{3}
\end{align*}
\انتہا{مثال}
%===========================
آگے بڑھنے سے پہلے قطب کی صورت مین بقیہ دریافت کرنے کا ایک منظم طریقہ سیکھتے ہیں۔

اگر نقطہ \عددی{z=a} پر \عددی{f(z)} کا \موٹا{سادہ قطب} پایا جاتا ہو تب تفاعل کا مطابقتی لوغوں تسلسل (مساوات \حوالہ{مساوات_بقیہ_تکمل_ب}) 
\begin{align*}
f(z)=\frac{c_1}{z-a}+b_0+b_1(z-a)+b_2(z-a)^2+\cdots\quad \quad (0<\abs{z-a}<R)
\end{align*}
ہو گا جہاں \عددی{c_1\ne 0} ہے۔دونوں اطراف کو \عددی{z-a} سے ضرب دیتے ہیں۔
\begin{align}
(z-a)f(z)=c_1+(z-a)[b_0+b_1(z-a)+\cdots]
\end{align}
اب \عددی{z\to 0} کرنے سے دایاں ہاتھ \عددی{c_1} تک پہنچتا ہے لہٰذا ہمیں درج ذیل حاصل ہو گا۔
\begin{align}\label{مساوات_بقیہ_پہلا_کلیہ}
\underset{z=a\hfill}{\Res f(z)}=c_1=\lim_{z\to a} (z-a)f(z)
\end{align}  
یہ پہلا درکار نتیجہ ہے جو سادہ قطب کی صورت میں بقیہ دیتا ہے۔

سادہ قطب کی صورت میں بقیہ کا دوسرا کلیہ حاصل کرت ہیں۔اگر \عددی{f(z)} کا نقطہ \عددی{z=a} پر سادہ قطب ہو تب ہم
\begin{align*}
 f(z)=\tfrac{p(z)}{q(z)}
\end{align*}
 لکھتے ہیں جہاں \عددی{p(z)} اور \عددی{q(z)} نقطہ \عددی{z=a} پر تحلیلی ہیں، \عددی{p(a)\ne 0} ہے اور \عددی{q(z)} کا نقطہ \عددی{z=a} پر سادہ صفر پایا جائے گا۔نتیجتاً \عددی{q(z)} کو ٹیلر تسلسل
\begin{align*}
q(z)=(z-a)q'(a)+\frac{(z-a)^2}{2!}q''(a)+\cdots
\end{align*}
کی صورت میں لکھا جا سکتا ہے۔یوں مساوات \حوالہ{مساوات_بقیہ_پہلا_کلیہ}  سے 
\begin{align*}
\underset{z=a\hfill}{\Res f(z)}=\lim_{z\to a} (z-a)\frac{p(z)}{q(q)}=\lim_{z\to a}\frac{(z-a)p(z)}{(z-a)[q'(a)+\tfrac{1}{2}(z-a)q''(a)+\cdots]}
\end{align*}
یعنی 
\begin{align}\label{مساوات_بقیہ_دوسرا_کلیہ}
\underset{z=a\hfill}{\Res f(z)}=\underset{z=a\hfill}{\Res} \frac{p(z)}{q(z)}=\frac{p(a)}{q'(a)}
\end{align}
حاصل ہو گا جو سادہ قطب کی صورت میں بقیہ حاصل کرنے کا دوسرا کلیہ ہے۔

%==============
\ابتدا{مثال}\quad \موٹا{سادہ قطب کی صورت میں بقیہ}\\
تفاعل \عددی{f(z)=\tfrac{4-3z}{z^2-z}} کا \عددی{z=0} اور \عددی{z=1} پر سادہ قطب پائے جاتے ہیں۔مساوات \حوالہ{مساوات_بقیہ_دوسرا_کلیہ}کی مدد سے درج ذیل حاصل ہوتا ہے۔
\begin{align*}
\underset{z=0\hfill}{\Res} f(z)=\big[\frac{4-3z}{2z-1}\big]_{z=0}=-4,\quad \underset{z=1\hfill}{\Res} f(z) =\big[\frac{4-3z}{2z-1}\big]_{z=1}=1
\end{align*}
\انتہا{مثال}
%===================

آئیں اب \اصطلاح{بلند درجی قطبین}\فرہنگ{قطب!بلند درجی} کی بات کرتے ہیں۔اگر نقطہ \عددی{z=a} پر \عددی{f(z)} کے قطب کا درجہ \عددی{m>1} ہو تب  تفاعل کا لوغوں تسلسل
\begin{align*}
f(z)=\frac{c_m}{(z-a)^m}+\frac{c_{m-1}}{(z-a)^{m-1}}+\cdots+\frac{c_2}{(z-a)^2}+\frac{c_1}{z-a}
+b_0+b_1(z-a)+\cdots
\end{align*}
ہو گا جہاں \عددی{c_m \ne 0} ہے اور نقطہ \عددی{z=a} کی پڑوس میں، ماسوائے نقطہ \عددی{z=a} پر، تسلسل مرتکز ہو گا۔ دونوں اطراف کو \عددی{(z-a)^m} سے ضرب دیتے ہوئے 
\begin{multline*}
(z-a)^m f(z)=c_m+c_{m-1}(z-a)+\cdots+c_2(z-a)^{m-2}+c_1(z-a)^{m-1}\\
+b_0(z-a)^m+b_1(z-a)^{m+1}+\cdots
\end{multline*}
ملتا ہے۔یوں نقطہ \عددی{z=a} پر \عددی{f(z)} کا بقیہ \عددی{c_1} اب تفاعل \عددی{g(z)=(z-a)^mf(z)} کا \عددی{z=a} کے گرد  ٹیلر تسلسل میں \عددی{(z-a)^{m-1}} کا عددی سر ہے۔یوں مسئلہ ٹیلر (مسئلہ \حوالہ{مسئلہ_ٹیلر_مسئلہ_ٹیلر}) کے تحت درج ذیل ہو گا۔
\begin{align*}
c_1=\frac{g^{(m-1)}(a)}{(m-1)!}
\end{align*}
یوں اگر نقطہ \عددی{z=a} پر \عددی{f(z)} کے قطب کا درجہ \عددی{m} ہو تب بقیہ درج ذیل (تیسرا) کلیہ دے گا۔
\begin{align}\label{مساوات_بقیہ_تیسرا_کلیہ}
\underset{z=a\hfill}{\Res} f(z)=\frac{1}{(m-1)!}\lim_{z\to a}\big\{\frac{\dif^{\, m-1}}{\dif z^{m-1}}[(z-a)^mf(z)]\big\}
\end{align}

%===================
\ابتدا{مثال}\quad \موٹا{بلند درجہ قطب پر بقیہ}\\
تفاعل
\begin{align*}
f(z)=\frac{2z}{(z+4)(z-1)^2}
\end{align*}
کا \عددی{z=1} پر دو درجی قطب پایا جاتا ہے۔یوں مساوات \حوالہ{مساوات_بقیہ_تیسرا_کلیہ} درج ذیل بقیہ دے گا۔
\begin{align*}
\underset{z=1\hfill}{\Res} f(z)=\lim_{z=1}\frac{\dif}{\dif z}[(z-1)^2 f(z)]=\lim_{z=1}\frac{\dif}{\dif z}\big(\frac{2z}{z+4}\big)=\frac{8}{25}
\end{align*}
\انتہا{مثال}
%=======================
ظاہر ہے کہ ناطق تفاعل \عددی{f(z)} کی صورت میں بقیہ کو \عددی{f(z)} کی جزوی کسری پھیلاو سے بھی حاصل کیا جا سکتا ہے۔

%==================
\ابتدا{مثال}\quad
\begin{align*}
f(z)=\frac{7z^4-13z^3+z^2+4z-1}{(z^3+z^2)(z-1)^2}=\frac{3}{z}-\frac{1}{z^2}+\frac{4}{z+1}-\frac{1}{(z-1)^2}
\end{align*}
لکھتے ہوئے درج ذیل بقیہ حاصل ہوں گے۔
\begin{align*}
\underset{z=0\hfill}{\Res} f(z)=3,\quad \underset{z=-1\hfill}{\Res} f(z)=4,\quad \underset{z=1\hfill}{\Res} f(z)=0
\end{align*}
\انتہا{مثال}
%======================= 

\حصہء{سوالات}
سوال \حوالہ{سوال_بقیہ_ندرت_پر_تلاش_الف} تا سوال \حوالہ{سوال_بقیہ_ندرت_پر_تلاش_ب} میں دیے تفاعل کا ندرت پر بقیہ تلاش کریں۔

%==================
\ابتدا{سوال}\شناخت{سوال_بقیہ_ندرت_پر_تلاش_الف}\quad
$\tfrac{1}{1-z}$\\
جواب:\quad
نقطہ \عددی{z=1} پر بقیہ \عددی{-1} ہے۔
\انتہا{سوال}
%==================
\ابتدا{سوال}\quad
$\tfrac{z-3}{z+1}$\\
جواب:\quad
نقطہ \عددی{z=-1} پر بقیہ \عددی{-4} ہے۔
\انتہا{سوال}
%==================
\ابتدا{سوال}\quad
$\tfrac{1}{z^2}$\\
جواب:\quad
نقطہ \عددی{z=0} پر بقیہ \عددی{0} ہے۔
\انتہا{سوال}
%==================
\ابتدا{سوال}\quad
$\tfrac{z}{z^2-1}$\\
جواب:\quad
نقطہ \عددی{z=1} اور \عددی{z=-1} پر بقیہ بالترتیب  \عددی{\tfrac{1}{2}} اور \عددی{\tfrac{1}{2}} ہیں۔
\انتہا{سوال}
%==================
\ابتدا{سوال}\quad
$\tfrac{1}{z^2+1}$\\
جواب:\quad
نقطہ \عددی{z=-i} اور \عددی{z=i} پر بقیہ بالترتیب  \عددی{\tfrac{i}{2}} اور \عددی{\tfrac{-}{2}-} ہیں۔
\انتہا{سوال}
%==================
\ابتدا{سوال}\quad
$\tfrac{1}{(z^2+1)^2}$\\
جواب:\quad
نقطہ \عددی{z=-i} اور \عددی{z=i} پر بقیہ بالترتیب  \عددی{\tfrac{i}{4}} اور \عددی{\tfrac{i}{4}-} ہیں۔
\انتہا{سوال}
%==================
\ابتدا{سوال}\quad
$\tfrac{1}{(z^2-1)^2}$\\
جواب:\quad
نقطہ \عددی{z=-1} اور \عددی{z=1} پر بقیہ بالترتیب  \عددی{\tfrac{1}{4}} اور \عددی{\tfrac{1}{4}-} ہیں۔
\انتہا{سوال}
%==================
\ابتدا{سوال}\quad
$\tfrac{z}{z^4-1}$\\
جواب:\quad
نقطہ \عددی{z=-1,1,-i,i} پر بقیہ اسی ترتیب سے  \عددی{\tfrac{1}{4}, \tfrac{1}{4},-\tfrac{1}{4},-\tfrac{1}{4}} ہیں۔
\انتہا{سوال}
%==================
\ابتدا{سوال}\quad
$\tfrac{1}{z^4-1}$\\
جواب:\quad
نقطہ \عددی{z=-1,1,-i,i} پر بقیہ اسی ترتیب سے  \عددی{-\tfrac{1}{4}, \tfrac{1}{4},-\tfrac{i}{4},\tfrac{i}{4}} ہیں۔
\انتہا{سوال}
%==================
\ابتدا{سوال}\quad
$\tfrac{1}{1-e^z}$\\
جواب:\quad
نقطہ \عددی{z=\mp i2n\pi} پر بقیہ \عددی{-1} ہے۔
\انتہا{سوال}
%==================
\ابتدا{سوال}\quad
$\sec z$\\
جواب:\quad 
نقطہ \عددی{z=\tfrac{\pi}{2}+2n\pi} اور \عددی{z=-\tfrac{\pi}{2}-2n\pi} پر بقیہ بالترتیب \عددی{-1} اور \عددی{1} ہے جہاں \عددی{n=0,1,2\cdots} ہے۔
\انتہا{سوال}
%=====================
\ابتدا{سوال}\quad
$\tan z$\\
جواب:\quad 
نقطہ \عددی{z=\tfrac{\pi}{2}+n\pi} پر بقیہ \عددی{-1}  ہے جہاں \عددی{n=\mp 1,\mp 2,\cdots} ہے۔
\انتہا{سوال}
%=====================
\ابتدا{سوال}\شناخت{سوال_بقیہ_ندرت_پر_تلاش_ب}\quad
$\cot z$\\
جواب:\quad 
نقطہ \عددی{z=\mp n\pi} پر بقیہ \عددی{1}  ہے۔
\انتہا{سوال}
%=====================
سوال \حوالہ{سوال_بقیہ_دائرہ_الف} تا سوال \حوالہ{سوال_بقیہ_دائرہ_ب} میں دائرہ \عددی{\abs{z}=1.5} کے اندر ندرت پر تفاعل کا بقیہ تلاش کریں۔

%===================
\ابتدا{سوال}\شناخت{سوال_بقیہ_دائرہ_الف}\quad
$\tfrac{3z^2}{1-z^4}$\\
جواب:
نقطہ \عددی{z=-1,1,-i,i} پر بقیہ اسی ترتیب سے  \عددی{\tfrac{3}{4}, i\tfrac{3}{4}} ہیں۔
\انتہا{سوال}
%=====================
\ابتدا{سوال}\quad
$\tfrac{z-\tfrac{3}{4}}{z^2-3z+2}$\\
جواب:
نقطہ \عددی{z=1} پر بقیہ  \عددی{-\tfrac{1}{4}} ہے۔
\انتہا{سوال}
%=====================
\ابتدا{سوال}\quad
$\tfrac{6z+1}{z^2-3z}$\\
جواب:
نقطہ \عددی{z=0} پر بقیہ  \عددی{-\tfrac{1}{3}} ہے۔
\انتہا{سوال}
%=====================
\ابتدا{سوال}\quad
$\tfrac{z-1}{(z+1)(z^2+16)}$\\
جواب:
نقطہ \عددی{z=-1} پر بقیہ  \عددی{-\tfrac{2}{17}} ہے۔
\انتہا{سوال}
%=====================
\ابتدا{سوال}\شناخت{سوال_بقیہ_دائرہ_ب}\quad
$\tfrac{4+3z}{z^3-3z^2+2z}$\\
جواب:
نقطہ \عددی{z=0,1} پر اسی ترتیب سے بقیہ  \عددی{2,-7} ہیں۔
\انتہا{سوال}
%=====================
سوال \حوالہ{سوال_بقیہ_تکمل_اکائی_دائرہ_الف} تا سوال \حوالہ{سوال_بقیہ_تکمل_اکائی_دائرہ_ب} میں اکائی دائرے پر گھڑی کی الٹ رخ تکمل کی قیمت تلاش کریں۔

%===============
\ابتدا{سوال}\شناخت{سوال_بقیہ_تکمل_اکائی_دائرہ_الف}\quad
$\int_C e^{\tfrac{1}{z}}\dif z$\\
جواب:\quad
$i2\pi$

\انتہا{سوال}
%======================
\ابتدا{سوال}\quad
$\int_C ze^{\tfrac{1}{z}}\dif z$
\انتہا{سوال}
%======================
\ابتدا{سوال}\quad
$\int_C \cot z\dif z$\\
جواب:\quad
$i2\pi$
\انتہا{سوال}
%======================
\ابتدا{سوال}\quad
$\int_C \tan z\dif z$
\انتہا{سوال}
%======================
\ابتدا{سوال}\quad
$\int_C \tfrac{\dif z}{\sin z}$\\
جواب:\quad
$i2\pi$
\انتہا{سوال}
%======================
\ابتدا{سوال}\quad
$\int_C \tfrac{z}{2z+i}\dif z$
\انتہا{سوال}
%======================
\ابتدا{سوال}\quad
$\int_C \tfrac{\dif z}{\cosh z}$\\
جواب:\quad
$0$
\انتہا{سوال}
%======================
\ابتدا{سوال}\quad
$\int_C \tfrac{z^2-4}{(z-2)^4}\dif z$
\انتہا{سوال}
%======================
\ابتدا{سوال}\quad
$\int_C \tfrac{z^2+1}{z^2-2z}\dif z$\\
جواب:\quad
$-i\pi$
\انتہا{سوال}
%======================
\ابتدا{سوال}\quad
$\int_C \tfrac{\sin \pi z}{z^4}\dif z$$-i\pi$
\انتہا{سوال}
%======================
\ابتدا{سوال}\quad
$\int_C \tfrac{\dif z}{1-e^z}\dif z$\\
جواب:\quad
$-i2\pi$
\انتہا{سوال}
%======================
\ابتدا{سوال}\شناخت{سوال_بقیہ_تکمل_اکائی_دائرہ_ب}\quad
$\int_C \tfrac{z^2+1}{e^z\sin z}\dif z$
\انتہا{سوال}
%======================

\حصہ{مسئلہ بقیہ}

