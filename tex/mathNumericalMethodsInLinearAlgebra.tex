\باب{خطی الجبرا کے اعدادی تراکیب}
اس باب میں ہم خطی الجبرائی مساوات کے نظام کے حل، مناسب سیدھی لکیروں کا حصول اور قالبی امتیازی اقدار کے حصول کے  اہم ترین تراکیب پر غور کریں گے۔یہ تراکیب اور اس سے ملتے جلتے تراکیب عملاً انتہائی اہم ثابت ہوتے ہیں جو انجینئری یا دیگر شعبوں (مثلاً شماریات) کے مسائل حل کرنے میں کام آتے ہیں۔

\حصہ{خطی مساوات کا نظام۔ گاوسی اسقاط، معکوس قالب}
\عددی{n} نا معلوم متغیرات  \عددی{x_1,\cdots,x_n} کے \عددی{m} خطی مساوات کے نظام (یا \عددی{m}  ہمزاد خطی مساوات) سے مراد درج ذیل روپ کی مساوات
\begin{gather}
\begin{aligned}\label{مساوات_خطی_اعدادی_قالبی_نظام_الف}
a_{11}x_1+\cdots+a_{1n}x_n&=b_1\\
a_{21}x_1+\cdots+a_{2n}x_n&=b_2\\
&\vdots\\
a_{m1}x_1+\cdots+a_{mn}x_n&=b_m
\end{aligned}
\end{gather}  
کا سلسلہ ہے  جہاں عددی سر \عددی{a_{jk}} اور \عددی{b_j} معلوم اعداد ہیں۔تمام \عددی{b_j} صفر ہونے کی صورت میں یہ نظام \اصطلاح{متجانس}\فرہنگ{متجانس}\حاشیہب{homogeneous}\فرہنگ{homogeneous} کہلاتا ہے ورنہ اس کو \اصطلاح{غیر متجانس}\فرہنگ{متجانس!غیر}\حاشیہب{nonhomogeneous}\فرہنگ{nonhomogeneous} کہتے ہیں۔اگر آپ قالبی ضرب (حصہ \حوالہ{حصہ_الجبرا_قالبی_ضرب}) سے آشنا ہوں  تب آپ دیکھ سکتے ہیں کہ نظام \حوالہ{مساوات_خطی_اعدادی_قالبی_نظام_الف} کو ایک سمتی مساوات
\begin{align}\label{مساوات_خطی_اعدادی_قالبی_نظام_ب}
\bM{A}\bM{x}=\bM{b}
\end{align}
لکھا جا سکتا ہے جہاں \اصطلاح{عددی سر قالب}\فرہنگ{عددی سر!قالب} \عددی{\bM{A}=[a_{ik}]} درج ذیل \عددی{m\times n} قالب ہے
\begin{align*}
\bM{A}=
\begin{bmatrix}
a_{11}&a_{12}\cdots a_{1n}\\
a_{21}&a_{22}\cdots a_{2n}\\
\vdots\\
a_{m1}&a_{m2}\cdots a_{mn}
\end{bmatrix},
\quad
\bM{x}=
\begin{bmatrix}
x_1\\
x_2\\
\vdots\\
x_n
\end{bmatrix},
\quad
\bM{b}=
\begin{bmatrix}
b_1\\
b_2\\
\vdots\\
b_m
\end{bmatrix}
\end{align*}
جبکہ \عددی{\bM{x}} اور \عددی{\bM{b}} سمتیہ قطار ہیں۔نظام \حوالہ{مساوات_خطی_اعدادی_قالبی_نظام_الف} کے حل سے مراد اعداد \عددی{x_1,\cdots,x_n} کا سلسلہ ہے جو ان تمام \عددی{m} مساوات کو مطمئن کرتے ہیں اور نظام \حوالہ{مساوات_خطی_اعدادی_قالبی_نظام_الف} کے حل سمتیہ سے مراد سمتیہ \عددی{\bM{x}} ہے جس کے اجزاء  نظام \حوالہ{مساوات_خطی_اعدادی_قالبی_نظام_الف} کے حل ہیں۔

زیادہ تعداد کی مساوات کے نظام کا حل بذریعہ قاعدہ کریمر (حصہ \حوالہ{حصہ_الجبرا_قاعدہ_کریمر})  قابل عمل نہیں ہے۔زیادہ بہتر ترکیب \اصطلاح{گاوسی اسقاط}\فرہنگ{گاوسی اسقاط} ہے جس کو ایک مثال کی مدد سے سمجھتے ہیں۔

%===============
\ابتدا{مثال}\شناخت{مثال_خطی_اعدادی_گاوسی_اسقاط}\quad \موٹا{گاوسی اسقاط}\\
درج ذیل نظام کو حل کریں۔
\begin{align*}
2w+x+2y+z&=6\\
6w-6x+6y+12z&=36\\
4w+3x+3y-3z&=-1\\
2w+2x-y+z&=10
\end{align*}
حل:\quad
\موٹا{پہلا قدم:} ہم پہلی مساوات کے  مضرب  کو باقی مساوات سے منفی کرتے ہوئے ان سے \عددی{w} حذف کرتے ہوئے درج ذیل حاصل کرتے ہیں۔
\begin{align*}
-9x\phantom{+4y}+9z&=18\\
x-y-5z&=-13\\
x-3y\phantom{+4z}&=4
\end{align*}
\موٹا{دوسرا قدم:} ان میں پہلی مساوات کے مضرب باقی مساوات سے منفی کرتے ہوئے ان سے \عددی{x} حذف کرتے ہوئے درج ذیل حاصل کرتے ہیں۔
\begin{align*}
-y-4z&=-11\\
-3y+z&=6
\end{align*}
\موٹا{تیسرا قدم:} ان میں پہلی مساوات کے مضرب کو باقی مساوات سے منفی کرتے ہوئے ان سے \عددی{y} حذف کرتے ہوئے درج ذیل حاصل کرتے ہیں۔
\begin{align*}
13z=39
\end{align*}
\موٹا{آخری قدم:} ہم اب واپس پر کرتے ہوئے تمام نا معلوم متغیرات حاصل کرتے ہیں۔
\begin{gather}
\begin{aligned}
13z&=39\\
-y-4\cdot3&=-11\\
-9x\phantom{+4y}+9\cdot 3&=18\\
2w+1+2\cdot(-1)+3&=6
\end{aligned}
\quad 
\begin{aligned}
z&=3\\
y&=-1\\
x&=1\\
w&=2
\end{aligned}
\end{gather} 
\انتہا{مثال}
%=======================

مثال \حوالہ{مثال_خطی_اعدادی_گاوسی_اسقاط} میں \عددی{a_{11}\ne 0} تھا۔اگر ایسا نہ ہوتا تب ہم باقی مساوات سے \عددی{w} حذف کرنے میں نا کام ہوتے۔یوں \عددی{a_{11}=0} کی صورت میں نظام میں مساوات کی ترتیب بدلی جائے گی  تا کہ نظام میں پہلی مساوات کا پہلا عددی سر غیر صفر ہو (اور ہو سکتا ہے کہ نا معلوم متغیرات کی ترتیب بھی بدلنی پڑے)۔باقی قدم پر بھی ایسا ہی کرنا پڑ سکتا ہے۔اس طرح درج ذیل ترکیب حاصل ہوتی ہے جس کی اطلاق کے بعد حاصل قیمتیں پر کرتے ہوئے تمام متغیرات حاصل کیے جاتے ہیں۔


\noindent\makebox[\linewidth]{\rule{\textwidth}{0.4pt}}
\موٹا{الخوارزمی: گاوسی اسقاط}\فرہنگ{الخوارزمی!گاوسی اسقاط}\حاشیہد{algorithm}\فرہنگ{algorithm}\\
مساوات \حوالہ{مساوات_خطی_اعدادی_قالبی_نظام_ب} میں \عددی{m=n} کی صورت میں \عددی{n\times n} قالب \عددی{\bM{A}} کے ساتھ بطور آخری صف \عددی{\bM{b}} شامل کرتے ہوئے \عددی{n\times (n+1)} قالب \عددی{\bM{B}=[b_{jk}]} حاصل ہو گا جس کے لئے گاوسی اسقاط کی \اصطلاح{الکراجی}\فرہنگ{الکراجی}\حاشیہب{algorithm}\فرہنگ{algorithm} درج ذیل ہے۔

\عددی{k=1} تا \عددی{k=n-1} \اصطلاح{کے لئے}\فرہنگ{کے لئے}  کریں:\\
ایسا کم تر \عددی{j\ge k} تلاش کریں کہ \عددی{b_{jk}\ne 0} ہو۔\\
\اصطلاح{اگر}\فرہنگ{اگر} ایسا کوئی \عددی{j} نہیں پایا جاتا ہو \اصطلاح{تب}\فرہنگ{تب} بتائیں کہ \عددی{\bM{A}} نادر ہے اور حساب \اصطلاح{روک}\فرہنگ{روک} دیں،\\
\اصطلاح{ورنہ}\فرہنگ{ورنہ} \عددی{\bM{B}} کے صف \عددی{j} اور صف \عددی{k} کے اجزاء کا آپس میں تبادلہ  کرتے ہوئے چلتے رہیں۔\\
\عددی{j=k+1} تا \عددی{j=n} \اصطلاح{کے لئے} کریں:\\
\عددی{q:\tfrac{b_{jk}}{b_{kk}}}\\
\عددی{p=k+1} تا \عددی{p=n+1} \اصطلاح{کے لئے} کریں:\\
\عددی{b_{jp}:b_{jp}-qb_{kp}}\\
\اصطلاح{اگر} \عددی{b_{nn}=0} ہو \اصطلاح{تب} بتائیں کہ \عددی{\bM{A}} نادر ہے اور حساب \اصطلاح{روک} دیں۔\\
\noindent\makebox[\linewidth]{\rule{\textwidth}{0.4pt}}
%=========================================

ہر قدم پر پہلی مساوات کے پہلی متغیر کے عددی سر کو \اصطلاح{چول عددی سر}\فرہنگ{چول!عددی سر}\حاشیہب{pivotal coefficient}\فرہنگ{pivotal coefficient} کہتے ہیں جس کا غیر صفر ہونا ضروری ہے۔اگر چول عددی سر کی قیمت کم ہو تب ہمیں مطابقتی مساوات کا بڑا مضرب باقی مساوات سے منفی کرنا ہو گا جس سے  تعداد ہندسہ خلل بڑھتے ہوئے  نتائج متاثر کرے گا۔اس سے بچنے کی ترکیب سمجھنے سے پہلے آئیں ایک مثال سے ایسا ہوتے دیکھیں۔

%=================
\ابتدا{مثال}\quad \موٹا{کم چول عددی سر سے پیدا مشکلات}\\
درج ذیل نظام 
\begin{align*}
0.0004x_1+1.402x_2&=1.406\\
0.4003x_1-1.502x_2&=2.501
\end{align*}
کا حل \عددی{x_1=10}، \عددی{x_2=1} ہے۔ہم چار ہندسی غیر مقررہ نقطہ نظام استعمال کرتے ہوئے اس کو گاوسی اسقاط سے حل کرتے ہیں۔

(الف)  پہلی مساوات کو مساوات چول لیتے ہوئے ہم اس کو \عددی{q=\tfrac{0.4003}{0.0004}=1001} سے ضرب دے کر دوسری مساوات سے منفی کر کے
\begin{align*}
-1405x_2=-1404
\end{align*}
حاصل کرتے ہیں۔یوں \عددی{x_2=\tfrac{-1404}{-1405}=0.9993} ہو گا اور یوں پہلی مساوات سے \عددی{x_1=10} کی بجائے 
\begin{align*}
x_1=\frac{1}{0.0004}(1.406-1.402\cdot 0.9993)=\frac{0.005}{0.0004}=12.5
\end{align*}
حاصل ہو گا۔اس ناکامی کی وجہ \عددی{\abs{a_{12}}} کے لحاظ سے  \عددی{\abs{a_{11}}} کی کم قیمت ہے جو \عددی{x_2} میں تعداد ہندسہ خلل کی قلیل قیمت سے \عددی{x_1} کی قیمت میں بہت زیادہ خلل پیدا کرتا ہے۔\\
(ب) آئیں اب دوسری مساوات کو چول مساوات لے کر اس کو \عددی{\tfrac{0.0004}{0.4003}=0.0009993} سے ضرب دے کر پہلی مساوات سے منفی کرتے ہوئے
\begin{align*}
1.404x_2=1.404
\end{align*}
حاصل کرتے ہیں۔یوں \عددی{x_2=1} حاصل ہو گا جس کو دوسری مساوات میں پر کرتے ہوئے \عددی{x_1=10} ملتا ہے۔یہاں \عددی{\abs{a_{22}}} کے لحاظ سے \عددی{\abs{a_{21}}} بہت کم نہیں ہے لہٰذا \عددی{x_2} میں معمولی تعداد ہندسہ خلل \عددی{x_1} کی قیمت میں بڑا خلل پیدا نہیں کرتا ہے۔یہی ہماری کامیابی کی  وجہ ہے۔یقیناً \عددی{x_2=1.002} کی صورت میں بھی دوسری مساوات سے \عددی{x_1=\tfrac{2.501+1.505}{0.4003}=10.01} حاصل ہوتا جو بہت بہتر نتیجہ  ہے۔
\انتہا{مثال}
%======================

وہ مساوات جس کے \عددی{x_1} کا عددی سر باقی مساواتوں کے \عددی{x_1} کے عددی سر سے بڑا ہو کو پہلی مساوات منتخب کرتے ہوئے اور اسی طرح دوسری قدم پر \عددی{x_2} کے لحاظ سے مساوات منتخب کرتے ہوئے نظام میں پہلی، دوسری، تیسری،\نقطے مساوات منتخب کی جا سکتی ہے۔اس عمل کو \اصطلاح{جزوی چول}\فرہنگ{چول!جزوی}\حاشیہب{partial pivoting}\فرہنگ{pivot!partial pivoting} کہتے ہیں۔ \اصطلاح{مکمل چول}\فرہنگ{چول!مکمل}\حاشیہب{total pivoting}\فرہنگ{pivot!total pivoting} میں ہم  پورے نظام میں سب سے بڑے حتمی عددی سر کو چول عددی سر لیتے ہوئے باقی مساوات میں سے اس کا مطابقتی متغیر حذف کرتے ہیں۔اگلی قدم میں اسی ترکیب کو دہراتے ہیں اور اسی طرح آخر تک چلتے ہیں۔عملاً مکمل چول کی ترکیب زیادہ مہنگی ثابت ہوتی ہے لہٰذا جزوی چول کی ترکیب ہی استعمال کی جاتی ہے۔

ہم پوری مساوات کو بڑی عدد سے ضرب دے کر کسی بھی عددی سر کی قیمت بڑھا سکتے ہیں لیکن ایسا کرنے سے نتائج پر کوئی اثر نہیں پڑتا ہے۔مساوات کو جزو ضربی سے ضرب دینے کو  \اصطلاح{تبدیلی پیما صف}\فرہنگ{تبدیلی پیما!صف}\حاشیہب{scaling}\فرہنگ{scaling} کہتے ہیں۔عملاً ہم \عددی{10} (یا کمپیوٹر کی اساس \عددی{\beta}) کی طاقت سے مساوات کو ضرب دے کر عددی سر کی سب سے بڑی حتمی قیمت کو \عددی{0.1} اور \عددی{1} (یعنی \عددی{\beta^{-1}} اور \عددی{1}) کے بیچ لاتے ہیں۔

عملاً ہم تبدیل پیما جزوی چول استعمال کرتے ہیں یعنی حذف کی \عددی{k} ویں قدم (جہاں \عددی{k=1,2,\cdots} ہو گا) میں ہم باقی میسر \عددی{n-k} مساواتوں میں سے اس کو مساوات چول منتخب کرتے ہیں جس کے متغیر \عددی{x_k} کے عددی سر اور اس مساوات میں سب سے بڑی حتمی قیمت کے عددی سر کے حاصل تقسیم کی حتمی قیمت سب سے زیادہ ہو۔

گاوسی اسقاط میں پیدا ہونے والے خلل پر اس کتاب میں غور نہیں کیا جائے گا۔   

\جزوحصہء{ترکیب گاوس میں ترمیم}
ترکیب گاوسی کے کئی ترامیم ممکن ہیں۔ہم \اصطلاح{شولسکی}\فرہنگ{شولسکی}\حاشیہد{فرانسیسی ریاضی دان اندرِ لوئی شولسکی [1875-1918]} کے ایک قاعدہ پر مبنی ترمیم پیش کرتے ہیں۔ شولسکی\حاشیہب{Cholesky}\فرہنگ{Cholesky} کا قاعدہ کہتا ہے  کہ حتمی مثبت چکور قالب \عددی{\bM{A}} کو 
\begin{align}\label{مساوات_خطی_اعدادی_شولسکی_الف}
\bM{A}=\bM{L}\bM{U}
\end{align} 
لکھا جا سکتا ہے جہاں \عددی{\bM{L}} اور \عددی{\bM{U}} بالترتیب نچلا تکونی قالب\فرہنگ{تکونی قالب!نچلا} اور بالائی تکونی قالب\فرہنگ{تکونی قالب!بالائی} ہیں۔\عددی{\bM{L}} اور \عددی{\bM{U}} عملاً یکتا ہوں گے۔ہم مساوات کو حل کیے بغیر \عددی{\bM{L}} اور \عددی{\bM{U}} کو حاصل کر سکتے ہیں (نیچے مثال دیکھیں)۔\عددی{n} متغیرات کے \عددی{n} مساوات کا نظام \عددی{\bM{A}\bM{x}=\bM{b}} حل کرنے  کے لئے ہم  مساوات \حوالہ{مساوات_خطی_اعدادی_شولسکی_الف} کا سہارا لیتے ہوئے  نظام کو
\begin{align*}
\bM{L}\bM{U}\bM{x}=\bM{b}
\end{align*}
لکھتے ہیں۔اس کو بائیں طرف \عددی{\bM{L}^{-1}} سے ضرب دے کر
\begin{align}\label{مساوات_خطی_اعدادی_شولسکی_ب}
\bM{U}\bM{x}=\bM{z}\quad \quad \bM{z}=\bM{L}^{-1}\bM{b}
\end{align}
حاصل ہو گا جو اس نظام کی تکونی صورت ہے۔ہم پہلے \عددی{\bM{z}}  کو درج ذیل تعلق
\begin{align}\label{مساوات_خطی_اعدادی_شولسکی_پ}
\bM{L}\bM{z}=\bM{b}
\end{align}
سے حاصل کر کے بعد میں 
\begin{align}\label{مساوات_خطی_اعدادی_شولسکی_ت}
\bM{U}\bM{x}=\bM{z}
\end{align}
سے \عددی{\bM{x}} حاصل کریں گے۔بہت سی اہم مسائل میں \عددی{\bM{A}} تشاکل قالب ہو گا جس کی بنا \عددی{\bM{U}=\bM{L}^T} ہو گا (درج ذیل مثال دیکھیں)۔

%=====================
\ابتدا{مثال}\quad \موٹا{ترکیب شولسکی}\\
آپ تسلی کر سکتے ہیں کہ نظام
\begin{align*}
x+2y+3z&=14\\
2x+3y+4z&=20\\
3x+4y+z&=14
\end{align*}
کا حل \عددی{x=1}، \عددی{y=2}، \عددی{z=3} ہے۔ہم اس حل کو ترکیب شولسکی سے حاصل کرتے ہیں۔عددی سر قالب تشاکلی ہے لہٰذا  \عددی{\bM{U}=\bM{L}^T} ہو گا۔ ہم ضرب قالب کی تعریف استعمال کرتے ہوئے
\begin{align*}
\begin{bmatrix}
1&2&3\\
2&3&4\\
3&4&1
\end{bmatrix}=
\begin{bmatrix}
a_{11}&0&0\\
a_{12}&a_{22}&0\\
a_{13}&a_{23}&a_{33}
\end{bmatrix}
\begin{bmatrix}
a_{11}&a_{12}&0\\
0&a_{22}&a_{23}\\
0&0&a_{33}
\end{bmatrix}
\end{align*}
کے دونوں اطراف مطابقتی اجزاء کو برابر پر کرتے ہوئے  \عددی{\bM{U}} کے اجزاء حاصل کرتے ہیں۔ایسا کرنے سے ہمیں  بالترتیب \عددی{a^2_{11}=1} مثلاً \عددی{a_{11}=1} جس سے \عددی{a_{11}a_{12}=a_{12}=2}، \عددی{a_{11}a_{13}=a_{13}=3}، \عددی{a^2_{12}+a^2_{22}=4+a^2_{22}=3} مثلاً
 \عددی{a_{22}=i\,(\sqrt{-1})} اور اس سے
\begin{align*}
a_{12}a_{13}+a_{22}a_{23}=6+ia_{23}=4,\quad a_{23}=i2
\end{align*}
اور آخر میں
\begin{align*}
a^2_{13}+a^2_{23}+a^2_{33}=9-4+a^2_{33}=1
\end{align*}
سے مثلاً \عددی{a_{33}=i2} حاصل ہو گا۔یوں مساوات \حوالہ{مساوات_خطی_اعدادی_شولسکی_پ}
\begin{align*}
\begin{bmatrix}
1&0&0\\
2&i&0\\
3&i2&i2
\end{bmatrix}
\begin{bmatrix}
z_1\\
z_2\\
z_3
\end{bmatrix}=
\begin{bmatrix}
14\\
20\\
14
\end{bmatrix}\quad\implies
\begin{bmatrix}
z_1\\
z_2\\
z_3
\end{bmatrix}=
\begin{bmatrix}
14\\
i8\\
i6
\end{bmatrix}
\end{align*}
دے گا۔آخر میں ہم مساوات \حوالہ{مساوات_خطی_اعدادی_شولسکی_ت} حل کرتے ہیں یعنی:
\begin{align*}
\begin{bmatrix}
1&2&3\\
0&i&i2\\
0&0&i2
\end{bmatrix}
\begin{bmatrix}
x_1\\
x_2\\
x_3
\end{bmatrix}=
\begin{bmatrix}
14\\
i8\\
i6
\end{bmatrix}\quad\implies
\begin{bmatrix}
x_1\\
x_2\\
x_3
\end{bmatrix}=
\begin{bmatrix}
1\\
2\\
3
\end{bmatrix}
\end{align*}

\انتہا{مثال}
%===========================

گاوسی اسقاط کی دوسری ترمیم کو \اصطلاح{گاوس جارڈن اسقاط}\فرہنگ{گاوس جارڈن اسقاط}\فرہنگ{Gauss-Jordan elimination} کہتے ہیں۔اس ترکیب میں قالب کو "تکونی صورت" کی بجائے مزید چال چلتے ہوئے  "وتری صورت" میں تبدیل کرتے ہوئے قیمتوں کے واپس پر کرنے کے عمل سے چھٹکارا حاصل کیا جاتا ہے۔ان اضافی چال کی بنا مساوات کا نظام حل کرنے میں کوئی آسانی پیدا نہیں ہوتی ہے۔البتہ معکوس قالب حاصل کرنے میں صورت حال مختلف ہے جہاں ترکیب گاوس اور ترکیب گاوس جارڈن دونوں میں \عددی{n^3} ضرب درکار ہیں۔

%==============
\جزوحصہء{معکوس قالب} 
غیر نادر چکور قالب \عددی{\bM{A}} کا معکوس اب اصولی طور پر \عددی{n} عدد نظام
\begin{align}\label{مساوات_خطی_اعدادی_شولسکی_ٹ}
\bM{A}\bM{x}=\bM{b}_j\quad \quad \quad (j=1,\cdots,n)
\end{align}
کے حل سے حاصل کیا جا سکتا ہے جہاں \عددی{n\times n} اکائی قالب کا \عددی{j} واں قطار \عددی{\bM{b}_j} ہے۔

البتہ اکائی قالب \عددی{\bM{I}} پر ترکیب گاوس جارڈن کی طرح عمل کرتے ہوئے \عددی{\bM{A}} کی تخفیف سے  \عددی{\bM{I}} حاصل کرتے ہوئے \عددی{\bM{A}^{-1}} کے حصول کو ترجیح دی جاتی ہے۔

%===================
\حصہء{سوالات}

