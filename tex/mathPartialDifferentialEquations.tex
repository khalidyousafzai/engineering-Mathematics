\باب{جزوی تفرقی مساوات}
مختلف طبعی اور جیومیٹریائی مسائل جہاں دو یا دو سے زیادہ متغیرات پر مبنی تفاعل پایا جاتے ہوں، جزوی تفرقی مساوات کو جنم دیتے ہیں۔یہ متغیرات وقت اور خلا کے محدد ہو سکتے ہیں۔اس باب میں انجینئری نقطہ نظر سے اہم مسائل پر غور کیا جائے گا۔ان مساوات کو طبعی نظام کی نمونہ کے طور پر حاصل کرنے کے بعد ابتدائی قیمت اور سرحدی قیمت مسائل حل کرنے کی تراکیب پر غور کیا جائے گا، یعنی ان مساوات کو دی گئی طبعی شرائط کے مطابق حل کیا جائے گا۔

ہم دیکھیں گے کہ جزوی تفرقی مساوات کو لاپلاس بدل کی مدد سے حل کیا جا سکتا ہے۔

\حصہ{بنیادی تصورات}
دو یا دو سے زیادہ غیر تابع متغیرات کی نا معلوم تفاعل اور اس کی ایک یا ایک سے زیادہ تفرقات پر مبنی مساوات کو \اصطلاح{جزوی تفرقی مساوات}\فرہنگ{تفرقی!جزوی مساوات}\فرہنگ{جزوی!تفرقی مساوات}\حاشیہب{partial differential equation}\فرہنگ{differential!partial equation} کہتے ہیں۔ بلند تر تفرق کا درجہ مساوت کا \اصطلاح{درجہ}\فرہنگ{درجہ!جزوی تفرقی مساوات}\فرہنگ{جزوی!درجہ مساوات}\حاشیہب{order}\فرہنگ{order!partial differential equation} کہلاتا ہے۔ 

سادہ تفرقی مساوات کی طرح اگر جزوی تفرقی مساوات میں تابع متغیر (نا معلوم تفاعل) اور اس کے تفرق کی طاقت اکائی ہو تب  یہ تفرقی مساوات \اصطلاح{خطی}\فرہنگ{خطی}\حاشیہب{linear}\فرہنگ{linear} کہلائے گی۔اگر مساوات کا ہر رکن تابع متغیرہ یا تابع متغیرہ کی تفرقات میں سے کوئی ایک تفرق ہو تب اس کو \اصطلاح{ہم جنسی}\فرہنگ{ہم جنسی!جزوی تفرقی مساوات}\حاشیہب{homogeneous}\فرہنگ{homogeneous!partial differential equation} کہیں گے ورنہ یہ \اصطلاح{غیر ہم جنسی}\فرہنگ{غیر ہم جنسی!جزوی تفرقی مساوات}\حاشیہب{non homogeneous}\فرہنگ{non homogeneous!partial differential equation} کہلائے گی۔   

%===============
\ابتدا{مثال}\quad اہم خطی دو درجی جزوی تفرقی مساوات\\
\begin{align}
&\frac{\partial^{\,2}u}{\partial t^2}=c^2\frac{\partial^{\,2}u}{\partial x^2}\label{مساوات_جزوی_الف}\quad \quad 
\text{\RL{یک بعدی مساوات موج}}\\
&\frac{\partial u}{\partial t}=c\frac{\partial^{\,2}u}{\partial x^2}\label{مساوات_جزوی_ب}\quad \quad \text{\RL{یک بعدی مساوات حرارت}}\\
&\frac{\partial^{\,2}u}{\partial x^2}+\frac{\partial^{\,2}u}{\partial y^2}=0\label{مساوات_جزوی_پ}\quad \quad \quad 
\text{\RL{دو بعدی لاپلاس مساوات}}\\
&\frac{\partial^{\,2}u}{\partial x^2}+\frac{\partial^{\,2}u}{\partial y^2}=f(x,y)\label{مساوات_جزوی_ت}\quad \quad \quad \text{\RL{دو بعدی پوئسن مساوات}}\\
&\frac{\partial^{\,2}u}{\partial x^2}+\frac{\partial^{\,2}u}{\partial y^2}+\frac{\partial^{\,2}u}{\partial z^2}=0\label{مساوات_جزوی_ٹ}\quad \quad \quad \text{\RL{تین بعدی لاپلاس مساوات}}
\end{align}
یہاں \عددی{c} مستقل ہے، \عددی{t} وقت کو ظاہر کرتی ہے جبکہ \عددی{x}، \عددی{y}، \عددی{z} کارتیسی محدد ہیں۔مساوات \حوالہ{مساوات_جزوی_ت} میں اگر \عددی{f(x,y)\ne 0} ہو تب یہ غیر ہم جنسی ہو گی۔باقی تمام مساوات ہم جنسی ہیں۔
\انتہا{مثال}
%===========================

فضا میں غیر تابع متغیرہ کی کسی خطہ  \عددی{R} میں جزوی تفرقی مساوات کے \اصطلاح{حل} سے مراد ایسا تفاعل ہے جو خود اور جس کے وہ تمام تفرقات جو اس مساوات میں پائے جاتے ہوں کسی ایسے خطے میں موجود ہوں  جس کا \عددی{R} حصہ ہو اور یہ تمام مل کر پورے خطہ \عددی{R} میں اس مساوات کو مطمئن کرتے ہوں۔(عموماً \عددی{R} کی سرحد پر اس تفاعل کا استمراری ہونا اور درکار تفرقات کا خطہ کے اندرون معین ہونے کے ساتھ ساتھ خطہ کے اندرون مساوات کو مطمئن کرنا درکار ہو گا۔)

عموماً جزوی تفرقی مساوات کے تمام حل کی تعداد بہت زیادہ ہو گی۔ مثلاً جیسا آپ تصدیق کر سکتے ہیں کہ تفاعل
\begin{align}\label{مساوات_جزوی_مثال_تفاعل}
u=x^2-y^2,\quad u=e^x\cos y,\quad u=\ln(x^2+y^2)
\end{align} 
جو ایک دوسرے سے بالکل مختلف ہیں مساوات \حوالہ{مساوات_جزوی_پ} کے حل ہیں۔ہم بعد میں دیکھیں گے کہ جزوی تفرقی مساوات کا یکتا حل حاصل کرنے کی خاطر مزید معلومات درکار ہو گی جو طبعی حالت سے حاصل ہو گی۔مثال کے طور پر کبھی کبھار سرحد کے کسی حصے پر درکار حل کی قیمت معلوم ہو گی (\اصطلاح{سرحدی شرائط}\فرہنگ{سرحدی شرائط}\حاشیہب{boundary conditions}\فرہنگ{boundary conditions}) جب کہ بعض اوقات ابتدائی لمحہ \عددی{t=0} پر حل کی قیمت معلوم ہو گی (\اصطلاح{ابتدائی شرائط}\فرہنگ{ابتدائی شرائط}\حاشیہب{initial conditions}\فرہنگ{initial conditions})۔ 

ہم جانتے ہیں کہ اگر سادہ تفرقی مساوات خطی اور ہم جنسی ہو تب اس کی معلوم حل سے مزید حل بذریعہ خطی میل حاصل کیے جا سکتے ہیں۔ جزوی تفرقی مساوات کے لئے بھی ایسا کرنا ممکن ہے جیسا درج ذیل مسئلہ کہتا ہے۔

%=========================
\ابتدا{مسئلہ}\شناخت{مسئلہ_جزوی_بنیادی}\quad بنیادی مسئلہ\\
اگر کسی خطہ \عددی{R} میں  خطی ہم جنسی جزوی تفرقی مساوات کے دو حل \عددی{u_1} اور \عددی{u_2} ہوں تب
\begin{align*}
u=c_1u_1+c_2u_2
\end{align*} 
جہاں \عددی{c_1} اور \عددی{c_2} کوئی مستقل ہیں، بھی اس خطے میں اس مساوات کا حل ہو گا۔
\انتہا{مسئلہ}
%====================

اس مسئلے کا ثبوت نہایت آسان اور مسئلہ \حوالہ{مسئلہ_دو_درجی_خطی_میل} کی ثبوت سے ملتا جلتا ہے لہٰذا یہ آپ پر چھوڑا جاتا ہے۔

%================
\حصہء{سوالات}

%==================
\ابتدا{سوال}\quad
مسئلہ \حوالہ{مسئلہ_جزوی_بنیادی} کو دو اور تین متغیرات کی دو درجی جزوی تفرقی مساوات کے لئے ثابت کریں۔
\انتہا{سوال}
%======================
\ابتدا{سوال}\quad تصدیق کریں کہ مساوات \حوالہ{مساوات_جزوی_مثال_تفاعل} میں دیے گئے تمام تفاعل مساوات \حوالہ{مساوات_جزوی_پ} کے حل ہیں۔\\
جواب:\quad \عددی{u=x^2+y^2} لیتے ہیں۔ یوں \عددی{\tfrac{\partial^{\,2}u}{\partial x^2}=2} اور \عددی{\tfrac{\partial^{\,2}u}{\partial y^2}=2} ہو گا۔انہیں مساوت \حوالہ{مساوات_جزوی_پ} میں پر کرتے ہوئے \عددی{0=0} ملتا ہے۔یوں \عددی{u} تفرقی مساوات کو مطمئن کرتا ہے۔
\انتہا{سوال}
%====================
سوال \حوالہ{سوال_جزوی_لاپلاس_مطمئن_الف} تا سوال \حوالہ{سوال_جزوی_لاپلاس_مطمئن_ب} میں تصدیق کریں کہ دیا گیا تفاعل لاپلاس مساوات کو مطمئن کرتا ہے۔

%==================
\ابتدا{سوال}\شناخت{سوال_جزوی_لاپلاس_مطمئن_الف} \quad
$u=2xy$
\انتہا{سوال}
%=====================
\ابتدا{سوال} \quad
$u=e^x\sin y$
\انتہا{سوال}
%=====================
\ابتدا{سوال} \quad
$u=\tan^{-1}\frac{y}{x}$
\انتہا{سوال}
%=====================
\ابتدا{سوال} \quad
$u=x^3-3xy^2$
\انتہا{سوال}
%=====================
\ابتدا{سوال} \quad
$u=\sin x\sinh y$
\انتہا{سوال}
%=====================
\ابتدا{سوال}\شناخت{سوال_جزوی_لاپلاس_مطمئن_ب} \quad
$u=x^4-6x^2y^2+y^4$
\انتہا{سوال}
%=====================
سوال \حوالہ{سوال_جزوی_حراری_مطمئن_الف} تا سوال \حوالہ{سوال_جزوی_حراری_مطمئن_ب} میں تصدیق کریں کہ دیا گیا تفاعل حراری مساوات \حوالہ{مساوات_جزوی_ب} کو مطمئن کرتا ہے۔

%==================
\ابتدا{سوال}\شناخت{سوال_جزوی_حراری_مطمئن_الف} \quad
$u=e^{-2t}\cos x$
\انتہا{سوال}
%======================
\ابتدا{سوال}\quad
$u=e^{-t}\sin 3x$
\انتہا{سوال}
%======================
\ابتدا{سوال}\شناخت{سوال_جزوی_حراری_مطمئن_ب}\quad
$u=e^{-4t}\cos \omega x$
\انتہا{سوال}
%======================
سوال \حوالہ{سوال_جزوی_موج_مطمئن_الف} تا سوال \حوالہ{سوال_جزوی_موج_مطمئن_ب} میں تصدیق کریں کہ دیا گیا تفاعل موج کی مساوات \حوالہ{مساوات_جزوی_الف} کو مطمئن کرتا ہے۔

%==================
\ابتدا{سوال}\شناخت{سوال_جزوی_موج_مطمئن_الف}\quad
$u=x^2+4t^2$
\انتہا{سوال}
%======================
\ابتدا{سوال}\quad
$u=x^3+3xt^2$
\انتہا{سوال}
%======================
\ابتدا{سوال}\شناخت{سوال_جزوی_موج_مطمئن_ب}\quad
$u=\sin \omega ct\sin \omega x$
\انتہا{سوال}
%======================
\ابتدا{سوال}\quad
تصدیق کریں کہ 
$u=\sqrt{x^2+y^2+z^2}$
تین بعدی لاپلاس مساوات \حوالہ{مساوات_جزوی_ٹ} کو مطمئن کرتا ہے۔
\انتہا{سوال}
%========================
\ابتدا{سوال}\quad
تصدیق کریں کہ
$u(x,y)=a\ln(x^2+y^2)+b$
دو بعدی لاپلاس مساوات \حوالہ{مساوات_جزوی_پ} کا حل ہے۔دی گئی سرحدی شرائط کے تحت دائرہ \عددی{x^2+y^2=1} پر  \عددی{u=0} اور دائرہ \عددی{x^2+y^2=9} پر \عددی{u=5} ہے۔مستقل \عددی{a} اور \عددی{b} کی ایسی قیمتیں دریافت کریں کہ \عددی{u} ان سرحدی شرائط کو مطمئن کرے۔حاصل \عددی{u} کی ترسیم کھینچیں۔
\انتہا{سوال}
%======================
\ابتدا{سوال}\quad
تصدیق کریں کہ
$u(x,t)=v(x+ct)+w(x-ct)$
موج کی مساوات \حوالہ{مساوات_جزوی_الف} کو مطمئن کرتا ہے۔یہاں \عددی{u} اور \عددی{v} دو مرتبہ قابل تفرق تفاعل ہیں۔
\انتہا{سوال}
%====================
اگر جزوی تفرقی مساوات میں صرف ایک متغیر کے ساتھ تفرقات پائے جاتے ہوں تب اس کو سادہ تفرقی مساوات تصور کر کے حل کیا جا سکتا ہے جہاں باقی متغیرات کو مستقل تصور کیا جاتا ہے۔سوال \حوالہ{سوال_جزوی_سادہ_الف} تا سوال \حوالہ{سوال_جزوی_سادہ_ب} کو حل کریں جہاں \عددی{u} کے متغیرات \عددی{x} اور \عددی{y} ہیں۔

%====================
\ابتدا{سوال}\شناخت{سوال_جزوی_سادہ_الف}\quad
$u_{xx}-u=0$\\
جواب:\quad
$u=c_1(y)e^x+c_2(y)e^{-x}$
\انتہا{سوال}
%====================
\ابتدا{سوال}\quad
$u_{y}+yu=0$\\
جواب:\quad
$u=c(x)e^{-\tfrac{y^2}{2}}$
\انتہا{سوال}
%====================
\ابتدا{سوال}\quad
$u_{yy}+9u=0$\\
جواب:\quad
$u=c_1(y)\cos 3x+c_2(y)\sin 3x$
\انتہا{سوال}
%====================
\ابتدا{سوال}\شناخت{سوال_جزوی_سادہ_ب}\quad
$u_x+2xyu=0$\\
جواب:\quad
$u=c(y)e^{-x^2y}$
\انتہا{سوال}
%====================
سوال \حوالہ{سوال_جزوی_تلاش_الف} تا سوال \حوالہ{سوال_جزوی_تلاش_ب} میں \عددی{u_x=p} لیتے ہوئے حل تلاش کریں۔

%================
\ابتدا{سوال}\شناخت{سوال_جزوی_تلاش_الف}\quad
$u_{xy}=0$\\
جواب:\quad
$u=v(x)+w(y)$
\انتہا{سوال}
%======================
\ابتدا{سوال}\quad
$u_{xy}=u_x$
\انتہا{سوال}
%=======================
\ابتدا{سوال}\quad
$u_{xy}+u_x=0$\\
جواب:\quad
$u=v(x)e^{-y}+w(y)$
\انتہا{سوال}
%=======================
\ابتدا{سوال}\شناخت{سوال_جزوی_تلاش_ب}\quad
$u_{xy}+u_x+x+y+1=0$
\انتہا{سوال}
%=======================
سوال \حوالہ{سوال_جزوی_نظام_الف} تا سوال \حوالہ{سوال_جزوی_نظام_ب} میں دیے گئے تفرقی مساوات کی نظام کے حل تلاش کریں۔ 

%=================
\ابتدا{سوال}\شناخت{سوال_جزوی_نظام_الف}\quad
$u_{xx}=0,\quad u_{yy}=0$\\
جواب:\quad
$u=axy+bx+cy+k$
\انتہا{سوال}
%==================
\ابتدا{سوال}\quad
$u_{x}=0,\quad u_{y}=0$
\انتہا{سوال}
%==================
\ابتدا{سوال}\quad
$u_{xx}=0,\quad u_{xy}=0$\\
جواب:\quad
$u=cx+g(y)$
\انتہا{سوال}
%==================
\ابتدا{سوال}\شناخت{سوال_جزوی_نظام_ب}\quad
$u_{xx}=0,\quad u_{xy}=0,\quad u_{yy}=0$
\انتہا{سوال}
%==================
\ابتدا{سوال}\quad
تصدیق کریں کہ اگر سطح \عددی{z=z(x,y)} پر منحنی \عددی{z=c} محور \عددی{x} کے متوازی سیدھے خطوط ہوں، جہاں \عددی{c} مستقل ہے، تب \عددی{z} تفرقی مساوات \عددی{z_x=0} کا حل ہو گا۔ایسی ایک مثال بھی پیش کریں۔  
\انتہا{سوال}
%=====================
\ابتدا{سوال}\quad
تصدیق کریں کہ \عددی{yz_x-xz_y=0} کا حل \عددی{z=z(x,y)} سطح گردش ہے۔اس کی مثال پیش کریں۔ (اشارہ: \عددی{x=r\cos \theta} اور \عددی{y=r\sin \theta} لے کر تفرقی مساوات کو \عددی{z_{\theta}=0} میں تبدیل کریں۔)
\انتہا{سوال}
%======================

\حصہ{نمونہ کشی: ارتعاش پذیر تار۔ یک بعدی مساوات موج}
ایک لچک دار تار کو  لمبائی \عددی{l} تک کھینچ کر سروں سے باندھا جاتا ہے۔ساکن تار کو \عددی{x} محور پر تصور کریں۔اس تار کو کسی نقطہ یا نقاط سے کھینچ کر لمحہ \عددی{t=0} پر چھوڑا دیا جاتا ہے تا کہ یہ ارتعاش کر سکے۔ہم تار کی ارتعاش معلوم کرنا چاہتے ہیں یعنی لمحہ \عددی{t>0} پر ساکن حالت سے تار کی نقطہ \عددی{x} کا انحراف \عددی{u(x,t)} جاننا چاہتے ہیں (شکل \حوالہ{شکل_جزوی_تار})۔
\begin{figure}
\centering
\begin{subfigure}{0.8\textwidth}
\centering
\begin{tikzpicture}
%\draw(0,0) grid (4,2);
%\draw[thin,gray,step=0.1](0,0) grid (6,2);
%
\draw(0,0)--(6.5,0);
\draw(0,0)--(0,2)node[right]{$u$};
%
\draw[thick](0,0)node[below]{$0$} to [out=45,in=120] (6,0)node[below]{$l$};
%
\draw[-latex](2.1,1.2)--++(17:-1.5)node[left]{$T_1$};
\draw[dashed](2.1,1.2)--(2.1,0)node[below]{$x$};
\draw[dashed](2.1,1.2)node[solid,ocirc]{}node[above]{$P$}--++(-1.5,0);
\draw([shift={(180:0.8)}]2.1,1.2) arc (180:197:0.8);
\draw(2.1,1.2)++(17:-1)node[shift={(0,0.15)}]{$\alpha$};
%
\draw[-latex](3,1.36)--++(10:1.5)node[shift={(0.2,0.1)}]{$T_2$};
\draw[dashed](3.1,1.36)--(3.1,0)node[below]{$x+\Delta x$};
\draw[dashed](3.1,1.36)node[solid,ocirc]{}node[above]{$Q$}--++(2,0);
\draw([shift={(0:0.8)}]3.1,1.36) arc (0:10:0.8);
\draw(3.9,1.43)to [out=0,in=-60]++(0,0.5)node[left]{$\beta$};
\end{tikzpicture}
\end{subfigure}
\begin{subfigure}{0.5\textwidth}
\centering
\begin{tikzpicture}
\draw[thick](0,0)coordinate(kA)to [out=20,in=-170] ++(1.5,0.4)coordinate(kB);
\draw[dashed](kA)--++(-1.5,0);
\draw[-latex](kA)node[ocirc]{}node[above]{$P$}--++(20:-1.5)node[left]{$T_1$};
\draw ([shift={(180:0.5)}]kA) arc (180:200:0.5);
\draw(kA)++(20:-1)node[shift={(0,0.2)}]{$\alpha$};
%
\draw[dashed](kB)--++(1.5,0);
\draw[-latex](kB)node[ocirc]{}node[above]{$B$}--++(10:1.5)node[shift={(0.3,0.2)}]{$T_2$};
\draw([shift={(0:1)}]kB) arc (0:10:1);
\draw(kB)++(10:1)++(0,-0.09) to [out=0,in=-60]++(0,0.5) node[above]{$\beta$};
\end{tikzpicture}
\end{subfigure}
\caption{ارتعاش پذیر تار}
\label{شکل_جزوی_تار}
\end{figure}
%
کسی بھی نظام کا ریاضی نمونہ اخذ کرتے وقت کئی ترسیلی مفروضے  فرض کیے جاتے ہیں تا کہ  حاصل مساوات ضرورت سے زیادہ پیچیدہ نہ ہوں۔ہم سادہ تفرقی مساوات کی طرح جزوی تفرقی مساوات حاصل کرتے ہوئے بھی ایسا کریں گے۔

موجودہ مسئلے میں ہم درج ذیل فرض کرتے ہیں۔

(الف) تار کی کمیت فی اکائی لمبائی یکساں ہے (ہم جنسی تار)۔ تار مکمل طور پر لچکدار ہے لہٰذا یہ مڑنے کے خلاف مزاحمت فراہم نہیں کرتا ہے۔ \\
(ب) تار کو اتنا تان کر باندھا گیا ہے کہ اس میں تناو، ثقلی قوت سے بہت زیادہ ہو۔یوں ثقلی قوت کو نظر انداز کیا جا سکتا ہے۔\\
(پ) تار سیدھی کھڑی سطح میں حرکت کرتا ہے۔تار پر کوئی بھی نقطہ اپنے ساکن مقام سے بہت کم انحراف کرتا ہے لہٰذا ہر نقطے پر تار کی انحراف اور ڈھلوان کی حتمی قیمتیں قلیل ہوں گی۔ 

%==========================

ہم توقع کر سکتے ہیں کہ یوں حاصل جزوی تفرقی مساوات کا حل \عددی{u(x,t)}،  "غیر کامل"  ہم جنسی تار جس میں ثقلی میدان سے بہت زیادہ تناو ہو  کا صحیح نقش پیش کرے گا۔  

مسئلے کی تفرقی مساوات حاصل کرنے کی خاطر ہم تار کے ایک چھوٹے ٹکڑے پر غور کرتے ہیں (شکل \حوالہ{شکل_جزوی_تار})۔چونکہ مڑنے کے خلاف تار مزاحمت فراہم نہیں کرتا ہے لہٰذا ہر نقطے پر تار میں تناو اس نقطے پر تار کا مماسی ہو گا۔فرض کریں کہ تار کے ٹکڑے  کی سروں \عددی{P} اور \عددی{Q}  پر تناو \عددی{T_1} اور \عددی{T_2} ہے۔چونکہ تار افقی حرکت نہیں کرتا ہے لہٰذا اس ٹکڑے پر تناو کا افقی جزو صفر کے برابر ہو گا۔ یوں شکل \حوالہ{شکل_جزوی_تار} کو دیکھ کر 
\begin{align*}
T_1\cos \alpha-T_2\cos \beta=0
\end{align*}
یا
\begin{align}\label{مساوات_جزوی_تار_الف}
T_1\cos \alpha=T_2\cos \beta=T=\text{مستقل}
\end{align}
لکھا جا سکتا ہے یعنی دونوں سروں پر یکساں  افقی تناو (\عددی{T}) ہو گا۔ انتصابی رخ میں \عددی{T_1} اور \عددی{T_2} کے اجزاء \عددی{-T_1\sin \alpha} اور \عددی{T_2\sin \beta} ہیں جہاں اوپر رخ تناو کو مثبت تصور کیا گیا ہے۔ نیوٹن کی دوسری قانون کے تحت ان دو قوتوں کا مجموعہ تار کے ٹکڑے کی کمیت \عددی{\rho \Delta x} ضرب  اسراع \عددی{\tfrac{\partial^{\,2} u}{\partial t^2}} کے برابر ہو گا جہاں اسراع،  \عددی{x} اور \عددی{x+\Delta x} کے مابین کسی نقطے  کی اسراع ہو گی۔ تار کی کمیت فی اکائی لمبائی \عددی{\rho} ہے جبکہ تار کے ٹکڑے کی لمبائی \عددی{\Delta x} ہے۔یوں
\begin{align}\label{مساوات_جزوی_تار_ب}
T_2\sin \beta-T_1\sin \alpha=\rho\Delta x\frac{\partial^{\,2}u}{\partial t^2}
\end{align} 
ہو گا۔اس کو مساوات \حوالہ{مساوات_جزوی_تار_الف} سے تقسیم کرتے ہیں۔
\begin{align}\label{مساوات_جزوی_تار_پ}
\frac{T_2\sin \beta}{T_2\cos \beta}-\frac{T_1\sin \alpha}{T_2\cos \alpha}=\tan \beta -\tan \alpha=\frac{\rho \Delta x}{T}\frac{\partial^{\,2}u}{\partial t^2}
\end{align}
آپ تسلی کر لیں کہ چونکہ مساوات \حوالہ{مساوات_جزوی_تار_الف} میں \عددی{T_1\cos \alpha=T_2\cos \beta=T} ہے لہٰذا مساوات \حوالہ{مساوات_جزوی_تار_ب} کو مساوات \حوالہ{مساوات_جزوی_تار_الف} سے تقسیم کرتے ہوئے کہیں \عددی{T_2\cos \beta}، کہیں \عددی{T_1\cos \alpha} اور کہیں \عددی{T} سے تقسیم کیا جا سکتا ہے۔ 

اب \عددی{\tan \beta} اور \عددی{\tan \alpha} تار کی \عددی{x} اور \عددی{x+\Delta x} پر مماس ہے یعنی
\begin{align*}
\tan \alpha=\left(\frac{\partial u}{\partial x}\right)_x \quad \text{اور} \quad \tan \beta=\left(\frac{\partial u}{\partial x}\right)_{x+\Delta x}
\end{align*}
جہاں جزوی تفرق اس لئے استعمال کیے  گئے ہیں کہ \عددی{u} متغیرہ \عددی{t} کا بھی تابع  ہے۔یوں مساوات \حوالہ{مساوات_جزوی_تار_پ} کو \عددی{\Delta x} سے تقسیم کرتے ہوئے
\begin{align*}
\frac{1}{\Delta x}\big[\left(\frac{\partial u}{\partial x}\right)_{x+\Delta x}-\left(\frac{\partial u}{\partial x}\right)_{x}\big]=\frac{\rho}{T}\frac{\partial^{\,2}u}{\partial t^2}
\end{align*}
لکھا جا سکتا ہے جس میں \عددی{\Delta x} کو صفر کے قریب تر کرتے ہوئے
\begin{align}\label{مساوات_جزوی_تار_ت}
\frac{\partial^{\,2}u}{\partial t^2}=c^2\frac{\partial^{\,2}u}{\partial x^2}\quad \quad \quad c^2=\frac{T}{\rho}
\end{align}
حاصل ہوتا ہے جس کو یک \اصطلاح{بعدی مساوات موج}\فرہنگ{موج!یک بعدی مساوات}\حاشیہب{one dimensional wave equation}\فرہنگ{wave!one dimensional equation} کہتے ہیں۔ مساوات \حوالہ{مساوات_جزوی_تار_ت} ہمارے مسئلے کی درکار جزوی تفرقی مساوات ہے جو ہم جنسی اور دو درجی ہے۔مساوات میں مستقل \عددی{\tfrac{T}{\rho}} کو \عددی{c} کی بجائے \عددی{c^2} سے ظاہر کیا گیا ہے تا کہ واضح رہے کہ یہ مثبت مستقل ہے۔اس مساوات کا حل اگلے حصے میں حاصل کیا جائے گا۔
%=================================

\حصہ{علیحدگی متغیرات (ترکیب ضرب)}
گزشتہ حصے میں ہم نے دیکھا کہ لچک دار تار کی ارتعاش کو جزوی تفرقی مساوات 
\begin{align}\label{مساوات_جزوی_مساوات_موج_الف}
\frac{\partial^{\,2}u}{\partial t^2}=c^2\frac{\partial^{\,2}u}{\partial x^2}
\end{align}
 بیان کرتی ہے جہاں \عددی{u(x,t)} تار کی انحراف ہے۔تار کی حرکت جاننے کی خاطر اس مساوات کا حل درکار ہو گا بلکہ ہمیں مساوات \حوالہ{مساوات_جزوی_مساوات_موج_الف} کا ایسا حل \عددی{u(x,t)} درکار ہے جو نظام پر لاگو شرائط کو بھی مطمئن کرے۔چونکہ تار کے دونوں سر غیر تغیر پذیر ہیں لہٰذا تمام \عددی{t} کے لئے \عددی{x=0} اور \عددی{x=l} پر سرحدی شرائط
\begin{align}\label{مساوات_جزوی_مساوات_موج_ب}
u(0,t)=0,\quad u(l,t)=0
\end{align}
لاگو ہیں۔تار کی حرکت ابتدائی انحراف (لمحہ \عددی{t=0} پر انحراف) اور ابتدائی رفتار (لمحہ \عددی{t=0} پر رفتار) پر منحصر ہو گی۔ابتدائی انحراف کو \عددی{f(x)} اور ابتدائی رفتار کو \عددی{g(x)} سے ظاہر کرتے ہوئے \اصطلاح{ابتدائی شرائط}\فرہنگ{ابتدائی شرائط}\حاشیہب{initial conditions}\فرہنگ{initial conditions} 
\begin{align}
u(x,0)&=f(x)\label{مساوات_جزوی_مساوات_موج_پ}\\
\left. \frac{\partial u}{\partial t}\right|_{t=0}&=g(x)\label{مساوات_جزوی_مساوات_موج_ت}
\end{align}
لکھی جائیں گی۔ہمیں اب مساوات \حوالہ{مساوات_جزوی_مساوات_موج_ب} کا ایسا حل چاہیے جو سرحدی شرائط مساوات \عددی{مساوات_جزوی_مساوات_موج_ب} اور ابتدائی شرائط  مساوات \حوالہ{مساوات_جزوی_مساوات_موج_پ} اور مساوات \حوالہ{مساوات_جزوی_مساوات_موج_ت} کو مطمئن کرے۔ہم درج ذیل اقدام کے ذریعہ ایسا حل تلاش کریں گے۔

پہلا قدم۔ علیحدگی متغیرات کی ترکیب سے ہم جزوی تفرقی مساوات سے دو عدد سادہ تفرقی مساوات حاصل کریں گے۔\\
دوسرا قدم۔ہم ان سادہ تفرقی مساوات کے ایسے حل تلاش کریں گے جو دی گئی سرحدی شرائط کو مطمئن کرتے ہوں۔\\
تیسرا قدم۔ حاصل حل سے ایسے حل حاصل کیے جائیں گے جو ابتدائی شرائط کو بھی مطمئن کرتے ہوں۔

ان اقدام کی تفصیل درج ذیل ہے۔

\موٹا{پہلا قدم۔} ترکیب ضرب مساوات \حوالہ{مساوات_جزوی_مساوات_موج_الف} کے حل دو عدد تفاعل کا حاصل ضرب
\begin{align}\label{مساوات_جزوی_مساوات_موج_ٹ}
u(x,t)=F(x)G(t)
\end{align}
کی روپ میں دیتا ہے جہاں ہر ایک تفاعل صرف ایک متغیرہ \عددی{x} یا \عددی{t} کا تابع ہے۔ہم جلد دیکھیں گے کہ انجینئری حساب میں اس ترکیب کے  کئی استعمال پائے جاتے ہیں۔ مساوات \حوالہ{مساوات_جزوی_مساوات_موج_ٹ} کے تفرق لیتے ہوئے
\begin{align*}
\frac{\partial^{\,2}u}{\partial t^2}=F\ddot{G}\quad \text{اور}\quad \frac{\partial^{\,2}u}{\partial x^2}=F''G
\end{align*}
ملتا ہے جہاں (\عددی{'}) سے مراد \عددی{x} کے ساتھ تفرق اور (\عددی{.}) سے مراد \عددی{t} کے ساتھ تفرق ہے۔انہیں مساوات \حوالہ{مساوات_جزوی_مساوات_موج_الف} میں پر کر کے
\begin{align*}
F\ddot{G}=c^2F''G
\end{align*}
دونوں اطراف کو \عددی{c^2FG} سے تقسیم کرنے سے
\begin{align*}
\frac{\ddot{G}}{c^2G}=\frac{F''}{F}
\end{align*}
ملتا ہے جس کا دایاں ہاتھ صرف متغیرہ \عددی{x} پر منحصر ہے جبکہ اس کا بایاں ہاتھ صرف متغیرہ \عددی{t} پر منحصر ہے۔اب \عددی{t} تبدیل کرنے سے صرف بایاں ہاتھ تبدیل ہونے کا امکان ہے لیکن اس مساوات کے تحت دونوں اطراف برابر ہیں اور دایاں ہاتھ \عددی{t} تبدیل کرنے سے ہرگز تبدیل نہیں ہوتا ہے۔اس کا مطلب ہے کہ \عددی{t} تبدیل کرنے سے بایاں ہاتھ بھی تبدیل نہیں ہوتا ہے۔اسی طرح \عددی{x} تبدیل کرنے سے صرف دایاں ہاتھ کا تبدیل ہونا ممکن ہے لیکن دونوں اطراف برابر ہیں اور \عددی{x} کی تبدیلی ہے بایاں ہاتھ ہرگز تبدیل نہیں ہوتا ہے لہٰذا \عددی{x} تبدیل کرنے سے دایاں ہاتھ بھی تبدیل نہیں ہوتا ہے۔یوں اس مساوات کے دونوں اطراف غیر تغیر پذیر ہیں لہٰذا انہیں مستقل \عددی{k} کے برابر لکھا جا سکتا ہے
\begin{align*}
\frac{\ddot{G}}{c^2G}=\frac{F''}{F}=k
\end{align*}
جس سے درج ذیل دو عدد مساوات علیحدہ علیحدہ لکھنا ممکن ہے جہاں \عددی{k} نا معلوم مستقل ہے۔ 
\begin{align}
F''-kF&=0\label{مساوات_جزوی_مساوات_موج_ث}\\
\ddot{G}-c^2kG&=0\label{مساوات_جزوی_مساوات_موج_ج}
\end{align}

\موٹا{دوسرا قدم} ہم مساوات \حوالہ{مساوات_جزوی_مساوات_موج_ث} اور مساوات \حوالہ{مساوات_جزوی_مساوات_موج_ج} کے حل \عددی{F} اور \عددی{G} حاصل کرتے ہوئے ایسا \عددی{u=FG} دریافت کرتے ہیں جو تمام \عددی{t} کے لئے سرحدی شرائط مساوات \حوالہ{مساوات_جزوی_مساوات_موج_ب} کو مطمئن کرتا ہو یعنی:
\begin{align*}
u(0,t)=F(0)G(t)=0,\quad u(l,t)=F(l)G(t)=0
\end{align*}
اب اگر درج بالا میں  \عددی{G\equiv 0} ہو تب \عددی{u\equiv 0} ہو گا جس میں ہم کوئی دلچسپی نہیں رکھتے ہیں لہٰذا \عددی{G\ne 0} ہو گا۔یوں درج بالا سے درج ذیل ملتا ہے۔
\begin{align}\label{مساوات_جزوی_مساوات_موج_چ}
\text{(الف)}\quad F(0)=0,\quad \quad \text{(ب)}\quad F(l)=0
\end{align}
اگر مساوات \حوالہ{مساوات_جزوی_مساوات_موج_ث} میں \عددی{k=0} ہو تب اس مساوات کا عمومی حل \عددی{F=ax+b} ہو گا جو مساوات \حوالہ{مساوات_جزوی_مساوات_موج_چ} کی استعمال سے \عددی{a=0}، \عددی{b-0} یعنی \عددی{F\equiv0} یا \عددی{u\equiv} دیتا ہے جو غیر دلچسپ حل ہے۔مثبت \عددی{k=\mu^2} کے لئے مساوات \حوالہ{مساوات_جزوی_مساوات_موج_ث} کا عمومی حل
\begin{align*}
F=Ae^{\mu x}+Be^{-\mu x}
\end{align*}
ہے جو مساوات \حوالہ{مساوات_جزوی_مساوات_موج_چ} کی استعمال سے \عددی{A=0}، \عددی{B=0} یعنی \عددی{F\equiv 0} یا \عددی{u\equiv =0} دیتا ہے جو غیر دلچسپ حل ہے۔یوں ہمارے پاس منفی \عددی{k=-p^2} لینا رہ جاتا ہے جس کو استعمال کرتے ہوئے مساوات \حوالہ{مساوات_جزوی_مساوات_موج_ث} کو دوبارہ لکھتے ہیں۔
\begin{align*}
F''+p^2F=0
\end{align*}
اس کا عمومی حل
\begin{align*}
F(x)=A\cos px+B\sin px
\end{align*}
ہے جو مساوات \حوالہ{مساوات_جزوی_مساوات_موج_چ}-الف کی مدد سے
\begin{align*}
F(0)=A=0
\end{align*}
لہٰذا \عددی{F=B\sin px} ہو گا جو مساوات \حوالہ{مساوات_جزوی_مساوات_موج_چ}-ب کے ساتھ مل کر
\begin{align*}
F(l)=B\sin pl=0
\end{align*}
دیتی ہے۔اب اگر \عددی{B=0} ہو تب \عددی{F\equiv 0} یعنی \عددی{u\equiv 0} ہو گا جو غیر دلچسپ حل ہے لہٰذا \عددی{B\ne 0} ہے۔اس طرح \عددی{\sin pl=0} ہو گا۔ہم جانتے ہیں کہ \عددی{\sin n\pi=0} ہوتا ہے لہٰذا یوں درج ذیل ملتا ہے جہاں \عددی{n} عدد صحیح ہے۔
\begin{align}\label{مساوات_جزوی_مساوات_موج_ح}
pl=n\pi \quad \implies \quad p=\frac{n\pi}{l}
\end{align}
ہم \عددی{B=1} منتخب کرتے ہوئے لامحدود تعداد کے حل \عددی{F(x)=F_n(x)} یعنی
\begin{align}\label{مساوات_جزوی_مساوات_موج_خ}
F_n(x)=\sin \frac{n\pi}{l}x\quad \quad \quad  n=1,2,\cdots
\end{align}
حاصل کرتے ہیں جو مساوات \حوالہ{مساوات_جزوی_مساوات_موج_چ} میں دیے گئے سرحدی شرائط کو مطمئن کرتے ہیں۔چونکہ \عددی{\sin(-\alpha)=-\sin \alpha} ہوتا ہے لہٰذا  منفی عدد صحیح \عددی{n} لینے سے  یہی حل دوبارہ ملتے ہیں پس ان کی علامت منفی ہو گی۔

اب مساوات \حوالہ{مساوات_جزوی_مساوات_موج_ح} کے تحت  \عددی{k} کی قیمت صرف \عددی{k=-p^2=-(\tfrac{n\pi}{l})^2} ممکن ہے۔\عددی{k} کی ان قیمتوں کے ساتھ  مساوات \حوالہ{مساوات_جزوی_مساوات_موج_ج} درج ذیل صورت اختیار کرتی ہے
\begin{align*}
\ddot{G}+\lambda_n^2G=0\quad \quad \lambda_n=\frac{cn\pi}{l}
\end{align*}
جس کا عمومی حل
\begin{align*}
G_n(t)=B_b\cos \lambda_nt+B^*_n\sin \lambda_n t
\end{align*}
ہے۔یوں تفاعل \عددی{u_n(x,t)=F_n(x)G_n(t)}
\begin{align}\label{مساوات_جزوی_مساوات_موج_د}
u_n(x,t)=(B_b\cos \lambda_nt+B^*_n\sin \lambda_n t)\sin \frac{n\pi}{l}x\quad \quad (n=1,2,\cdots)
\end{align}
مساوات \حوالہ{مساوات_جزوی_مساوات_موج_ج} کے ایسے حل ہیں جو  مساوات \حوالہ{مساوات_جزوی_مساوات_موج_چ} میں دی گئی سرحدی شرائط کو مطمئن کرتے ہیں۔ان تفاعل کو ارتعاش پذیر تار کے  \اصطلاح{آئگنی تفاعل}\فرہنگ{آئگنی!تفاعل}\حاشیہب{eigenfunctions}\فرہنگ{eigenfunctions} یا \اصطلاح{امتیازی تفاعل}\فرہنگ{امتیازی!تفاعل}\حاشیہب{characteristic functions}\فرہنگ{characteristic!functions} کہتے ہیں جبکہ \عددی{\lambda_n=\tfrac{cn\pi}{l}} کی قیمتوں کو ارتعاش پذیر تار کے \اصطلاح{آئگنی اقدار}\فرہنگ{آئگنی!قدر}\حاشیہب{eigenvalues}\فرہنگ{eigenvalues} یا \اصطلاح{امتیازی اقدار}\فرہنگ{امتیازی!قدر}\حاشیہب{characteristic values}\فرہنگ{characteristic!values} کہتے ہیں۔ مزید \عددی{\{   \lambda_1,\lambda_2,\cdots\}} کا سلسلہ \اصطلاح{طیف}\فرہنگ{طیف}\حاشیہب{spectrum}\فرہنگ{spectrum} کہلاتا ہے۔

ہم دیکھتے ہیں کہ ہر ایک \عددی{u_n} ایک مخصوص ہارمونی ارتعاش کو ظاہر کرتی ہے جس کی تعدد \عددی{\tfrac{\lambda_n}{2\pi}=\tfrac{cn}{2l}} چکر فی اکائی وقت ہے۔ اس حرکت کو تار کی \عددی{n} ویں \اصطلاح{عمودی انداز}\فرہنگ{عمودی انداز}\فرہنگ{انداز!عمودی}\حاشیہب{normal mode}\فرہنگ{normal mode} کہتے ہیں۔ پہلا عمودی انداز جس کا \عددی{n=1} ہو گا \اصطلاح{بنیادی انداز}\فرہنگ{بنیادی انداز}\فرہنگ{انداز!بنیادی}\حاشیہب{fundamental mode}\فرہنگ{fundamental mode} کہلاتا ہے جبکہ باقی کو \عددی{n} ویں \اصطلاح{ہارمونی انداز}\فرہنگ{ہارمونی!انداز}\حاشیہب{harmonics}\فرہنگ{harmonics} کہتے ہیں۔چونکہ مساوات \حوالہ{مساوات_جزوی_مساوات_موج_د} میں 
\begin{align*}
\sin \frac{n\pi x}{l}=0\quad \implies \quad x=\frac{l}{n},\frac{2l}{n},\cdots,\frac{n-1}{n}l
\end{align*}
ہے لہٰذا \عددی{n} ویں عمودی انداز کے \عددی{n-1} \اصطلاح{نقطہ صفر ہٹاو}\فرہنگ{نقطہ صفر ہٹاو}\حاشیہب{node}\فرہنگ{node} پائے جائیں گے۔ ان نقطوں پر تار ساکن رہتی ہے۔ 
