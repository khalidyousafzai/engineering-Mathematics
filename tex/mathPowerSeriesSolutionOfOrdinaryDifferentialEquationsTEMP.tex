\باب{قائمہ الزاویہ تفاعل کا سلسلہ}
لیژانڈر تفاعل (حصہ \حوالہ{حصہ_لیژانڈر_تفاعل}) اور بیسل تفاعل کی ایک خاصیت جسے \اصطلاح{قائمیت}\فرہنگ{قائمیت}\حاشیہب{orthogonality}\فرہنگ{orthogonality} کہتے ہیں انجینئری حساب میں نمایاں کردار ادا کرتی ہے۔اس حصے میں قائمیت سے وابستہ تصورات اور  علامت نویسی سیکھتے ہیں۔اگلے حصے میں  ایسی سرحدی قیمت مسائل (سٹیورم لیوویل مسائل) پر غور کیا جائے گا جن کے حل قائمہ الزاویہ تفاعل کا سلسلہ دیتے ہیں۔ان مسائل  پر غور کے دوران حاصل نتائج کو استعمال کرتے ہوئے لیژانڈر تفاعل اور بیسل تفاعل پر غور کیا جائے گا۔

آئیں پہلے تفاعل کی قائمیت کی تعریف پیش کرتے ہیں۔فرض کریں کہ وقفہ \عددی{a\le x\le b} پر حقیقی قیمت تفاعل \عددی{g_m(x)} اور \عددی{g_n(x)} معین ہیں اور اس وقفے پر ان تفاعل کے حاصل ضرب \عددی{g_m(x)g_n(x)} کا تکمل موجود ہے۔اس تکمل کو روایتی طور پر \عددی{(g_m,g_n)} لکھا جاتا ہے۔
\begin{align}
(g_m,g_n)=\int_{a}^{b} g_m(x)g_n(x)\dif x
\end{align}
اگر درج بالا تکمل صفر کے برابر ہو تب تفاعل \عددی{g_m(x)} اور \عددی{g_n(x)} وقفہ \عددی{a\le x\le b} پر  \اصطلاح{قائمہ الزاویہ}\فرہنگ{قائمہ الزاویہ}\حاشیہب{orthogonal}\فرہنگ{orthogonal} کہلاتے ہیں۔
\begin{align}
(g_m,g_n)=\int_{a}^{b} g_m(x)g_n(x)\dif x=0\quad \quad (m \ne n)
\end{align}
حقیقی قیمت تفاعل کا سلسلہ \عددی{g_1(x)}، \عددی{g_2(x)}، \عددی{g_3(x)}، \نقطے اس صورت وقفہ  \عددی{a\le x\le b} پر \اصطلاح{قائمہ الزاویہ سلسلہ}\فرہنگ{قائمہ الزاویہ!سلسلہ}\حاشیہب{orthogonal set}\فرہنگ{orthogonal set}  کہلائے گا جب اس وقفے پر یہ تمام تفاعل معین اور تمام تکمل \عددی{(g_m,g_n)} موجود ہوں اور اس سلسلے میں تمام ممکنہ منفرد جوڑیوں کے یہ تکمل صفر کے برابر ہوں۔ 

\عددی{(g_m,g_m)} کے غیر صفر جذر کو \عددی{g_m} کا \اصطلاح{معیار}\فرہنگ{معیار}\حاشیہب{norm}\فرہنگ{norm} کہتے ہیں جسے عموماً \عددی{\norm{g_m}} سے ظاہر کیا جاتا ہے۔
\begin{align}
\norm{g_m}=\sqrt{(g_m,g_m)}=\sqrt{\int_{a}^{b} g_m^2(x)\dif x}
\end{align}
ہم پوری بحث کے دوران درج ذیل فرض کریں گے۔\\
\موٹا{عمومی مفروضہ:} \quad تمام تفاعل جن پر غور کیا جا رہا ہو محدود ہیں، جن تکمل پر غور کیا جا رہا ہو وہ موجود ہیں اور معیار غیر صفر ہیں۔

 ظاہر ہے کہ وقفہ \عددی{a\le x\le b} پر ایسے قائمہ الزاویہ سلسلہ \عددی{g_1}، \عددی{g_2}، \نقطے جن  میں ہر تفاعل کا معیار اکائی \عددی{(1)} ہو درج ذیل تعلقات پر پورا اترتے ہیں۔
\begin{align}
(g_m,g_n)=\int_{a}^{b}g_m(x) g_n(x)\dif x=
\begin{cases}
0 & m\ne n \quad (m=1,2,\cdots)\\
1& m=n \quad (n=1,2,\cdots)
\end{cases}
\end{align}
ایسے سلسلے کو وقفہ \عددی{a\le x\le b} پر \اصطلاح{معیاری قائمہ الزاویہ سلسلہ}\فرہنگ{معیاری قائمہ الزاویہ سلسلہ}\فرہنگ{قائمہ الزاویہ!معیاری سلسلہ}\حاشیہب{orthonormal set}\فرہنگ{orthonormal!set} کہتے ہیں۔

کسی بھی قائمہ الزاویہ سلسلے کے ہر تفاعل کو،زیر غور وقفے پر، اس تفاعل کی  معیار سے تقسیم کرتے ہوئے معیاری قائمہ الزاویہ سلسلہ حاصل کیا جا سکتا ہے۔ 

%====================
\ابتدا{مثال}\شناخت{مثال_طاقتی_اضافی_سائن_عمودیت_الف}
تفاعل  \عددی{g_m(x)=\sin mx} جہاں \عددی{m=1,2,\cdots} کا سلسلہ وقفہ \عددی{-\pi\le x\le \pi} پر قائمہ الزاویہ ہے کیونکہ ان تفاعل کے لئے درج ذیل لکھا جا سکتا ہے (ضمیمہ \حوالہ{ضمیمہ_مفید_معلومات} میں مساوات \حوالہ{مساوات_ضمیمہ_مفید_گیارہ})۔
\begin{gather}
\begin{aligned}
(g_m,g_n)&=\int_{-\pi}^{\pi}\sin mx\sin nx\dif x\quad (m\ne n)\\
&=\frac{1}{2}\int_{-\pi}^{\pi}\cos(m-n)x\dif x-\frac{1}{2}\int_{-\pi}^{\pi}\cos(m+n)\dif x=0
\end{aligned}
\end{gather}
ان تفاعل کا معیار \عددی{\norm{g_m}=\sqrt{\pi}}  ہے۔
\begin{align*}
\norm{g_m}^2=\int_{-\pi}^{\pi} \sin^2 mx\dif x=\pi\quad \quad (m=1,2,\cdots)
\end{align*} 
یوں اس سلسلے سے درج ذیل معیاری قائمہ الزاویہ سلسلہ حاصل ہوتا ہے۔
\begin{align*}
\frac{\sin x}{\sqrt{\pi}},\quad \frac{\sin 2x}{\sqrt{\pi}},\quad \frac{\sin 3x}{\sqrt{\pi}}
\end{align*} 
\انتہا{مثال}
%=====================
\ابتدا{مثال}\شناخت{مثال_طاقتی_اضافی_سائن_عمودیت_ب}
کوسائن تفاعل \عددی{\cos mx} کے سلسلے کو بھی مثال \حوالہ{مثال_طاقتی_اضافی_سائن_عمودیت_الف} کی طرح قائمہ الزاویہ ثابت کیا جا سکتا ہے۔مزید تمام \عددی{m,n=0,1,\cdots} کے لئے درج ذیل لکھا جا سکتا ہے۔
\begin{align*}
\int_{-\pi}^{\pi} \cos mx \sin nx \dif x=\frac{1}{2}\int_{-\pi}^{\pi}\sin(m+n)x\dif x-\frac{1}{2}\int_{-\pi}^{\pi}\sin(m-n)x \dif x=0
\end{align*}
یوں ظاہر ہے کہ درج ذیل سلسلہ  وقفہ \عددی{-\pi\le x\le \pi} پر قائمہ الزاویہ ہے
\begin{align*}
1,\quad \cos x,\quad \sin x,\quad \cos 2x,\quad \sin 2x,\quad \cdots
\end{align*}
جس سے درج ذیل معیاری قائمہ الزاویہ سلسلہ حاصل ہوتا ہے۔
\begin{align*}
\frac{1}{\sqrt{2\pi}},\quad \frac{\cos x}{\sqrt{\pi}},\quad \frac{\sin x}{\sqrt{\pi}},\quad \frac{\cos 2x}{\sqrt{\pi}},\quad \frac{\sin 2x}{\sqrt{\pi}},\quad \cdots
\end{align*}
\انتہا{مثال}
%=======================

قائمہ الزاویہ سلسلہ استعمال کرتے ہوئے مختلف تفاعل کو تسلسل کی صورت میں لکھا جا سکتا ہے۔فرض کریں کہ وقفہ \عددی{1\le x\le b} پر \عددی{g_1(x)}، \عددی{g_2(x)}، \نقطے کوئی بھی قائمہ الزاویہ سلسلہ ہے۔اب فرض کریں کہ \عددی{f(x)} کوئی بھی تفاعل ہے جس کو ان \عددی{g(x)} کی ایسی تسلسل
\begin{align}\label{مساوات_طاقتی_اضافی_فوریئر_عمومی_الف}
f(x)=\sum_{n=1}^{\infty} c_n g_n(x)=c_1g_1(x)+c_2g_2(x)+\cdots
\end{align}
لکھنا ممکن ہو جو \اصطلاح{مرکوز} ہو۔اس تسلسل کو \عددی{f(x)} کی \اصطلاح{عمومی فوریئر تسلسل}\فرہنگ{فوریئر تسلسل!عمومی}\حاشیہب{generalized Fourier series}\فرہنگ{Fourier series!generalized} کہتے ہیں جبکہ \عددی{c_1}، \عددی{c_2}، \نقطے کو ان قائمہ الزاویہ سلسلے کے لحاض سے تسلسل کے \اصطلاح{فوریئر مستقل}\فرہنگ{فوریئر!مستقل}\حاشیہب{Fourier constants}\فرہنگ{Fourier constants} کہتے ہیں۔ 

قائمیت کی بنا ان مستقل کو نہایت آسانی سے حاصل کیا جا سکتا ہے۔مساوات \حوالہ{مساوات_طاقتی_اضافی_فوریئر_عمومی_الف} کے دونوں اطراف کو \عددی{g_m(x)} (معین \عددی{m} ) سے ضرب دیتے ہوئے وقفہ \عددی{a\le x\le b} پر تکمل لینے سے درج ذیل ملتا ہے جہاں فرض کیا گیا ہے کہ جزو در جزو تکمل لیا جا سکتا ہے۔ 
\begin{align*}
(f,g_m)=\int_{a}^{b}fg_m\dif x=\sum_{n=1}^{\infty} c_n(g_n,g_m)=\sum_{n=1}^{\infty} c_n\int_{a}^{b} g_n g_m \dif x
\end{align*}
بائیں ہاتھ جن تکملات میں \عددی{n=m} ہو، وہ  \عددی{(g_n,g_m)=\norm{g_m}^2} کے برابر ہوں گے جبکہ قائمیت کی بنا  باقی تمام تکملات صفر کے برابر ہوں گے لہٰذا
\begin{align}\label{مساوات_طاقتی_اضافی_فوریئر_عمومی_ب}
(f,g_m)=c_m\norm{g_m}^2
\end{align}
ہو گا اور یوں فوریئر مستقل کا درج ذیل کلیہ حاصل ہوتا ہے۔
\begin{align}\label{مساوات_طاقتی_اضافی_فوریئر_عمومی_پ}
c_m=\frac{(f,g_m)}{\norm{g_m}^2}=\frac{1}{\norm{g_m}^2}\int_{a}^{b} f(x)g_m(x)\dif x \quad \quad (m=1,2,\cdots)
\end{align}
%===============
\ابتدا{مثال}\quad فوریئر تسلسل\\
مساوات \حوالہ{مساوات_طاقتی_اضافی_فوریئر_عمومی_الف} کو مثال \حوالہ{مثال_طاقتی_اضافی_سائن_عمودیت_ب} کے معیاری قائمہ الزاویہ سلسلہ کی صورت درج ذیل لکھا جا سکتا ہے
\begin{align}\label{مساوات_طاقتی_اضافی_فوریئر_عمومی_ت}
f(x)=a_0+\sum_{n=1}^{\infty}(a_n\cos nx+b_n\sin nx)
\end{align}
اور مساوات \حوالہ{مساوات_طاقتی_اضافی_فوریئر_عمومی_پ} اب درج ذیل دے گا۔
\begin{gather}
\begin{aligned}\label{مساوات_طاقتی_اضافی_فوریئر_عمومی_ٹ}
a_0&=\frac{1}{2\pi}\int_{-\pi}^{\pi} f(x)\dif x\\
a_n&=\frac{1}{\pi}\int_{-\pi}^{\pi} f(x)\cos nx \dif x\\
b_n&=\frac{1}{\pi}\int_{-\pi}^{\pi}f(x)\sin nx \dif x\quad \quad (n=1,2,\cdots)
\end{aligned}
\end{gather} 
اب اگر تسلسل \حوالہ{مساوات_طاقتی_اضافی_فوریئر_عمومی_ت} مرکوز ہو تب یہ \عددی{f(x)} کی \اصطلاح{فوریئر تسلسل} کہلائے گا اور \عددی{a_0}، \عددی{a_n}، \عددی{b_n} اس کے \اصطلاح{فوریئر عددی سر}\فرہنگ{فوریئر!عددی سر}\فرہنگ{عددی سر!فوریئر}\حاشیہب{Fourier coefficients}\فرہنگ{Fourier!coefficients} کہلائیں گے۔کلیات \حوالہ{مساوات_طاقتی_اضافی_فوریئر_عمومی_ٹ} کو ان عددی سر کے \اصطلاح{یولر کلیات}\فرہنگ{یولر!کلیات}\حاشیہب{Euler formulae}\فرہنگ{Euler!formulae} کہتے ہیں۔
\انتہا{مثال}
%============================

ایسے کئی اہم سلسلے پائے جاتے ہیں جو از خود قائمہ الزاویہ نہیں ہیں البتہ ان کے حقیقی تفاعل \عددی{g_1}، \عددی{g_2}، \نقطے  درج ذیل پر پورا اترتے ہیں جہاں \عددی{p(x)} کوئی غیر صفر تفاعل ہے۔
\begin{align}\label{مساوات_طاقتی_اضافی_عمودی_تفاعل_قدر_الف}
\int_{a}^{b} p(x) g_m(x)g_n(x)\dif x=0\quad \quad (m\ne n)
\end{align}  
ہم کہتے ہیں کہ ایسا سلسلہ وقفہ \عددی{a\le x\le b} پر \اصطلاح{تفاعل قدر}\فرہنگ{تفاعل قدر}\حاشیہب{weight function}\فرہنگ{weight function} \عددی{p(x)} کے لحاض سے قائمہ الزاویہ ہے۔\عددی{g_m} کے \اصطلاح{معیار} کی تعریف اب درج ذیل ہے۔
\begin{align}\label{مساوات_طاقتی_اضافی_معیار_قدر_تفاعل}
\norm{g_m}=\sqrt{\int_{a}^{b}p(x)g_m^2\dif x}
\end{align}
اگر سلسلے کے ہر تفاعل \عددی{g_m} کا معیار اکائی \عددی{(1)} ہو تب وقفہ \عددی{a\le x\le b} پر تفاعل قدر \عددی{p(x)} کے لحاض سے یہ سلسلہ معیاری قائمہ الزاویہ کہلائے گا۔

ہم \عددی{h_m=\sqrt{p}g_m} اور \عددی{h_n=\sqrt{p}g_n} لکھ کر مساوات \حوالہ{مساوات_طاقتی_اضافی_عمودی_تفاعل_قدر_الف} کو درج ذیل لکھ سکتے ہیں
\begin{align}
\int_{a}^{b}h_m(x)h_n(x)\dif x=0 \quad \quad (m\ne n)
\end{align}
اور یوں ظاہر ہے کہ \عددی{h_m} تفاعل قائمہ الزاویہ ہیں۔

اگر تفاعل قدر \عددی{p(x)} کے لحاض سے، وقفہ \عددی{a\le x\le b} پر تفاعل \عددی{g_1(x)}، \عددی{g_2(x)}، \نقطے  قائمہ الزاویہ ہوں اور اگر کسی تفاعل \عددی{f(x)} کو درج ذیل عمومی فوریئر تسلسل کی صورت میں لکھنا ممکن ہو (مساوات \حوالہ{مساوات_طاقتی_اضافی_فوریئر_عمومی_الف} دیکھیں)
\begin{align}\label{مساوات_ضمیمہ_اضافی_تفاعل_قدر_تسلسل_الف}
f(x)=c_1g_1(x)+c_2g_2(x)+\cdots
\end{align} 
 تب اس سلسلے کے لحاض سے فوریئر مستقل \عددی{c_1}، \عددی{c_2}،\نقطے کو بھی پہلی کی طرح حاصل کیا جا سکتا ہے بس فرق اتنا ہے کہ اب مساوات
 \حوالہ{مساوات_ضمیمہ_اضافی_تفاعل_قدر_تسلسل_الف} کے دونوں اطراف کو (\عددی{g_m} کی بجائے) \عددی{pg_m} سے ضرب دے کر آگے بڑھا جائے گا۔باقی سب کچھ پہلے کی طرح حل کرتے ہوئے درج ذیل ملتا ہے جہاں تفاعل کا معیار اب مساوات \حوالہ{مساوات_طاقتی_اضافی_معیار_قدر_تفاعل} دے گا۔ 
\begin{align}
c_m=\frac{1}{\norm{g_m}^2}\int_{a}^{b}p(x)f(x)g_m(x)\dif x\quad \quad (m=1,2,\cdots)
\end{align} 
%====================
\حصہء{سوالات}
سوال \حوالہ{سوال_طاقتی_اضافی_فوریئر_الف} تا سوال \حوالہ{سوال_طاقتی_اضافی_فوریئر_ت} میں ثابت کریں کہ دیے گئے وقفے پر دیا گیا سلسلہ قائمہ الزاویہ ہے۔معیاری قائمہ الزاویہ سلسلہ بھی دریافت کریں۔

%=============
\ابتدا{سوال}\شناخت{سوال_طاقتی_اضافی_فوریئر_الف}\quad
$1,\, \cos x,\, \cos 2x,\, \cos 3x,\, \cdots,\quad \quad 0\le x\le 2\pi$\\
جوابات:
$\frac{1}{\sqrt{2\pi}},\, \frac{\cos x}{\sqrt{\pi}},\, \frac{\cos 2x}{\sqrt{\pi}},\, \frac{\cos 3x}{\sqrt{\pi}}$
\انتہا{سوال}
%=====================
\ابتدا{سوال}\quad
$\sin x, \sin 2x, \sin 3x,\, \cdots,\quad \quad 0\le x\le \pi$\\
جوابات:
$\sqrt{\frac{2}{\pi}}\sin x, \sqrt{\frac{2}{\pi}}\sin 2x, \sqrt{\frac{2}{\pi}}\sin 3x,\, \cdots$
\انتہا{سوال}
%======================
\ابتدا{سوال}\quad
$\sin \pi x, \sin 2\pi x, \sin 3\pi x,\, \cdots,\quad \quad -1\le x\le 1$\\
جوابات:
$\sin \pi x, \sin 2\pi x, \sin 3\pi x,\, \cdots$
\انتہا{سوال}
%======================
\ابتدا{سوال}\quad
$1,\, \cos 2x,\, \cos 4x,\, \cos 6x,\, \cdots,\quad \quad 0\le x\le \pi$\\
جوابات:
$\frac{1}{\sqrt{\pi}},\, \sqrt{\frac{2}{\pi}}\cos 2x,\, \sqrt{\frac{2}{\pi}}\cos 4x,\, \sqrt{\frac{2}{\pi}}\cos 6x,\, \cdots$
\انتہا{سوال}
%=====================
\ابتدا{سوال}\شناخت{سوال_طاقتی_اضافی_فوریئر_ب}\quad
$1,\, \cos \frac{2n\pi}{T}x, \quad (n=1,2,\cdots),\quad \quad 0\le x\le T$\\
جوابات:
$\frac{1}{\sqrt{T}},\, \sqrt{\frac{2}{T}}\cos \frac{2n\pi}{T}x, $
\انتہا{سوال}
%======================
\ابتدا{سوال}\quad
$\sin \frac{2n\pi}{T}x, \quad (n=1,2,\cdots),\quad \quad 0\le x\le T$\\
جوابات:
$\sqrt{\frac{2}{T}}\sin \frac{2n\pi}{T}x, $
\انتہا{سوال}
%======================
\ابتدا{سوال}\quad (حصہ \حوالہ{حصہ_لیژانڈر_تفاعل} کے لیژانڈر تفاعل)
$P_0(x),\, P_1(x),\, P_2(x),\, \cdots,\quad \quad -1\le x\le 1$\\
جوابات:
$\frac{P_0}{\sqrt{2}},\, \sqrt{\frac{3}{2}}P_1,\,  \sqrt{\frac{5}{2}}P_2,\, \sqrt{\frac{7}{2}}P_3$
\انتہا{سوال}
%============================
\ابتدا{سوال}\quad ایسے \عددی{a_0}، \عددی{b_0}، \نقطے،\عددی{c_2} دریافت کریں کہ  وقفہ \عددی{-1\le x\le 1} پر  \عددی{g_1}، \عددی{g_2} اور \عددی{g_3} معیاری قائمہ الزاویہ ہوں۔ حاصل جواب کا لیژانڈر تفاعل کے ساتھ موازنہ کریں۔ \quad 
$g_1=a_0,\, g_2=b_0+b_1x,\, g_3=c_0+c_1x+c_2x^2$\\
\انتہا{سوال}
%===========================
\ابتدا{سوال}\شناخت{سوال_طاقتی_اضافی_فوریئر_پ}
ثابت کریں کہ اگر وقفہ \عددی{a\le x\le b} پر تفاعل \عددی{g_1(x)}، \عددی{g_2(x)}، \نقطے قائمہ الزاویہ ہوں تب وقفہ \عددی{\tfrac{a-k}{c} \le t \le \tfrac{b-k}{c}} پر تفاعل \عددی{g_1(ct+k)}، \عددی{g_2(ct+k)}، \نقطے قائمہ الزاویہ ہوں گے۔
\انتہا{سوال}
%=====================
\ابتدا{سوال}\شناخت{سوال_طاقتی_اضافی_فوریئر_ت}
سوال \حوالہ{سوال_طاقتی_اضافی_فوریئر_پ} کے نتیجے کو استعمال کرتے ہوئے سوال \حوالہ{سوال_طاقتی_اضافی_فوریئر_الف} سے سوال \حوالہ{سوال_طاقتی_اضافی_فوریئر_ب} کا نتیجہ حاصل کریں۔
\انتہا{سوال}
%=========================

\حصہ{مسئلہ سٹیورم لیوویل}
انجینئری حساب میں کئی اہم قائمہ الزاویہ سلسلوں کے تفاعل وقفہ \عددی{a\le x \le b} پر بطور درج ذیل دو درجی تفرقی مساوات کے حل سامنے آتے ہیں
\begin{align}\label{مساوات_طاقتی_سٹیورم_لیوویل_مساوات_الف}
[r(x)y']'+[q(x)+\lambda p(x)]y=0
\end{align}
جو درج ذیل شرائط پر پورا اترتے ہیں۔
\begin{gather}
\begin{aligned}\label{مساوات_طاقتی_سٹیورم_لیوویل_مساوات_ب}
\text{(الف)}\quad k_1y(a)+k_2y'(a)&=0\quad \quad (\text{\RL{\عددی{k_1} اور \عددی{k_2} بیکوقت صفر نہیں ہو سکتے ہیں}})\\
\text{(ب)}\quad l_1y(b)+l_2y'(b)&=0\quad \quad (\text{\RL{\عددی{l_1} اور \عددی{l_2} بیکوقت صفر نہیں ہو سکتے ہیں}})
\end{aligned}
\end{gather}
یہاں \عددی{\lambda} مقدار معلوم ہے جبکہ \عددی{k_1}، \عددی{k_2}، \عددی{l_1} اور \عددی{l_2} حقیقی مستقل ہیں۔

مساوات \حوالہ{مساوات_طاقتی_سٹیورم_لیوویل_مساوات_الف} کو \اصطلاح{مساوات سٹیورم لیوویل}\فرہنگ{سٹیورم لیوویل مساوات}\حاشیہب{Sturm-Liouville equation}\فرہنگ{Sturm-Liouville equation} کہتے ہیں جبکہ مساوات \حوالہ{مساوات_طاقتی_سٹیورم_لیوویل_مساوات_ب}  \اصطلاح{سرحدی شرائط}\فرہنگ{سٹیورم لیوویل!سرحدی شرائط}\فرہنگ{Sturm-Liouville!boundary conditions} کہلاتے ہیں۔
