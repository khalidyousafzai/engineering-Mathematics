\باب{قائمہ الزاویہ تفاعل کا سلسلہ}
لیژانڈر تفاعل (حصہ \حوالہ{حصہ_لیژانڈر_تفاعل}) اور بیسل تفاعل کی ایک خاصیت جسے \اصطلاح{قائمیت}\فرہنگ{قائمیت}\حاشیہب{orthogonality}\فرہنگ{orthogonality} کہتے ہیں انجینئری حساب میں نمایاں کردار ادا کرتی ہے۔اس حصے میں قائمیت سے وابستہ تصورات اور  علامت نویسی سیکھتے ہیں۔اگلے حصے میں  ایسی سرحدی قیمت مسائل (سٹیورم لیوویل مسائل) پر غور کیا جائے گا جن کے حل قائمہ الزاویہ تفاعل کا سلسلہ دیتے ہیں۔ان مسائل  پر غور کے دوران حاصل نتائج کو استعمال کرتے ہوئے لیژانڈر تفاعل اور بیسل تفاعل پر غور کیا جائے گا۔

آئیں پہلے تفاعل کی قائمیت کی تعریف پیش کرتے ہیں۔فرض کریں کہ وقفہ \عددی{a\le x\le b} پر حقیقی قیمت تفاعل \عددی{g_m(x)} اور \عددی{g_n(x)} معین ہیں اور اس وقفے پر ان تفاعل کے حاصل ضرب \عددی{g_m(x)g_n(x)} کا تکمل موجود ہے۔اس تکمل کو روایتی طور پر \عددی{(g_m,g_n)} لکھا جاتا ہے۔
\begin{align}
(g_m,g_n)=\int_{a}^{b} g_m(x)g_n(x)\dif x
\end{align}
اگر درج بالا تکمل صفر کے برابر ہو تب تفاعل \عددی{g_m(x)} اور \عددی{g_n(x)} وقفہ \عددی{a\le x\le b} پر  \اصطلاح{قائمہ الزاویہ}\فرہنگ{قائمہ الزاویہ}\حاشیہب{orthogonal}\فرہنگ{orthogonal} کہلاتے ہیں۔
\begin{align}
(g_m,g_n)=\int_{a}^{b} g_m(x)g_n(x)\dif x=0\quad \quad (m \ne n)
\end{align}
حقیقی قیمت تفاعل کا سلسلہ \عددی{g_1(x)}، \عددی{g_2(x)}، \عددی{g_3(x)}، \نقطے اس صورت وقفہ  \عددی{a\le x\le b} پر \اصطلاح{قائمہ الزاویہ سلسلہ}\فرہنگ{قائمہ الزاویہ!سلسلہ}\حاشیہب{orthogonal set}\فرہنگ{orthogonal set}  کہلائے گا جب اس وقفے پر یہ تمام تفاعل معین اور تمام تکمل \عددی{(g_m,g_n)} موجود ہوں اور اس سلسلے میں تمام ممکنہ منفرد جوڑیوں کے یہ تکمل صفر کے برابر ہوں۔ 

\عددی{(g_m,g_m)} کے غیر صفر جذر کو \عددی{g_m} کا \اصطلاح{معیار}\فرہنگ{معیار}\حاشیہب{norm}\فرہنگ{norm} کہتے ہیں جسے عموماً \عددی{\norm{g_m}} سے ظاہر کیا جاتا ہے۔
\begin{align}
\norm{g_m}=\sqrt{(g_m,g_m)}=\sqrt{\int_{a}^{b} g_m^2(x)\dif x}
\end{align}
ہم پوری بحث کے دوران درج ذیل فرض کریں گے۔\\
\موٹا{عمومی مفروضہ:} \quad تمام تفاعل جن پر غور کیا جا رہا ہو محدود ہیں، جن تکمل پر غور کیا جا رہا ہو وہ موجود ہیں اور معیار غیر صفر ہیں۔

 ظاہر ہے کہ وقفہ \عددی{a\le x\le b} پر ایسے قائمہ الزاویہ سلسلہ \عددی{g_1}، \عددی{g_2}، \نقطے جن  میں ہر تفاعل کا معیار اکائی \عددی{(1)} ہو درج ذیل تعلقات پر پورا اترتے ہیں۔
\begin{align}
(g_m,g_n)=\int_{a}^{b}g_m(x) g_n(x)\dif x=
\begin{cases}
0 & m\ne n \quad (m=1,2,\cdots)\\
1& m=n \quad (n=1,2,\cdots)
\end{cases}
\end{align}
ایسے سلسلے کو وقفہ \عددی{a\le x\le b} پر \اصطلاح{معیاری قائمہ الزاویہ سلسلہ}\فرہنگ{معیاری قائمہ الزاویہ سلسلہ}\فرہنگ{قائمہ الزاویہ!معیاری سلسلہ}\حاشیہب{orthonormal set}\فرہنگ{orthonormal!set} کہتے ہیں۔

کسی بھی قائمہ الزاویہ سلسلے کے ہر تفاعل کو،زیر غور وقفے پر، اس تفاعل کی  معیار سے تقسیم کرتے ہوئے معیاری قائمہ الزاویہ سلسلہ حاصل کیا جا سکتا ہے۔ 

%====================
\ابتدا{مثال}

\انتہا{مثال}
%=====================
