\باب{دیباچہ}
اس کتاب میں تقریباً \عددی{265} حل شدہ مثال اور \عددی{531} اشکال پائے جاتے ہیں۔اس کے علاوہ باب کے اندر \عددی{239} مشق اور باب کے آخر میں \عددی{458} سوالات دئے گئے ہیں۔تمام کے تمام سوالات کے جوابات بھی دئے گئے ہیں۔

کتاب کے آخر میں فرہنگ دیا گیا ہے۔کتاب میں کسی بھی موضوع تک جلد پہنچنے کے لئے فرہنگ کو استعمال کریں۔اردو کے علاوہ انگریزی زبان میں بھی فرہنگ دیا گیا ہے۔

یہ کتاب \تحریر{Ubuntu} استعمال کرتے ہوئے \تحریر{XeLatex} میں تشکیل دی گئی جبکہ سوالات کے جوابات \تحریر{wxMaxima} کی مدد سے حاصل کئے گئے ہیں۔

یہ کتاب درج ذیل کتاب کو سامنے رکھتے ہوئے لکھی گئی ہے۔

{
\begin{otherlanguage}{english}
Advanced Engineering Mathematics by Erwin Kreyszig
\end{otherlanguage}
}

جبکہ اردو اصطلاحات چننے میں درج ذیل لغت سے استفادہ  کیا گیا۔
{
\begin{otherlanguage}{english}
\begin{itemize}
\item
http:/\!\!/www.urduenglishdictionary.org
\item
http:/\!\!/www.nlpd.gov.pk/lughat/
\end{itemize}
\end{otherlanguage}
}
آپ سے گزارش ہے کہ اس کتاب کو زیادہ سے زیادہ طلبہ و طالبات تک پہنچائیں اور کتاب میں غلطیوں کی نشاندہی میرے  ای میل پتہ پر کریں۔میری تمام کتابوں کی مکمل \تحریر{XeLatex} معلومات

{
\begin{otherlanguage}{english}
https:/\!\!/www.github.com/khalidyousafzai
\end{otherlanguage}
}

سے حاصل کی جا سکتی ہیں جنہیں آپ مکمل اختیار کے ساتھ استعمال کر سکتے ہیں۔
\vspace{5mm}

{\raggedleft{
خالد خان یوسفزئی

23 اگست 2017}}


