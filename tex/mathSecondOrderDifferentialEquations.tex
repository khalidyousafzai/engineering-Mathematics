\باب{درجہ دوم سادہ تفرقی مساوات}
کئی اہم میکانی اور برقی مسائل کو خطی دو درجی تفرقی مساوات سے ظاہر کیا جا سکتا ہے۔خطی دو درجی تفرقی مساوات  تمام خطی تفرقی مساوات کی نمائندگی کرتا ہے۔چونکہ دو درجی مساوات کا حل نسبتاً آسان ہوتا ہے لہٰذا اس باب میں اسی پر پہلے غور کرتے ہیں۔اگلے باب کا موضوع تین درجی مساوات ہے۔

تفرقی مساوات کو خطی اور غیر خطی گروہوں میں تقسیم کیا جاتا ہے۔غیر خطی تفرقی مساوات کے حل کا حصول مشکل ثابت ہوتا ہے جبکہ خطی مساوات حل کرنے کے کئی عمدہ ترکیب پائے جاتے ہیں۔اس باب میں عمومی حل اور ابتدائی معلومات کی صورت میں جبری حل کا حصول دکھایا جائے گا۔

\حصہ{متجانس خطی دو درجی تفرقی مساوات}
یک درجی مساوات پر پہلے باب میں غور کیا گیا۔اس باب میں دو درجی مساوات پر غور کیا جائے گا۔یہ مساوات میکانی اور برقی \اصطلاح{ارتعاش}\فرہنگ{ارتعاش}\حاشیہب{oscillations}\فرہنگ{oscillations}، متحرک امواج، منتقلی حرارتی توانائی اور طبیعیات کے دیگر شعبوں میں کلیدی کردار ادا کرتے ہیں۔

ایسا دو درجی تفرقی مساوات جس کو
\begin{align}\label{مساوات_سادہ_دو_درجی_تعریف}
y''+p(x)y'+q(x)y=r(x)
\end{align}
صورت میں لکھا جا سکے \اصطلاح{خطی}\فرہنگ{خطی!دو درجی}\حاشیہب{linear}\فرہنگ{linear!second order} کہلاتا ہے ورنہ اس کو \اصطلاح{غیر خطی}\فرہنگ{غیر خطی}\حاشیہب{nonlinear}\فرہنگ{nonlinear} کہتے ہیں۔

اس مساوات کی خاصیت یہ ہے کہ اس میں \عددی{y}، \عددی{y'} اور \عددی{y''} کی طاقت اکائی ہے  یعنی تینوں خطی ہیں البتہ \عددی{p(x)}، \عددی{q(x)} اور \عددی{r(x)} متغیرہ \عددی{x} کے کوئی بھی تفاعل ہو سکتے ہیں۔دو درجی مساوات کا پہلا جزو \عددی{f(x)y''} ہونے کی صورت میں مساوات کو \عددی{f(x)} سے تقسیم کرتے ہوئے اس کو مساوات \حوالہ{مساوات_سادہ_دو_درجی_تعریف} کی \اصطلاح{معیاری صورت}\فرہنگ{معیاری صورت!دو درجی}\حاشیہب{standard form}\فرہنگ{standard form!second order} میں لکھیں جہاں  \عددی{y''} پہلا  جزو ہے۔

متجانس اور غیر متجانس دو درجی مساوات کی تعریف ہو بہو ایک درجی متجانس اور غیر متجانس مساوات کی تعریف کی طرح ہے جس پر حصہ \حوالہ{حصہ_سادہ_اول_متجانس_خطی} میں تبصرہ کیا گیا۔یقیناً \عددی{r \equiv 0} [جہاں زیر غور تمام \عددی{x} پر  \عددی{r(x)=0} ہو؛ اس کو \اصطلاح{مکمل صفر}\فرہنگ{مکمل صفر}\حاشیہب{identically zero}\فرہنگ{identically zero} پڑھیں۔] کی صورت میں مساوات \حوالہ{مساوات_سادہ_دو_درجی_تعریف} درج ذیل لکھی جائے گی 
\begin{align}\label{مساوات_سادہ_متجانس_دو_درجی_تعریف}
y''+p(x)y'+q(x)y=0
\end{align}
جو \اصطلاح{متجانس} ہے۔اگر \عددی{r(x) \not\equiv 0} ہو تب مساوات \حوالہ{مساوات_سادہ_دو_درجی_تعریف} \اصطلاح{غیر متجانس}\فرہنگ{غیر متجانس!دو درجی}\حاشیہب{nonhomogenous}\فرہنگ{nonhomogenous!second order} کہلائے گا۔

متجانس خطی تفرقی مساوات کی مثال درج ذیل ہے
\begin{align*}
xy''+2y'+y=0,\quad \text{\RL{ جو کو معیاری صورت میں لکھتے ہیں}} \quad y''+\frac{2y'}{x}+\frac{y}{x}=0
\end{align*}
جبکہ غیر متجانس خطی تفرقی مساوات کی مثال
\begin{align*}
y''+x^2y=\sec x
\end{align*}
ہے۔آخر میں غیر خطی مساوات کی دو مثال پیش کرتے ہیں۔
\begin{align*}
\left(y''\right)^3+xy=\sin x, \quad y''+xy'+4y^2=0
\end{align*}

دو درجی مساوات کے حل کی تعریف عین ایک درجی مساوات کے حل کی مانند ہے۔ تفاعل \عددی{y=h(x)} کو کھلا فاصلے \عددی{I} پر اس صورت خطی (یا غیر خطی) دو درجی تفرقی مساوات کا حل تصور کیا جاتا ہے جب اس پورے فاصلے پر \عددی{h(x)}، \عددی{h'} اور \عددی{h''}  پائے جاتے ہوں اور  تفرقی مساوات میں \عددی{y} کی جگہ \عددی{h}، \عددی{y'} کی جگہ \عددی{h'} اور \عددی{y''} کی جگہ \عددی{h''} پر کرنے سے مساوات کے دونوں اطراف بالکل یکساں صورت اختیار کرتے ہوں۔چند مثال جلد پیش کرتے ہیں۔
%==========================

\حصہء{متجانس خطی تفرقی مساوات}
اس باب کے پہلے حصے میں متجانس خطی مساوات پر غور کیا جائے گا جبکہ بقایا باب میں غیر متجانس خطی مساوات پر غور کیا جائے گا۔ 

خطی تفرقی مساوات حل کرنے کے نہایت عمدہ تراکیب پائے جاتے ہیں۔متجانس مساوات کے حل میں \اصطلاح{اصول خطیت}\فرہنگ{اصول!خطیت}\فرہنگ{خطیت!اصول}\حاشیہب{linearity principle}\فرہنگ{linearity!principle} یا \اصطلاح{اصول نفاذ}\فرہنگ{اصول!نفاذ}\فرہنگ{نفاذ!اصول}\حاشیہب{superposition principle}\فرہنگ{superposition!principle} کلیدی کردار ادا کرتا ہے جس کے تحت متجانس مساوات کے مختلف حل کو آپس میں جمع کرنے یا انہیں مستقل سے ضرب دینے سے دیگر حل حاصل کئے جا سکتے ہیں۔
%===================

\ابتدا{مثال}\quad اصول نفاذ\\
تمام \عددی{x} پر درج ذیل متجانس خطی تفرقی مساوات کے حل \عددی{y_1=\cos 2x} اور \عددی{y_2=\sin 2x} ہیں۔
\begin{align*}
y''+4y=0
\end{align*}
ان حل کی درستگی ثابت کرنے کی خاطر انہیں دیے گئے مساوات میں پر کرتے ہیں۔پہلے \عددی{y_1=\cos 2x} کو درست حل ثابت کرتے ہیں۔چونکہ \عددی{(\cos 2x)''=-4\cos 2x} کے برابر ہے لہٰذا 
\begin{align*}
y''+4y=(\cos 2x)''+4(\cos 2x)=-4\cos 2x+4\cos 2x=0
\end{align*}
ملتا ہے۔اسی طرح \عددی{y_2=\sin 2x} کو پر کرتے ہوئے 
\begin{align*}
y''+4y=(\sin 2x)''+4(\sin 2x)=-4\sin 2x+4\sin 2x=0
\end{align*}
ملتا ہے۔ہم دیے گئے حل سے نئے حل حاصل کر سکتے ہیں۔یوں ہم \عددی{\cos 2x} کو کسی مستقل مثلاً \عددی{2.73} سے ضرب دیتے ہوئے اور \عددی{\sin 2x} کو  \عددی{-1.25} سے ضرب دیتے ہوئے ان کا مجموعہ
\begin{align*}
y_3=2.73\cos 2x-1.25\sin 2x
\end{align*}
لیتے ہوئے  توقع کرتے ہیں کہ یہ بھی دیے گئے تفرقی مساوات کا حل ہو گا۔آئیں نئے حل کو تفرقی مساوات میں پر کرتے ہوئے اس کی درستگی ثابت کریں۔
\begin{align*}
y''+4y&=(2.73\cos 2x-1.25\sin 2x)''+4(2.73\cos 2x-1.25\sin 2x)\\
&=4(-2.73\cos 2x+1.25\sin 2x)+4(2.73\cos 2x-1.25\sin 2x)\\
&=0
\end{align*}
\انتہا{مثال}
%=======================

اس مثال میں ہم نے دیے گئے حل \عددی{y_1} اور \عددی{y_2} سے نیا حل 
\begin{align}
y_3=c_1 y_1+c_2 y_2, \quad \text{\RL{($c_1$\, اور \, $c_2$\,اختیاری مستقل ہیں)}}
\end{align}
حاصل کیا۔ اس کو \عددی{y_1} اور \عددی{y_2} کا \اصطلاح{خطی میل}\فرہنگ{خطی!میل}\حاشیہب{linear combination}\فرہنگ{linear!combination} کہتے ہیں۔اس مثال سے ہم \اصطلاح{مسئلہ خطی میل}\فرہنگ{مسئلہ!خطی میل} بیان کرتے ہیں جسے عموماً \اصطلاح{اصول خطیت}\فرہنگ{اصول!خطیت} یا \اصطلاح{اصول نفاذ}\فرہنگ{اصول نفاذ} کہا جاتا ہے۔

%======================================

\ابتدا{مسئلہ}\شناخت{مسئلہ_دو_درجی_خطی_میل}\quad مسئلہ خطی میل\\
کھلے فاصلہ \عددیء{I} پر متجانس خطی دو درجی تفرقی مساوات کے دو عدد حل کا خطی میل بھی \عددیء{I} پر اس مساوات کا حل ہو گا۔بالخصوص ان حل کو مستقل مقدار سے ضرب دینے سے بھی مساوات کے حل حاصل ہوتے ہیں۔
\انتہا{مسئلہ}

ثبوت: تصور کریں کہ متجانس مساوات \حوالہ{مساوات_سادہ_متجانس_دو_درجی_تعریف} کے دو حل \عددی{y_1} اور \عددی{y_2} پائے جاتے ہیں لہٰذا
\begin{gather}
\begin{aligned}\label{مساوات_درجہ_دو_خطی_میل}
y_1''+py_1'+qy_1&=0\\
y_2''+y_2'+qy_2&=0
\end{aligned}
\end{gather}
ہو گا۔خطی میل سے نیا حل \عددی{y_3=c_1 y_1+c_2 y_2} حاصل کرتے ہیں۔اس کا ایک درجی تفرق اور دو درجی تفرق درج ذیل ہیں۔
\begin{align*}
y_3' &=c_1y_1'+c_2y_2'\\
y_3'' &=c_1 y_1''+c_2y_2''
\end{align*}
\عددی{y_3}، \عددی{y_3'} اور \عددی{y_3''} کو متجانس مساوات میں پر کرتے ہیں
\begin{align*}
y_3''+py_3'+qy_3&=(c_1 y_1''+c_2y_2'')+p(c_1y_1'+c_2y_2')+q(c_1 y_1+c_2 y_2)\\
&=c_1(y_1''+py_1'+qy_1)+c_2(y_2''+py_2'+qy_2)\\
&=0
\end{align*}
جہاں مساوات \حوالہ{مساوات_درجہ_دو_خطی_میل} سے آخری قدم پر دونوں قوسین صفر کے برابر پر کئے گئے ہیں۔اس سے ثابت ہوتا ہے کہ \عددی{y_3} بھی مساوات   \حوالہ{مساوات_سادہ_متجانس_دو_درجی_تعریف} کا حل ہے۔

یہاں یاد رہے کہ مسئلہ \حوالہ{مسئلہ_دو_درجی_خطی_میل} صرف متجانس مساوات کے لئے قابل استعمال ہے۔غیر متجانس مساوات کے دیگر حل اس مسئلے سے حاصل نہیں کئے جا سکتے ہیں۔
%======================================

