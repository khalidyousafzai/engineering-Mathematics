\حصہ{تہرا تکمل۔ گاوس کا مسئلہ پھیلاو}
دہرا تکمل (حصہ \حوالہ{حصہ_خطی_تکمل_دوہرا_تکمل}) کی تصور کو وسعت دیتے ہوئے تہرا تکمل حاصل کیا جا سکتا ہے۔ فرض کریں کہ فضا کے کسی بند محدود\حاشیہد{"بند" سے مراد ہے کہ وقفے کی سرحد بھی وقفے کا حصہ ہے اور "محدود" سے مراد ہے کہ پورے وقفے کو معقول وسعت کی کرہ میں گھیرا جا سکتا ہے۔} خطہ \عددی{T} میں تفاعل \عددی{f(x,y,z)} معین ہے۔ہم تینوں محور کے متوازی سطحوں سے \عددی{T} کو ٹکڑوں میں تقسیم کرتے ہیں۔ہم \عددی{T} کے متوازی السطوح ٹکڑوں  کو ہم \عددی{1} تا \عددی{n} سے ظاہر  کرتے ہیں۔ایسے ہر ٹکڑے کے اندر ہم بے قاعدگی سے کوئی نقطہ منتخب کرتے ہیں، مثلاً ٹکڑا \عددی{k} میں نقطہ  \عددی{(x_k,y_k,z_k)} چنا جاتا ہے، اور درج ذیل مجموعہ حاصل کرتے ہیں
\begin{align*}
J_n=\sum_{k=1}^{n} f(x_k,y_k,z_k)\Delta H
\end{align*} 
جہاں ٹکڑا \عددی{k} کی حجم \عددی{\Delta H_k} ہے۔ہم مثبت عدد صحیح \عددی{n} کی قیمت بتدریج بڑھاتے ہوئے  بالکل آزادانہ طریقے سے اس طرح کے مجموعے حاصل کرتے ہیں پس اتنا خیال رکھا جاتا ہے کہ جیسے جیسے \عددی{n} کی قیمت لامتناہی کے قریب پہنچتی ہو، مستطیلی ٹکڑوں کی وتر کی زیادہ سے زیادہ لمبائی صفر تک پہنچتی ہو۔  یوں ہمیں حقیقی اعداد \عددی{J_{n1}}، \عددی{J_{n2}}، \نقطے کا سلسلہ حاصل ہو گا۔ہم فرض کرتے ہیں کہ کسی  ایسے خطہ میں، جس کا \عددی{T} حصہ ہو، \عددی{f(x,y,z)} استمراری ہے اور \عددی{T} کو لامتناہی تعداد  کی ہموار سطحیں گھیرتی ہیں۔ایسی صورت میں یہ ثابت کیا جا سکتا ہے کہ  حقیقی اعداد \عددی{J_{n1}}، \عددی{J_{n2}}، \نقطے کا سلسلہ مرتکز ہو گا جس کا حد ٹکڑوں کی چنائی یا ٹکڑوں میں نقطوں \عددی{(x_,y_k,z_k)} کی چنائی سے بالکل آزاد ہو گا (مثال \حوالہ{مثال_سمتی_تکمل_رقبہ_اور_تکمل} کی طرح)۔ اس حد کو خطہ \عددی{T} پر \عددی{f(x,y,z)} کا \اصطلاح{تہرا تکمل}\فرہنگ{تہرا!تکمل}\فرہنگ{تکمل!تہرا}\حاشیہب{triple integral}\فرہنگ{integral!triple} کہتے ہیں جس کو درج ذیل سے ظاہر کیا جاتا ہے۔
\begin{align*}
\iiint\limits_T f(x,y,z)\dif x\dif y\dif z \quad \text{یا}\quad \iiint\limits_T f(x,y,z)\dif H
\end{align*}

ہم اب  ثاب کرتے ہیں کہ  ایسا استمراری سمتی تفاعل \عددی{\bM{u}} جس کے استمراری ایک درجی جزوی تفرق پائے جاتے ہوں کی پھیلاو  کا فضا میں خطہ \عددی{T} پر تہرا تکمل کا تبادلہ \عددی{T} کی سطح پر \عددی{\bM{u}}  کے عمودی  جزو کی سطحی تکمل میں کیا جا سکتا ہے۔ایسا مسئلہ پھیلاو کی مدد سے کیا جاتا ہے جو دو بعدی مسئلہ گرین کا  تین بعدی مماثل ہے۔ مسئلہ پھیلاو کئی نظریاتی اور عملی مسائل میں بنیادی اہمیت رکھتی ہے۔

%============================
\ابتدا{مسئلہ}\شناخت{مسئلہ_خطی_تکمل_مسئلہ_پھیلاو}\quad گاوس کا مسئلہ پھیلاو (حجمی تکمل سے سطحی تکمل اور سطحی تکمل سے حجمی تکمل کا حصول)\\
فرض کریں کہ فضا میں بند محدود خطہ \عددی{T} کی سرحد \عددی{S} ٹکڑوں میں ہموار (حصہ \حوالہ{حصہ_خطی_تکمل_سطحیں}) اور قابل سمت بند ہے۔مزید فرض کریں کہ خطہ \عددی{T} میں \عددی{\bM{u}(x,y,z)} ایک استمراری سمتی تفاعل ہے جس کے \عددی{T} میں استمراری ایک درجی جزوی تفرق پائے جاتے ہیں۔ تب درج ذیل ہو گا
\begin{align}\label{مساوات_خطی_تکمل_مسئلہ_گاوس_الف}
\iiint\limits_T \nabla \cdot \bM{u}\dif H=\iint\limits_S u_n\dif A
\end{align}
جہاں \عددی{T} کی لحاظ سے سطح \عددی{S} پر \عددی{\bM{u}} کا باہر رخ عمودی جزو
\begin{align}\label{مساوات_خطی_تکمل_مسئلہ_گاوس_ب}
u_n=\bM{u}\cdot \bM{n}
\end{align}
ہے اور \عددی{\bM{n}} سطح \عددی{S} کا باہر رخ  اکائی عمودی سمتیہ ہے۔
\انتہا{مسئلہ}
%============================
\ابتدا{ثبوت}
ہم \عددی{\bM{u}} اور \عددی{\bM{n}} کو ارکان کی صورت میں لکھتے ہیں
\begin{align*}
\bM{u}=u_1\bM{i}+u_2\bM{j}+u_3\bM{k} \quad \bM{n}=\cos \alpha \,\bM{i}+\cos \beta \,\bM{j}+\cos \gamma\,\bM{k} 
\end{align*}
جہاں \عددی{\bM{n}} اور مثبت \عددی{x}، \عددی{y}، \عددی{z} محور کے مابین زاویے بالترتیب \عددی{\alpha}، \عددی{\beta}، \عددی{\gamma} ہیں۔یوں مساوات \حوالہ{مساوات_خطی_تکمل_مسئلہ_گاوس_الف} درج ذیل لکھی جا سکتی ہے
\begin{align}\label{مساوات_خطی_تکمل_مسئلہ_گاوس_پ}
\iiint\limits_T \big(\frac{\partial u_1}{\partial x}+\frac{\partial u_2}{\partial y}+\frac{\partial u_3}{\partial z}\big)\dif x\dif y\dif z=\iint\limits_S (u_1\cos \alpha+u_2\cos \beta+u_3\cos \gamma)\dif A
\end{align}
جسے مساوات \حوالہ{مساوات_خطی_تکمل_عمودی_سمتیہ_اجزاء_استعمال} کی مدد سے درج ذیل لکھی جا سکتی ہے۔
\begin{align}\label{مساوات_خطی_تکمل_مسئلہ_گاوس_ت}
\iiint\limits_T \big(\frac{\partial u_1}{\partial x}+\frac{\partial u_2}{\partial y}+\frac{\partial u_3}{\partial z}\big)\dif x\dif y\dif z=\iint\limits_S (u_1\dif y\dif z+u_2\dif x\dif z +u_3\dif x\dif y)
\end{align}
اب ظاہر ہے کہ اگر درج ذیل تین تعلقات یک وقت درست ہوں تب مساوات \حوالہ{مساوات_خطی_تکمل_مسئلہ_گاوس_پ} درست ہو گا۔
\begin{align}
\iiint\limits_T \frac{\partial u_1}{\partial x}\dif x\dif y\dif z&=\iint\limits_S u_1\cos \alpha \dif A \label{مساوات_خطی_تکمل_مسئلہ_گاوس_ٹ}\\
\iiint\limits_T \frac{\partial u_2}{\partial y}\dif x\dif y\dif z&=\iint\limits_S u_2\cos \beta \dif \label{مساوات_خطی_تکمل_مسئلہ_گاوس_ث}\\
\iiint\limits_T \frac{\partial u_3}{\partial z}\dif x\dif y\dif z&=\iint\limits_S u_3\cos \gamma \dif A\label{مساوات_خطی_تکمل_مسئلہ_گاوس_ج}
\end{align}

ہم مساوات \حوالہ{مساوات_خطی_تکمل_مسئلہ_گاوس_ج} کو ایک خصوصی خطہ \عددی{T} کے لئے ثابت کرتے ہیں جس کی سرحد ٹکڑوں میں ہموار قابل سمت بند سطح \عددی{S} ہے۔اس مخصوص \عددی{T} کی خاصیت ہے کہ \عددی{x}، \عددی{y} یا \عددی{z} محور کے متوازی کوئی بھی خط جو  \عددی{T} کو قطع کرتی ہو، کا زیادہ سے زیادہ صرف ایک  حصہ (یا صرف ایک نقطہ) \عددی{T} کے ساتھ مشترک ہو گا۔ اس خاصیت کا مطلب ہے کہ \عددی{T} کو درج ذیل روپ میں لکھا جا سکتا ہے
\begin{align}\label{مساوات_خطی_تکمل_مسئلہ_گاوس_چ}
g(x,y)\le z\le h(x,y)
\end{align}
جہاں \عددی{xy} مستوی پر \عددی{T} کے قائمہ الزاویہ سائے \عددی{\overline{R}} میں نقطہ \عددی{(x,y)} ہو گا۔ظاہر ہے کہ \عددی{g(x,y)} سطح \عددی{S} کی نچلی سطح \عددی{S_2} کو ظاہر کرتی ہے جبکہ \عددی{h(x,y)} سطح \عددی{S} کی بالائی سطح \عددی{S_1} کو ظاہر کرتی ہے۔عین ممکن ہے کہ \عددی{S} کا کوئی کھڑا حصہ \عددی{S_3} بھی پایا جاتا ہو۔(حصہ \عددی{S_3} کی انحطاطی شکل ایک منحنی ہو سکتی ہے مثلاً  کروی \عددی{T} کی صورت میں \عددی{S_3} ایک گول دائرہ ہو گا۔)   

مساوات \حوالہ{مساوات_خطی_تکمل_مسئلہ_گاوس_ج} کو مساوات \حوالہ{مساوات_خطی_تکمل_مسئلہ_گاوس_چ} کی مدد سے ثابت کرتے ہیں۔چونکہ کسی خطہ جس کا \عددی{T} حصہ ہے میں \عددی{\bM{u}} استمراری قابل تفرق ہے لہٰذا درج ذیل ہو گا۔
\begin{align}\label{مساوات_خطی_تکمل_مسئلہ_گاوس_ح}
\iiint\limits_T \frac{\partial u_3}{\partial z} \dif x\dif y\dif z=\iint\limits_{\overline{R}}\big[\int_{g(x,y)}^{h(x,y)} \frac{\partial u_3}{\partial z}\dif z\big]\dif x\dif y
\end{align}
اس میں اندرونی تکمل لیتے ہیں۔
\begin{align*}
\int_g^h \frac{\partial u_3}{\partial z}\dif z=u_3(x,y,h)-u_3(x,y,g)
\end{align*}
یوں مساوات \حوالہ{مساوات_خطی_تکمل_مسئلہ_گاوس_ح} کا بایاں ہاتھ درج ذیل کے برابر ہو گا۔
\begin{align}\label{مساوات_خطی_تکمل_مسئلہ_گاوس_خ}
\iint\limits_{\overline{R}}u_3[x,y,h(x,y)]\dif x\dif y-\iint\limits_{\overline{R}}u_3[x,y,g(x,y)]\dif x\dif y
\end{align}
آئیں اب ثابت کرتے ہیں کہ  مساوات \حوالہ{مساوات_خطی_تکمل_مسئلہ_گاوس_ج} کا دایاں ہاتھ بھی اسی کے برابر ہے۔ چونکہ \عددی{S_3} پر \عددی{\gamma=\tfrac{\pi}{2}}  ہے لہٰذا \عددی{\cos \gamma=0} ہو گا اور یوں مساوات \حوالہ{مساوات_خطی_تکمل_مسئلہ_گاوس_ح} کے دائیں ہاتھ  \عددی{S_3} پر سطحی تکمل صفر کے برابر ہو گا۔یوں درج ذیل رہ جاتا ہے۔
\begin{align*}
\iint\limits_S u_3\cos \gamma \dif A=\iint\limits_{S_1}u_3\cos \gamma \dif A+\iint\limits_{S_2}u_3\cos \gamma \dif A
\end{align*}
\عددی{S_1} پر \عددی{\gamma} زاویہ حادہ ہے لہٰذا \عددی{\sigma=\gamma} لیتے ہوئے مساوات \حوالہ{مساوات_خطی_تکمل_صریح_تفاعل_رقبہ_پ} سے \عددی{\dif A=\sec \gamma \dif x\dif y} ملتا ہے۔چونکہ \عددی{\cos \gamma\sec\gamma=1} کے برابر ہے لہٰذا یوں
\begin{align*}
\iint\limits_{S_1}u_3\cos \gamma \dif A=\iint\limits_{\overline{R}}u_3[x,y,h(x,y)]\dif x\dif y
\end{align*}
حاصل ہو گا جو مساوات \حوالہ{مساوات_خطی_تکمل_مسئلہ_گاوس_خ} میں پہلی دوہرا  تکمل کے برابر ہے۔اسی طرح \عددی{S_2} پر \عددی{\gamma} زاویہ منفرجہ ہے لہٰذا \عددی{\pi-\gamma} مساوات  \حوالہ{مساوات_خطی_تکمل_صریح_تفاعل_رقبہ_پ} میں زاویہ حادہ \عددی{\sigma} کے مترادف ہو گا۔یوں
\begin{align*}
\dif A=\sec(\pi-\gamma)\dif x\dif y=-\sec \gamma \dif x\dif y
\end{align*}
لکھتے ہوئے
\begin{align}
\iint\limits_{S_2}u_3\cos \gamma \dif A=-\iint\limits_{\overline{R}}u_3[x,y,g(x,y)]\dif x\dif y
\end{align}
ہو گا جو عین \حوالہ{مساوات_خطی_تکمل_صریح_تفاعل_رقبہ_پ} میں دوسرے دوہرا تکمل کے برابر ہے۔یوں مساوات \حوالہ{مساوات_خطی_تکمل_مسئلہ_گاوس_ج} ثابت ہوا۔

مساوات \حوالہ{مساوات_خطی_تکمل_مسئلہ_گاوس_ٹ} اور مساوات \حوالہ{مساوات_خطی_تکمل_مسئلہ_گاوس_ث} کو بالکل اسی طرح ثابت کیا جا سکتا ہے جہاں مساوات \حوالہ{مساوات_خطی_تکمل_مسئلہ_گاوس_چ} کی طرح \عددی{T} کو درج ذیل سے ظاہر کیا جائے گا۔
\begin{align*}
\tilde{g}(y,z) \le x\le \tilde{h}(y,z) \quad \text{اور}\quad g^*(x,z) \le y\le h^*(x,z)
\end{align*}
اس طرح مسئلہ پھیلاو کا مخصوص خطے میں ثبوت مکمل ہوتا ہے۔

ایسا خطہ \عددی{T} جس کو اضافی سطحوں کی مدد سے محدود تعداد کی مخصوص ٹکڑوں میں تقسیم کرنا ممکن ہو کے ہر ٹکڑے پر مسئلہ پھیلاو لاگو کرتے ہوئے تمام جوابات کو مجموعہ لینے سے  پوری خطے پر مسئلہ ثابت ہو گا۔اس ترکیب بالکل مسئلہ گرین میں استعمال کی گئی ترکیب کی طرح ہے۔ہر اضافی سطح پر دو مرتبہ حاصل سطحی تکمل کے جوابات  کا مجموعہ صفر کے برابر ہو گا جبکہ باقی سطحوں پر سطحی تکمل \عددی{T} کی پوری سطح \عددی{S} پر سطحی تکمل ہی ہو گا۔\عددی{T} کے تمام ٹکڑوں کے حجمی تکملات کا مجموعہ \عددی{T} کے حجمی تکمل کے برابر ہو گا۔

یوں کسی بھی عملی استعمال کے محدود خطہ \عددی{T} کے لئے مسئلہ پھیلاو کا ثبوت مکمل ہوتا ہے۔ 
\انتہا{ثبوت}
%============================

مسئلہ گرین خطی تکمل کے حل میں کار آمد ثابت ہوتا ہے۔اسی طرح مسئلہ پھیلاو سطحی تکمل کے حل میں کار آمد ثابت ہوتا ہے۔

%=================
\ابتدا{مثال}\quad سطحی تکمل کا حصول بذریعہ مسئلہ پھیلاو\\
درج ذیل کو تہرا  تکمل میں تبدیل کرتے ہوئے حل کریں جہاں \عددی{S} بیلن \عددی{x^2+y^2=a^2\, (0\le z\le b)} اور اس کے دونوں اطراف کی ڈھکنوں کی سطح ہے۔
\begin{align*}
I=\iint\limits_S (x^3\dif y\dif z+x^2y\dif x\dif z+x^2z\dif x\dif y)
\end{align*}
حل:یہاں مساوات \حوالہ{مساوات_خطی_تکمل_مسئلہ_گاوس_پ} اور مساوات \حوالہ{مساوات_خطی_تکمل_مسئلہ_گاوس_ت} میں \عددی{u_1=x^3}، \عددی{u_2=x^2y}، \عددی{u_3=x^2z} ہیں۔یوں خطہ \عددی{T} کی تشاکل کو دیکھ کر ہم درج ذیل لکھ سکتے ہیں۔
\begin{align*}
\iiint\limits_T (3x^2+x^2+x^2)\dif x\dif y\dif z=4\cdot 5\int_0^b \int_0^a\int_0^{\sqrt{a^2-y^2}}x^2\dif x\dif y\dif z
\end{align*}
اندرونی تکمل \عددی{\tfrac{1}{3}(a^2-y^2)^{\tfrac{3}{2}}} کے برابر ہے۔یوں \عددی{y=a\cos t} چنتے ہوئے
\begin{align*}
\dif y=-a\sin t\dif t,\quad (a^2-y^2)^{\frac{3}{2}}=a^3\sin^3 t
\end{align*}
لکھا جا سکتا ہے۔اب \عددی{y} پر تکمل
\begin{align*}
\frac{1}{3}\int_0^a (a^2-y^2)^{\frac{3}{2}}\dif y=-\frac{1}{3} a^4\int_{\frac{\pi}{2}}^{0}\sin^4 t \dif t=\frac{\pi a^4}{16}
\end{align*}
ہو گا اور آخر میں \عددی{z} پر تکمل جزو \عددی{b} دیتا  ہے لہٰذا جواب درج ذیل ہو گا۔
\begin{align*}
I=4\cdot 5\frac{\pi a^4}{16}b=\frac{5}{4}\pi a^4b
\end{align*}
\انتہا{مثال}
%===================

\حصہ{مسئلہ پھیلاو کے نتائج اور استعمال}
مسئلہ پھیلاو کی عملی استعمال اور اس کے چند اہم نتائج کی مثالیں اس حصے میں پیش کی جائیں گی۔ان مثالوں میں فرض کیا جاتا ہے کی تفاعل اور خطہ  مسئلہ پھیلاو کے شرائط  پر پورا اترتے ہیں۔ مزید کہ سطح \عددی{S} پر خطہ \عددی{T} کا باہر رخ اکائی عمودی سمتیہ  \عددی{\bM{n}} ہے۔

%=================
\ابتدا{مثال}\quad محدد سے آزاد پھیلاو\\
مسئلہ پھیلاو کی (مساوات \حوالہ{مساوات_خطی_تکمل_مسئلہ_گاوس_الف}) کے دونوں اطراف کو خطہ \عددی{T} کی حجم \عددی{H(T)} سے تقسیم کرتے ہوئے
\begin{align}\label{مساوات_خطی_تکمل_پھیلاو_مثال_الف}
\frac{1}{H(T)}\iiint\limits_T \nabla \cdot \bM{u}\dif H=\frac{1}{H(T)}\iint\limits_{S(T)}u_n \dif A
\end{align}
ملتا ہے جہاں \عددی{T} کی سرحدی سطح \عددی{S(T)} ہے۔دوہرا تکمل کی خصوصیات کو حصہ \حوالہ{حصہ_خطی_تکمل_دوہرا_تکمل} میں بیان کیا گیا۔تہرا تکمل بھی یہی خصوصیات رکھتا ہے۔بالخصوص تہرا تکمل کا \اصطلاح{مسئلہ اوسط قیمت}\فرہنگ{مسئلہ!اوسط قیمت} کہتا ہے کہ خطہ \عددی{T} میں کسی بھی استمراری تفاعل \عددی{f(x,y,z)}  کے لئے \عددی{T} میں ایسا نقطہ \عددی{Q: (x_0,y_0,z_0)} پایا جائے گا کہ درج ذیل درست ہو گا۔
\begin{align*}
\iiint\limits_T f(x,y,z)\dif H=f(x_0,y_0,z_0) H(T)
\end{align*} 
یوں \عددی{f=\nabla \cdot \bM{u}} پر کرتے ہوئے مساوات \حوالہ{مساوات_خطی_تکمل_پھیلاو_مثال_الف} سے درج ذیل ملتا ہے۔
\begin{align}\label{مساوات_خطی_تکمل_پھیلاو_مثال_ب}
\frac{1}{H(T)}\iiint\limits_T \nabla \cdot \bM{u}\, \dif H=\nabla \cdot \bM{u}(x_0,y_0,z_0)
\end{align}
فرض کریں کہ \عددی{T} میں \عددی{N:(x_1,y_1,z_1)} کوئی مقررہ نقطہ ہے اور \عددی{T} نقطہ \عددی{N} کے گرد یوں سکڑتا ہے کہ \عددی{N} سے \عددی{T} کے  دور ترین نقطے کا فاصل \عددی{d(T)} صفر کے قریب پہنچے۔اس طرح نقطہ \عددی{Q} نقطہ \عددی{N} کے قریب پہنچے گا اور مساوات \حوالہ{مساوات_خطی_تکمل_پھیلاو_مثال_الف} اور مساوات \حوالہ{مساوات_خطی_تکمل_پھیلاو_مثال_ب} سے ظاہر کہ کہ نقطہ \عددی{N} پر \عددی{\bM{u}} کی پھیلاو درج ذیل ہو گی۔
\begin{align}\label{مساوات_خطی_تکمل_پھیلاو_مثال_پ}
\nabla \cdot \bM{u}(x_1,y_1,z_1)=\lim_{d(T)\to 0}\frac{1}{V(T)}\iint\limits_{S(T)}u_n\dif A
\end{align}
اس کلیہ کو بعض اوقات پھیلاو کی تعریف تصور کیا جاتا ہے۔جہاں حصہ \حوالہ{حصہ_الاحصاء_پھیلاو} میں پھیلاو کی تعریف میں \عددی{x}، \عددی{y}، \عددی{z} محدد پائے جاتے ہیں مساوات \حوالہ{مساوات_خطی_تکمل_پھیلاو_مثال_پ} میں دی گئی پھیلاو کی تعریف محدد سے پاک ہے۔اس سے یک دم اخذ کیا جا سکتا ہے کہ پھیلاو کی قیمت پر محددی نظام کی انتخاب کا کوئی اثر نہیں پایا جاتا ہے۔
\انتہا{مثال}
%===================
\ابتدا{مثال}\quad پھیلاو کا طبعی مفہوم\\
مسئلہ پھیلاو سے سمتیہ کی پھیلاو کا مفہوم سمجھا جا سکتا ہے۔ایسا ہی کرنے کی خاطر ہم اکائی کمیتی کثافت \عددی{\rho=1} کی  داب نا پذیر  سیال کی برقرار حال (وقت کے ساتھ نہ تبدیل ہوتا) بہاو پر غور کرتے ہیں (مثال \حوالہ{مثال_الاحصاء_حرکت_سیال} بھی دیکھیں)۔ کسی بھی نقطہ \عددی{N} پر ایسی بہاو کا تعین اس نقطہ پر سمتی رفتار سمتیہ \عددی{\bM{v}(N)} سے کیا جاتا ہے۔

فرض کریں کہ فضا میں خطہ \عددی{T} کی سرحدی سطح \عددی{S} ہے اور \عددی{\bM{n}} باہر رخ \عددی{S} کا اکائی عمودی سمتیہ ہے۔اس سطح کے چھوٹے حصہ \عددی{\Delta S} جس کا رقبہ \عددی{\Delta A} ہے سے،  اندرون \عددی{S} سے بیرون \عددی{S}  رخ، اکائی وقت میں کمیت کی اخراج \عددی{v_n \Delta A}\حاشیہد{کسی نقطہ پر \عددی{v_n}  منفی ہو سکتا ہے لہٰذا ایسے نقطے پر سیال \عددی{S} میں داخل ہو گا۔} ہو گی جہاں \عددی{v_n=\bM{v}\cdot \bM{n}} سمتیہ \عددی{\bM{v}} کا \عددی{\bM{n}}  رخ جزو ہے (یعنی \عددی{S} کا عمودی جزو ہے) اور \عددی{\bM{n}} کو \عددی{\Delta S} کے کسی موزوں نقطے پر لیا گیا ہے۔یوں \عددی{T} سے کل اخراج جو \عددی{S} سے گزرتا ہے سطحی تکمل
\begin{align*}
\iint\limits_S v_n \dif A
\end{align*}
سے حاصل ہو گا۔یہ تکمل \عددی{T} کا کل اخراج دیتا ہے۔یوں \عددی{T} کی اوسط اخراج
\begin{align}\label{مساوات_خطی_تکمل_پھیلاو_مثال_ت}
\frac{1}{H}\iint\limits_S v_n\dif A
\end{align}
ہو گی جہاں \عددی{T} کا حجم \عددی{H} ہے۔چونکہ بہاو برقرار حال ہے اور سیال داب نا پذیر ہے لہٰذا \عددی{T} سے  اخراج برابر کمیت \عددی{T} کو  مہیا کی جاتی ہو گی۔یوں اگر مساوات \حوالہ{مساوات_خطی_تکمل_پھیلاو_مثال_ت} کے تکمل کی قیمت غیر صفر ہو تب \عددی{T} میں \اصطلاح{منبع}\فرہنگ{منبع}\حاشیہب{source}\فرہنگ{source} (مثبت منبع یا منفی منبع) پایا جاتا ہو گا جہاں سیال پیدا یا غائب ہوتا ہے۔

اگر ہم \عددی{T} کو ایک نقطہ \عددی{N} مانند کر دیں تب  مساوات \حوالہ{مساوات_خطی_تکمل_پھیلاو_مثال_ت} ہمیں \عددی{N} پر \اصطلاح{شدت منبع}\فرہنگ{شدت منبع}\حاشیہب{source intensity}\فرہنگ{source intensity} دیگا (مساوات \حوالہ{مساوات_خطی_تکمل_پھیلاو_مثال_پ} کا دائیں ہاتھ جہاں \عددی{v_n} کی جگہ \عددی{u_n} لکھا گیا ہے)۔ اس سے ظاہر ہے کہ داب نا پذیر سیال کی برقرار حال سمتی رفتار سمتیہ \عددی{\bM{v}} کا نقطہ \عددی{N} پر پھیلاو سے مراد \عددی{N} پر شدت منبع ہے۔صرف اور صرف اس صورت \عددی{T} میں کوئی منبع نہ ہو گا جب \عددی{\nabla \cdot \bM{v} \equiv 0} ہو اور ایسی صورت میں \عددی{} میں کسی بھی بند سطح \عددی{S^*} کے لئے درج ذیل درست ہو گا۔
\begin{align*}
\iint\limits_{S^*}v_n\dif A=0
\end{align*}
آپ نے دیکھا کہ  کسی نقطہ سے سیال کی اخراج کو اس نقطہ پر \عددی{\bM{v}} کی پھیلاو ظاہر کرتی ہے۔ہم کہتے ہیں سیال اس نقطہ سے نکل کر پھیلتا ہے۔اسی سے اس عمل کو \اصطلاح{پھیلاو}\فرہنگ{پھیلاو} کہتے ہیں۔ 
\انتہا{مثال}
%=====================
\ابتدا{مثال}\quad مساوات حرارت۔ حراری بہاو\\
ہم جانتے ہیں کہ کسی بھی جسم میں حراری توانائی کا بہاو  گرم سے سرد مقام کے رخ  ہو گا۔اس کا مطلب ہے کہ حراری بہاو کی سمتی رفتار \عددی{\bM{v}} درج طرز کی ہو گی
\begin{align}\label{مساوات_خطی_تکمل_حراری_اخراج_الف}
\bM{v}=-K\,\nabla U
\end{align}
جہاں \عددی{U(x,y,z,t)} لمحہ \عددی{t} پر نقطہ \عددی{(x,y,z)} کا درجہ حرارت ہے اور \عددی{K} جسم کی \اصطلاح{حراری موصلیت}\فرہنگ{موصلیت!حراری}\فرہنگ{حراری موصلیت}\حاشیہب{thermal conductivity}\فرہنگ{thermal conductivity} ہے۔عمومی طبعی حالات میں \عددی{K} ایک مستقل ہو گا۔
 
فرض کریں کہ جسم میں \عددی{R} کوئی خطہ ہے جس کی سرحدی سطح \عددی{S} ہے۔یوں اکائی وقت میں \عددی{R} سے کل حراری توانائی کا اخراج
\begin{align*}
\iint\limits_S v_n \dif A
\end{align*}
ہو گا جہاں \عددی{v_n=\bM{v}\cdot \bM{n}} سرحد \عددی{S} پر باہر رخ اکائی عمودی سمتیہ \عددی{\bM{n}} کی رخ  \عددی{\bM{v}} کا جزو ہے۔یہ تعلق گزشتہ مثال کی حاصل کیا گیا ہے۔ مساوات \حوالہ{مساوات_خطی_تکمل_حراری_اخراج_الف} اور مسئلہ پھیلاو سے درج ذیل لکھا جا سکتا ہے (مساوات \حوالہ{مساوات_الاحصاء_لاپلاسی_ڈھلوان_اور_پھیلاو})۔
\begin{align}
\iint\limits_S v_n \dif A=-K\iiint\limits_R \nabla \cdot (\nabla U)\dif x\dif y\dif z=-K\iiint\limits_R \nabla^{\,2}U \dif x\dif y\dif z
\end{align}
 \عددی{R} میں کل حراری توانائی \عددی{W} درج ذیل ہے
\begin{align*}
W=\iiint\limits_R \sigma \rho U \dif x\dif y\dif z
\end{align*}
جہاں \عددی{\sigma} جسم کے مواد کی \اصطلاح{خصوصی حراری استعداد}\فرہنگ{حراری!خصوصی استعداد}\حاشیہب{specific heat capacity}\فرہنگ{heat!specific capacity} ہے جبکہ \عددی{\rho} جسم کی کمیتی کثافت (کمیت فی اکائی حجم) ہے۔یوں جسم میں حراری توانائی کی وقت کے ساتھ گھٹاو 
\begin{align*}
-\frac{\partial W}{\partial t}=-\iiint\limits_R \sigma \rho \frac{\partial U}{\partial t} \dif x\dif y\dif z
\end{align*}
ہو گی جو عین \عددی{R} سے توانائی کی اخراج کے برابر ہو گا یعنی
\begin{align*}
-\iiint\limits_R \sigma \rho \frac{\partial U}{\partial t} \dif x\dif y\dif z=-K\iiint\limits_R \nabla^{\,2}U \dif x\dif y\dif z
\end{align*} 
یا:
\begin{align*}
\iiint\limits_R \big(\sigma \rho\frac{\partial U}{\partial t}-K\nabla^{\,2}U\big)\dif x\dif y\dif z=0
\end{align*}
 چونکہ یہ مساوات کسی بھی خطہ \عددی{R} کے لئے درست ہے لہٰذا متکمل (اگر استمراری ہو تب) تمام \عددی{R} میں صفر کے برابر ہو گا یعنی:
\begin{align}
\frac{\partial U}{\partial t}=c^2\nabla^{\,2}U\quad \quad \quad (c^2=\frac{K}{\sigma \rho})
\end{align}
یہ \اصطلاح{حراری مساوات}\فرہنگ{حراری!مساوات}\حاشیہب{heat equation}\فرہنگ{heat!equation} کہلاتی ہے جو حراری بہاو کی بنیادی مساوات ہے۔
\انتہا{مثال}
%=====================
\ابتدا{مثال}\quad لاپلاسی مساوات کے حل کی بنیادی خصوصیت\\
مسئلہ پھیلاو کی مساوات
\begin{align}\label{مساوات_خطی_تکمل_مسئلہ_پھیلاو_دوبارہ_الف}
\iiint\limits_T \nabla \bM{u}\dif H=\iint\limits_S u_n \dif A
\end{align}
پر غور کریں۔فرض کریں کہ \عددی{\bM{u}} کسی غیر سمتی تفاعل کی ڈھلوان \عددی{\bM{u}=\nabla f} ہے۔یوں
\begin{align*}
\nabla \cdot \bM{u}=\nabla \cdot (\nabla f)=\nabla^{\,2}f
\end{align*}
ہو گا (مساوات \حوالہ{مساوات_الاحصاء_لاپلاسی_ڈھلوان_اور_پھیلاو})۔مزید 
\begin{align*}
u_n=\bM{u}\cdot \bM{n}=\bM{n}\cdot \nabla f
\end{align*}
لکھا جائے گا جو مساوات \حوالہ{مساوات_الاحصاء_سمتی_تفرق_ج} کے تحت \عددی{S} کے باہر رخ \عددی{f} کا سمتی تفرق ہے جس کو \عددی{\tfrac{\partial f}{\partial n}} سے ظاہر کرتے ہوئے  مساوات \حوالہ{مساوات_خطی_تکمل_مسئلہ_پھیلاو_دوبارہ_الف} کو درج ذیل لکھا جا سکتا ہے۔
\begin{align}\label{مساوات_خطی_تکمل_مسئلہ_پھیلاو_دوبارہ_ب}
\iiint\limits_T \nabla^{\,2}f\dif H=\iint\limits_S \frac{\partial f}{\partial n}\dif A
\end{align}
ظاہر ہے کہ یہ مساوات \حوالہ{مساوات_خطی_تکمل_لاپلاسی_قائمہ_تبادل} کی تین بعدی مماثل ہے۔
\انتہا{مثال}
%=========================

مسئلہ پھیلاو کے لئے درکار شرائط کو مد نظر رکھتے ہوئے  مساوات \حوالہ{مساوات_خطی_تکمل_مسئلہ_پھیلاو_دوبارہ_ب} سے درج ذیل اخذ کیا جا سکتا ہے۔ 

%==========================
\ابتدا{مسئلہ}\quad (لاپلاسی مساوات کے حل کی خصوصیت)\\
فرض کریں کہ کسی خطہ \عددی{D} میں تفاعل \عددی{f(x,y,z)} لاپلاسی مساوات
\begin{align*}
\nabla^{\,2}f=\frac{\partial^{\,2} f}{\partial x^2}+\frac{\partial^{\,2} f}{\partial y^2}+\frac{\partial^{\,2} f}{\partial z^2}=0
\end{align*}
کا حل ہے اور \عددی{D} میں \عددی{f} کے دو درجی جزوی تفرق استمراری ہیں۔تب \عددی{D} میں کسی بھی ٹکڑوں میں ہموار  بند اور قابل سمت بند سطح \عددی{S} پر \عددی{f} کے عمودی  (سمتی) تفرق کا تکمل صفر ہو گا۔
\انتہا{مسئلہ}
%========================
\ابتدا{مثال}\quad مسئلہ گرین\\

\انتہا{مثال}
%=========================
