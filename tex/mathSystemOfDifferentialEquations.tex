\باب{نظامِ تفرقی مساوات}
گزشتہ باب میں آپ نے بلند درجی سادہ تفرقی مساوات کو حل کرنا سیکھا۔اس باب میں سادہ تفرقی مساوات حل کرنے کا نیا طریقہ دکھایا جائے گا جس میں \عددی{n} درجی سادہ تفرقی مساوات سے \عددی{n} عدد درجہ اول سادہ تفرقی مساوات کا نظام حاصل کیا جائے گا۔اس نظام کو حل کرنا بھی سکھایا جائے گا۔تفرقی مساوات کے نظام کو قالب اور سمتیہ کی صورت میں لکھنا زیادہ مفید ثابت ہوتا ہے لہٰذا حصہ \حوالہ{حصہ_نظام_قالب} میں قالب اور سمتیہ کے بنیادی حقائق پر غور کیا جائے گا۔

اسی باب میں تفرقی مساوات کے نظام کو حل کرنے کی بجائے تمام مساوات کی مجموعی طرز عمل پر غور کیا جائے گا جس سے نظام کے حل کی \اصطلاح{توازن}\فرہنگ{توازن}\حاشیہب{stability}\فرہنگ{stability} کے بارے میں معلومات حاصل ہوتی ہے۔انجینئری میں متوازن نظام  اہمیت رکھتے ہیں۔متوازن نظام میں کسی لمحے پر معمولی تبدیلی، بعد کے لمحات پر معمولی تبدیلی ہی پیدا کرتی ہے۔اس ترکیب سے مساوات کا اصل حل دریافت نہیں ہوتا لہٰذا اس کو \اصطلاح{کیفی ترکیب}\فرہنگ{کیفی ترکیب}\فرہنگ{ترکیب!کیفی}\حاشیہب{qualitative method}\فرہنگ{qualitative method} کہتے ہیں۔جس ترکیب سے نظام کا اصل حل حاصل ہوتا ہو اس کو \اصطلاح{مقداری ترکیب}\فرہنگ{مقداری ترکیب}\فرہنگ{ترکیب!مقداری}\حاشیہب{quantitative method}\فرہنگ{quantitative method} کہتے ہیں۔

 
\حصہ{قالب اور سمتیہ کے بنیادی حقائق}\شناخت{حصہ_نظام_قالب}
تفرقی مساوات کے نظام پر غور کے دوران قالب اور سمتیات استعمال کئے جائیں گے۔

دو عدد خطی سادہ تفرقی مساوات کے نظام
\begin{gather}\label{مساوات_نظام_دو_مساوات}
\begin{aligned}
y_1' &=a_{11}y_1+a_{12}y_2\\
y_2' &=a_{21}y_1+a_{22}y_2
\end{aligned}\quad \text{یا} \quad
\begin{aligned}
y_1' &=2y_1-7y_2\\
y_2' &=5y_1+y_2
\end{aligned}
\end{gather}
 میں دو عدد نا معلوم تفاعل \عددی{y_1(t)} اور \عددی{y_2(t)} پائے جاتے ہیں۔ان مساوات میں دائیں جانب اضافی تفاعل \عددی{g_1(t)} اور \عددی{g_2(t)} بھی موجود ہو سکتے ہیں۔اسی طرح \عددی{n} عدد درجہ اول سادہ تفرقی مساوات پر مبنی نظام
\begin{gather}
\begin{aligned}\label{مساوات_نظام_متعدد_مساوات}
y_1' &=a_{11}y_1+a_{12}y_2+\cdots+a_{1n}y_n\\
y_2' &=a_{21}y_1+a_{22}y_2+\cdots a_{2n}y_n\\
\vdots &\\
y_n'&=a_{n1}y_1+a_{n2}y_2+\cdots+a_{nn}y_n
\end{aligned}
\end{gather}
 میں \عددی{y_1(t)} تا \عددی{y_n(t)} نا معلوم تفاعل پائے جائیں گے۔درج بالا ہر مساوات میں دائیں جانب اضافی تفاعل بھی پائے جا سکتے ہیں۔
%====================
\جزوحصہء{تکنیکی اصطلاحات}
\جزوحصہء{قالب}
نظام \حوالہ{مساوات_نظام_دو_مساوات} کے عددی سر (جو مستقل یا متغیرات ممکن ہیں) کو \عددی{2 \times 2} \اصطلاح{قالب}\فرہنگ{قالب}\حاشیہب{matrix}\فرہنگ{matrix} \bM{A} کی صورت میں لکھا جا سکتا ہے۔
\begin{gather}\label{مساوات_نظام_دو_جمع_دو_قالب}
\begin{aligned}
\bM{A}=[a_{jk}]=
\begin{bmatrix}
a_{11} & a_{12}\\
a_{21} & a_{22}
\end{bmatrix}
\end{aligned}\quad \text{یا} \quad
\begin{aligned}
\bM{A}=[a_{jk}]=
\begin{bmatrix}
3 & 2\\
-1 &\frac{2}{3}
\end{bmatrix}
\end{aligned}
\end{gather}
اسی طرح نظام \حوالہ{مساوات_نظام_متعدد_مساوات} کے عددی سر کو \عددی{n \times n} قالب کی صورت میں لکھا جا سکتا ہے۔
\begin{align}\label{مساوات_نظام_متعدد_جمع_متعدد_قالب}
\bM{A}=[a_{jk}]=
\begin{bmatrix}
a_{11} & a_{12} &\cdots & a_{1n}\\
a_{21} & a_{22}& \cdots & a_{2n}\\
& \vdots &  &\\
a_{n1} & a_{n2}& \cdots & a_{nn}
\end{bmatrix}
\end{align}
قالب میں درج \عددی{a_{11}}، \عددی{a_{12}}، \عددی{a_{21}} وغیرہ کو \اصطلاح{ارکان}\فرہنگ{ارکان}\حاشیہب{entry}\فرہنگ{entry} کہتے ہیں۔ افقی لکیروں کو \اصطلاح{صف}\فرہنگ{صف}\حاشیہب{row}\فرہنگ{row} جبکہ عمودی لکیروں کو \اصطلاح{قطار}\فرہنگ{قطار}\حاشیہب{column}\فرہنگ{column} کہتے ہیں۔قالب \حوالہ{مساوات_نظام_دو_جمع_دو_قالب} میں پہلا صف \عددی{[a_{11} \,\,\,  a_{12}]} یا \عددی{[3\,\,\, 2]} جبکہ دوسرا صف \عددی{[a_{21} \,\,\,  a_{22}]} یا \عددی{[-1\,\,\,\tfrac{2}{3}]} ہے۔اسی طرح پہلا قطار درج ذیل ہے۔
\begin{gather*}
\begin{aligned}
\begin{bmatrix}
a_{11}\\
a_{21}
\end{bmatrix}\quad \text{یا}\quad 
\begin{bmatrix}
3\\
-1
\end{bmatrix}\quad \text{}\quad 
\end{aligned}
\end{gather*}
ارکان کی علامتی اظہار میں دو گنا زیر نوشت کا پہلا عدد صف کو ظاہر کرتا ہے جبکہ دوسرا عدد قطار کو ظاہر کرتا ہے۔یوں \عددی{a_{21}} دوسری صف اور پہلی قطار کا رکن ہے۔ قالب \حوالہ{مساوات_نظام_دو_جمع_دو_قالب} کا \اصطلاح{مرکزی وتر}\فرہنگ{مرکزی وتر}\فرہنگ{وتر!مرکزی}\حاشیہب{main diagonal}\فرہنگ{diagonal!main} \عددی{a_{11}} اور \عددی{a_{22}} پر مبنی ہے جبکہ قالب \حوالہ{مساوات_نظام_متعدد_جمع_متعدد_قالب} کا مرکزی وتر \عددی{a_{11}}، \عددی{a_{22}}، \نقطے، \عددی{a_{nn}} پر مبنی ہے۔ہمیں یہاں صرف \اصطلاح{مربع قالب}\فرہنگ{مربع قالب}\فرہنگ{قالب!مربع}\حاشیہب{square matrix}\فرہنگ{matrix!square} درکار ہوں گے۔مربع قالب سے مراد ایسی قالب ہے جس میں صفوں کی تعداد قطاروں کی تعداد کے برابر ہو۔ قالب \حوالہ{مساوات_نظام_دو_جمع_دو_قالب} اور قالب \حوالہ{مساوات_نظام_متعدد_جمع_متعدد_قالب} مربع قالب ہیں۔

\اصطلاح{سمتیہ}۔\quad ایک قطار اور \عددی{n} ارکان کا \اصطلاح{سمتیہ قطار}\فرہنگ{سمتیہ!قطار}\فرہنگ{قطار!سمتیہ}\حاشیہب{column vector}\فرہنگ{vector!column} درج ذیل ہے۔
\begin{align*}
\bM{x}=
\begin{bmatrix}
x_1\\
x_2\\
x_3\\
\vdots\\
x_n
\end{bmatrix}
\end{align*}
اسی طرح ایک صف اور \عددی{n} ارکان کا \اصطلاح{سمتیہ صف}\فرہنگ{سمتیہ صف}\فرہنگ{سمتیہ!صف}\حاشیہب{row vector}\فرہنگ{vector!row} درج ذیل ہے۔
\begin{align*}
\bM{v}=
\begin{bmatrix}
v_1 & v_2& v_3& \cdots & v_n
\end{bmatrix}
\end{align*}
%=========================
\جزوحصہء{قالب اور سمتیات کا حساب}
\جزوحصہء{برابری مساوات} دو عدد \عددی{n \times n} قالب صرف اور صرف اس صورت برابر ہوں گے جب ان کے تمام \اصطلاح{نظیری}\فرہنگ{نظیری}\حاشیہب{corresponding}\فرہنگ{corresponding} ارکان \ترچھا{برابر} ہوں۔ظاہر ہے کہ دو قالب کی برابری کے لئے لازم ہے کہ ان میں صفوں کی تعداد یکساں ہو اور ان میں قطاروں کی تعداد یکساں ہو۔یوں \عددی{n=2} کی صورت میں 
\begin{align*}
\bM{A}=
\begin{bmatrix}
a_{11} & a_{12}\\
a_{21} & a_{22}
\end{bmatrix} \quad \text{اور} \quad
\bM{B}=
\begin{bmatrix}
b_{11} & b_{12}\\
b_{21} & b_{22}
\end{bmatrix}
\end{align*}
صرف اور صرف اس صورت برابر \عددی{(\bM{A}=\bM{B})} ہوں گے جب
\begin{align*}
a_{11}&=b_{11}, \quad a_{12}=b_{12}\\
a_{21}&=b_{21}, \quad a_{22}=b_{22}
\end{align*}
ہوں۔دو عدد سمتیہ صف (یا دو عدد سمتیہ قطار) صرف اور صرف اس صورت \ترچھا{برابر} ہوں گے جب دونوں میں ارکان کی تعداد  \عددی{n} برابر ہو اور  ان کے تمام نظیری  ارکان \ترچھا{برابر} ہوں ۔یوں 
\begin{align*}
\bM{v}=
\begin{bmatrix}
v_1\\
v_2
\end{bmatrix} \quad \text{اور} \quad
\bM{x}=
\begin{bmatrix}
x_1\\
x_2
\end{bmatrix}
\end{align*} 
کی صورت میں \عددی{\bM{v}=\bM{x}} صرف اور صرف تب ہو گا جب
\begin{align*}
v_1=x_1\quad \text{اور} \quad v_2=x_2
\end{align*}
ہوں۔

\جزوحصہء{مجموعہ} 
مجموعہ حاصل کرنے کی خاطر دونوں قالب کے نظیری ارکان کا مجموعہ لیا جاتا ہے۔دونوں قالب یکساں \عددی{m \times n} ہونا لازم ہے۔اسی طرح دونوں سمتیہ صف (یا دونوں سمتیہ قطار) میں برابر ارکان ہونا لازم ہے۔یوں \عددی{2 \times 2} قالب کا مجموعہ درج ذیل ہو گا۔
\begin{align}
\bM{A}+\bM{B}=
\begin{bmatrix}
a_{11}+b_{11} & a_{12}+b_{12}\\
a_{21}+b_{21}& a_{22}+b_{bb}
\end{bmatrix}, \quad \bM{v}+\bM{x}=
\begin{bmatrix}
v_1+x_1\\
v_2+x_2
\end{bmatrix}
\end{align} 
\جزوحصہء{غیر سمتی ضرب}\فرہنگ{غیر سمتی ضرب}\حاشیہب{scalar product}\فرہنگ{scalar product} غیر سمتی ضرب یعنی مستقل  \عددی{c} سے قالب کا ضرب حاصل کرنے کی خاطر قالب کے تمام ارکان کو \عددی{c} سے ضرب دیا جاتا ہے۔مثلاً 
\begin{align*}
\bM{A}=
\begin{bmatrix}
2& -3\\
5&1
\end{bmatrix}, \quad 
-4\bM{A}=
\begin{bmatrix}
-8&12\\
-20&-4
\end{bmatrix}
\end{align*}
اور
\begin{align*}
\bM{v}=
\begin{bmatrix}
9\\
-4
\end{bmatrix}, \quad 
3\bM{v}=
\begin{bmatrix}
27\\
-12
\end{bmatrix}
\end{align*}
\جزوحصہء{قالب ضرب قالب}
دو عدد \عددی{n \times n} قالب \عددی{\bM{A}=[a_{jk}]} اور \عددی{\bM{B}=[b_{jk}]}  کا حاصل ضرب \عددی{\bM{C}=\bM{A}\bM{B}}، (اسی ترتیب میں) \عددی{n \times n} قالب \عددی{\bM{C}=[c_{jk}]} ہو گا جس کے ارکان 
\begin{align}
c_{jk}=\sum_{m=1}^{n} a_{jm} b_{mk} \quad \quad j=1, \cdots, n, \quad\quad k=1,\cdots,n
\end{align}
ہوں گے یعنی \عددی{\bM{A}} قالب کے \عددی{j} صف کے ہر رکن کو \عددی{\bM{B}} قالب کے \عددی{j} قطار کے نظیری رکن کے ساتھ ضرب دیتے ہوئے \عددی{n} حاصل ضرب  کا مجموعہ لیں۔ہم کہتے ہیں کہ قالب کے ضرب سے مراد  \ترچھا{صف ضرب قطار} ہے۔مثلاً
\begin{align*}
\begin{bmatrix}
2 & 1\\
-3 & 0
\end{bmatrix}
\begin{bmatrix}
7 &1\\
2&-4
\end{bmatrix}=
\begin{bmatrix}
2\cdot 7+1\cdot 2& 2 \cdot 1+1\cdot(-4)\\
(-3)\cdot 7+0\cdot 2& (-3)\cdot 1+0\cdot(-4)
\end{bmatrix}=
\begin{bmatrix}
16&-2\\
-21&-3
\end{bmatrix}
\end{align*}
یہاں دھیان رہے  کہ ضرب قالب \اصطلاح{غیر مستبدل}\فرہنگ{مستبدل!غیر}\حاشیہب{non commutative}\فرہنگ{commutative!non} ہے لہٰذا  عموماً \عددی{\bM{A}\bM{B} \ne \bM{B}\bM{A}} ہو گا۔ یوں دو قالب کو آپس میں ضرب دیتے ہوئے  قالبوں کی ترتیب تبدیل نہیں کی جا سکتی۔اس حقیقت کی وضاحت کی خاطر درج بالا مثال میں قالبوں کی ترتیب بدلتے ہوئے ان کو آپس میں ضرب  دیتے ہیں۔
\begin{align*}
\begin{bmatrix}
7 & 1\\
2 & -4
\end{bmatrix}
\begin{bmatrix}
2&1\\
-3&0
\end{bmatrix}=
\begin{bmatrix}
7\cdot 2+1\cdot(-3)& 7\cdot 1+1\cdot 0\\
2\cdot 2+(-4)\cdot (-3)& 2\cdot 1 +(-4)\cdot 0
\end{bmatrix}=
\begin{bmatrix}
11& 7\\
16 &2
\end{bmatrix}
\end{align*}
\عددی{n \times n} قالب \عددی{\bM{A}} کو \عددی{n} ارکان کی سمتیہ قطار \عددی{\bM{x}} سے ضرب بھی اسی قاعدے کے تحت حاصل کی جاتی ہے۔یوں \عددی{\bM{v}=\bM{A}\bM{x}} کے \عددی{n} عدد ارکان درج ذیل ہوں گے۔
\begin{align}
v_j=\sum_{m=1}^{n} a_{jm}x_{m} \quad \quad j=1,\cdots, n
\end{align}
یوں
\begin{align*}
\begin{bmatrix}
7 & -3\\
1 & 4
\end{bmatrix}
\begin{bmatrix}
x_1\\
x_2
\end{bmatrix}=
\begin{bmatrix}
7x_1-3x_2\\
x_1+4x_2
\end{bmatrix}
\end{align*}
ہو گا۔
%============================
\جزوحصہء{سادہ تفرقی مساوات کے نظام کا اظہار بذریعہ سمتیات}
\جزوحصہء{تفرق}
قالب یا سمتیہ کا تفرق، تمام ارکان کا تفرق حاصل کرنے سے حاصل ہوتا ہے۔
\begin{align*}
\bM{y}(t)=
\begin{bmatrix}
y_1(t)\\[0.5ex]
y_2(t)
\end{bmatrix}=
\begin{bmatrix}
5t^3\\[0.5ex]
6\cos 2t
\end{bmatrix},
\quad 
\bM{y}'(t)=
\begin{bmatrix}
y'_1(t)\\[0.5ex]
y'_2(t)
\end{bmatrix}=
\begin{bmatrix}
15t^2\\[0.5ex]
-12\sin 2t
\end{bmatrix}
\end{align*}
قالب کی تفرق اور ضرب کو استعمال کرتے ہوئے مساوات \حوالہ{مساوات_نظام_دو_مساوات} کو درج ذیل لکھا جا سکتا ہے۔ 
\begin{gather}
\begin{aligned}
\bM{y}'=
\begin{bmatrix}
y'_1\\[0.5ex]
y'_2
\end{bmatrix}=\bM{A}\bM{x}=
\begin{bmatrix}
a_{11}& a_{12}\\[0.5ex]
a_{21}& a_{22}
\end{bmatrix}
\begin{bmatrix}
y_1\\[0.5ex]
y_2
\end{bmatrix}\quad \text{یا}\quad
\begin{bmatrix}
y'_1\\[0.5ex]
y'_2
\end{bmatrix}=
\begin{bmatrix}
2&-7\\[0.5ex]
5&1
\end{bmatrix}
\begin{bmatrix}
y_1\\[0.5ex]
y_2
\end{bmatrix}
\end{aligned}
\end{gather}
اسی طرح مساوات \حوالہ{مساوات_نظام_متعدد_مساوات} کو  درج ذیل \عددی{\bM{y}=\bM{A}\bM{x}} صورت میں لکھا جا سکتا ہے۔
\begin{gather}
\begin{aligned}
\begin{bmatrix}
y_1'\\
y_2'\\
\vdots\\
y_n'
\end{bmatrix}=
\begin{bmatrix}
a_{11}&a_{12}&\cdots &a_{1n}\\
a_{21}& a_{22}&\cdots &a_{2n}\\
&\vdots & &\\
a_{n1}&a_{n2}&\cdots &a_{nn}
\end{bmatrix}
\begin{bmatrix}
y_1\\
y_2\\
\vdots\\
y_n
\end{bmatrix}
\end{aligned}
\end{gather}
\جزوحصہء{مزید اعمال اور اصطلاحات}
\جزوحصہء{تبدیل محل} 
\اصطلاح{تبدیلی محل}\فرہنگ{تبدیلی محل}\حاشیہب{transposition}\فرہنگ{transposition} کے عمل سے قالب کے قطاروں کو صفوں کی جگہ لکھا جاتا ہے۔یوں \عددی{2\times 2} قالب \عددی{\bM{A}} سے \اصطلاح{تبدیلی محل}\فرہنگ{تبدیلی محل}\حاشیہب{transposition}\فرہنگ{transposition} کے ذریعہ \اصطلاح{تبدیلی محل قالب}\فرہنگ{تبدیلی محل قالب}\فرہنگ{قالب!تبدیلی محل}\حاشیہب{transpose matrix}\فرہنگ{transpose matrix} \عددی{\bM{A}^T} حاصل ہو گا۔ 
\begin{gather*}
\begin{aligned}
\bM{A}=
\begin{bmatrix}
a_{11} & a_{12}\\
a_{21}& a_{22}
\end{bmatrix}=
\begin{bmatrix}
2&-11\\
4&3
\end{bmatrix},\quad \quad
\bM{A}^T=
\begin{bmatrix}
a_{11}& a_{21}\\
a_{12}& a_{22}
\end{bmatrix}=
\begin{bmatrix}
2 &4\\
-11&3
\end{bmatrix}
\end{aligned}
\end{gather*}
سمتیہ صف \عددی{\bM{x}} کا تبدیلی محل سمتیہ \عددی{\bM{x}^T} سمتیہ قطار ہو گا۔اسی طرح سمتیہ قطار \عددی{\bM{v}} کا تبدیلی محل سمتیہ \عددی{\bM{v}^T} سمتیہ صف ہو گا۔
\begin{gather*}
\begin{aligned}
\bM{x}=
\begin{bmatrix}
x_1& x_2
\end{bmatrix} \quad
\bM{x}^T=
\begin{bmatrix}
x_1\\
x_2
\end{bmatrix}, \quad \quad
\bM{v}=
\begin{bmatrix}
v_1\\
v_2
\end{bmatrix}\quad
\bM{v}^T=
\begin{bmatrix}
v_1 & v_2
\end{bmatrix}
\end{aligned}
\end{gather*}
\جزوحصہء{قالب کا معکوس}
 ایسا \عددی{n \times n} قالب جس کے مرکزی وتر کے تمام ارکان اکائی \عددی{(1)}  اور بقایا  ارکان صفر ہوں کو \اصطلاح{اکائی قالب}\فرہنگ{اکائی قالب}\حاشیہب{unit matrix}\فرہنگ{unit matrix} \عددی{\bM{I}} کہتے ہیں۔
\begin{align}
\bM{I}=
\begin{bmatrix}
1&0 &0 & \cdots &0\\
0&1&0&\cdots & 0\\
0&0&1&\cdots &0\\
\vdots &&&&\\
0&0&0&\cdots&1
\end{bmatrix}
\end{align}
ایسا \عددی{\bM{B}} قالب، جس کا \عددی{\bM{A}} قالب کے ساتھ حاصل ضرب اکائی قالب ہو \عددی{\bM{B}\bM{A}=\bM{B}\bM{A}=\bM{I}}،   قالب \عددی{\bM{A}} کا معکوس قالب کہلاتا ہے جسے \عددی{\bM{A}^{-1}} لکھا جاتا ہے جبکہ ایسی صورت میں \عددی{\bM{A}} \اصطلاح{غیر نادر قالب}\فرہنگ{غیر نادر قالب}\فرہنگ{قالب!غیر نادر}\حاشیہب{non singular matrix}\فرہنگ{matrix!non singular} کہلاتا ہے۔یہاں \عددی{\bM{A}} اور \عددی{\bM{B}} دونوں \عددی{n \times n} قالب ہیں۔
\begin{align}
\bM{A}^{-1}\bM{A}=\bM{A}\bM{A}^{-1}=\bM{I}
\end{align}
قالب \عددی{\bM{A}} کا معکوس تب پایا جاتا ہے جب \عددی{\bM{A}} کی حتمی قیمت غیر صفر\عددی{\abs{\bM{A}}\ne 0} ہو۔اگر \عددی{\bM{A}} کا معکوس نہ پایا جاتا ہو تب \عددی{\bM{A}} \اصطلاح{نادر}\فرہنگ{نادر!قالب}\فرہنگ{قالب!نادر}\حاشیہب{singular}\فرہنگ{singular!matrix}\فرہنگ{matrix!singular} قالب کہلاتا ہے۔ مربع \عددی{2 \times 2} قالب کا معکوس
\begin{align}
\bM{A}^{-1}=\frac{1}{\abs{\bM{A}}}
\begin{bmatrix}
a_{22}& -a_{12}\\
-a_{21}& a_{22}
\end{bmatrix}
\end{align}
ہے جہاں \عددی{\bM{A}} کی حتمی قیمت \عددی{\abs{\bM{A}}} درج ذیل ہے۔
\begin{align}
\abs{\bM{A}}=
\begin{vmatrix}
a_{11}&a_{12}\\
a_{21}&a_{22}
\end{vmatrix}=a_{11}a_{22}-a_{12}a_{21}
\end{align}
\جزوحصہء{خطی طور تابعیت}
\عددی{r} عدد سمتیات \عددی{\bM{v}^{(1)}} تا \عددی{\bM{v}^{(r)}} جہاں ہر سمتیہ \عددی{n} ارکان پر مشتمل ہو، اس صورت \اصطلاح{خطی طور غیر تابع سلسلہ}\فرہنگ{خطی طور!غیر تابع سلسلہ}\فرہنگ{سلسلہ!خطی طور غیر تابع}\حاشیہب{linearly independent set}\فرہنگ{linearly!independent set}\فرہنگ{set!linearly independent} یا \اصطلاح{خطی طور غیر تابع} کہلاتے ہیں جب
\begin{align}\label{مساوات_نظام_خطی-طور_غیر_تابع_قالب}
c_1\bM{v}^{(1)}+\cdots+c_r \bM{v}^{(r)}=\bM{0}
\end{align}
سے مراد  \عددی{c_1} تا \عددی{c_r} کی قیمتیں صفر ہو۔درج بالا مساوات میں \عددی{\bM{0}} \اصطلاح{صفر سمتیہ}\فرہنگ{صفر سمتیہ}\فرہنگ{سمتیہ!صفر}\حاشیہب{zero vector}\فرہنگ{zero vector} ہے جس کے تمام \عددی{n} ارکان صفر کے برابر ہیں۔اگر مساوات \حوالہ{مساوات_نظام_خطی-طور_غیر_تابع_قالب} میں \عددی{c_1} تا \عددی{c_r} کوئی ایک یا ایک سے زائد مستقل غیر صفر ہوں تب \عددی{\bM{v}^{(1)}} تا \عددی{\bM{v}^{(r)}} \اصطلاح{خطی طور تابع سلسلہ}\فرہنگ{خطی طور!تابع سلسلہ}\فرہنگ{سلسلہ!خطی طور تابع}\حاشیہب{linearly dependent vector}\فرہنگ{linearly dependent vector} یا \اصطلاح{خطی طور تابع} کہلائیں گے چونکہ ایسی صورت میں کم از کم ایک سمتیہ کو بقایا سمتیات کی مدد سے لکھا جا سکتا ہے، مثلاً \عددی{c_1 \ne 0} کی صورت میں مساوات \حوالہ{مساوات_نظام_خطی-طور_غیر_تابع_قالب} کو \عددی{c_1} سے تقسیم کرتے ہوئے
\begin{align*}
\bM{v}^{(1)}=-\frac{1}{c_1}\left[c_2\bM{v}^{(2)}+\cdots+c_r \bM{v}^{(r)}\right]
\end{align*}

لکھا جا سکتا ہے۔

\جزوحصہء{آئگنی قدر اور آئگنی سمتیات}
\اصطلاح{آئگنی قدر}\فرہنگ{آئگنی!قدر}\حاشیہب{Eigenvalues}\فرہنگ{Eigenvalues} اور \اصطلاح{آئگنی سمتیات}\فرہنگ{آئگنی!سمتیات}\حاشیہب{Eigenvectors}\فرہنگ{Eigenvectors} انتہائی اہم ہیں جو \اصطلاح{کوانٹم میکانیات}\فرہنگ{کوانٹم میکانیات}\حاشیہب{quantum mechanics}\فرہنگ{quantum mechanics} میں  کلیدی کردار ادا کرتے ہیں۔مساوات
\begin{align}\label{مساوات_نظام_آئگنی_الف}
\bM{A}\bM{x}=\lambda \bM{x}
\end{align}
میں \عددی{\bM{A}=[a_{jk}]} معلوم \عددی{n \times n} قالب ہے جبکہ  \عددی{\lambda} نا معلوم مستقل (جو حقیقی یا مخلوط مقدار ہو سکتا ہے) اور \عددی{\bM{x}} نا معلوم  سمتیہ ہے جنہیں حاصل کرنا درکار ہے۔کسی بھی \عددی{\lambda} کے لئے مساوات \حوالہ{مساوات_نظام_آئگنی_الف} کا ایک حل \عددی{\bM{x}=\bM{0}} ممکن ہے۔ایسی \اصطلاح{غیر سمتی}\فرہنگ{غیر سمتی!مقدار}\حاشیہب{scalar}\فرہنگ{scalar}  \عددی{\lambda} جو \عددی{\bM{x} \ne \bM{0}} کی صورت میں مساوات \حوالہ{مساوات_نظام_آئگنی_الف} پر پورا اترتی ہو، \عددی{\bM{A}} کی \اصطلاح{آئگنی قدر}\فرہنگ{آئگنی قدر}\حاشیہب{Eigenvalue}\فرہنگ{Eigenvalue} کہلاتی ہے جبکہ،  اس \عددی{\lambda} کی نظیری، \عددی{\bM{x}} کو  \عددی{\bM{A}} کی  \اصطلاح{آئگنی سمتیہ}\فرہنگ{آئگنی سمتیہ}\حاشیہب{Eigenvector}\فرہنگ{Eigenvector} کہتے ہیں۔

ہم مساوات \حوالہ{مساوات_نظام_آئگنی_الف} کو \عددی{\bM{A}\bM{x}-\lambda\bM{x}=0} یا 
\begin{align}\label{مساوات_نظام_آئگنی_ب}
(\bM{A}-\lambda \bM{I})\bM{x}=\bM{0}
\end{align}
لکھ سکتے ہیں جو \عددی{n} عدد خطی الجبرائی مساوات کو ظاہر کرتی ہے جس کے نا معلوم متغیرات \عددی{x_1} تا \عددی{x_n}، سمتیہ \عددی{\bM{x}} کے ارکان ہیں۔اس مساوات کے غیر صفر حل \عددی{\bM{x} \ne \bM{0}} کے لئے ضروری ہے کہ \عددی{\bM{A}-\bM{I}} کے عددی سر قالب کی حتمی قیمت صفر ہو۔(یہ خطی الجبرا کی بنیادی حقیقت ہے)۔ اس باب میں ہمیں  \عددی{n=2} سے دلچسپی ہے لہٰذا مساوات \حوالہ{مساوات_نظام_آئگنی_ب} کو
\begin{align}\label{مساوات_نظام_آئگنی_پ}
\begin{bmatrix}
a_{11}-\lambda& a_{12}\\
a_{21}& a_{22}-\lambda
\end{bmatrix}
\begin{bmatrix}
x_1\\
x_2
\end{bmatrix}=
\begin{bmatrix}
0\\
0
\end{bmatrix}
\end{align}
لکھتے ہیں جو درج ذیل مساوات کو ظاہر کرتی ہے۔
\begin{gather}\label{مساوات_نظام_آئگنی_ت}
\begin{aligned}
(a_{11}-\lambda)x_1+a_{12}x_2&=0\\
a_{21}x_1+(a_{22}-\lambda)x_2&=0
\end{aligned}
\end{gather}
اب نادر قالب کی حتمی قیمت صفر ہوتی ہے لہٰذا \عددی{\bM{A}-\lambda\bM{I}}  اس صورت نادر قالب ہو گا جب اس قالب کی حتمی قیمت (جسے \عددی{\b{A}} کی \اصطلاح{امتیازی حتمی قیمت}\فرہنگ{امتیازی!حتمی قیمت}\حاشیہب{characteristic determinant}\فرہنگ{characteristic determinant} کہتے ہیں) صفر ہو۔
\begin{gather}\label{مساوات_نظام_آئگنی_ٹ}
\begin{aligned}
\abs{\bM{A}-\lambda \bM{I}}&=
\begin{vmatrix}
a_{11}-\lambda& a_{12}\\
a_{21}& a_{22}-\lambda
\end{vmatrix}\\
&=(a_{12}-\lambda)(a_{22}-\lambda)-a_{12}a_{21}\\
&=\lambda^2-(a_{11}+a_{22})\lambda+a_{11}a_{22}-a_{12}a_{21}=0
\end{aligned}
\end{gather}
اس دو درجی مساوات کو \عددی{\bM{A}} کی \اصطلاح{امتیازی مساوات}\فرہنگ{امتیازی!مساوات}\حاشیہب{characteristic equation}\فرہنگ{characteristic!equation} کہتے ہیں۔اس کے حل \عددی{\lambda_1} اور \عددی{\lambda_2}، قالب \عددی{\bM{A}} کے آئگنی قدر یا آئگنی قیمتیں ہیں۔پہلے آئگنی قدر حاصل کریں۔اس کے بعد \عددی{\lambda_1} کو مساوات \حوالہ{مساوات_نظام_آئگنی_ت} میں پر کرتے ہوئے، \عددی{\lambda_1} کی نظیری، \عددی{\bM{A}} کی آئگنی سمتیہ \عددی{\bM{x}^{(1)}} دریافت کریں۔اسی طرح \عددی{\lambda_2} کو مساوات \حوالہ{مساوات_نظام_آئگنی_ت} میں پر کرتے ہوئے، \عددی{\lambda_2} کی نظیری، \عددی{\bM{A}} کی آئگنی سمتیہ \عددی{\bM{x}^{(2)}} دریافت کریں۔یاد رہے کہ اگر \عددی{\bM{x}} قالب \عددی{\bM{A}} کا آئگنی سمتیہ ہو تب \عددی{k\bM{x}} بھی \عددی{\bM{A}} کا آئگنی سمتیہ ہو گا جہاں \عددی{k \ne 0} ہے۔
%=========================

\ابتدا{مثال}
درج ذیل قالب کی آئگنی قیمتیں اور آئگنی سمتیات دریافت کریں۔
\begin{align*}
\bM{A}=
\begin{bmatrix}
-3& 3\\
-0.8&0.4
\end{bmatrix}
\end{align*}
حل:امتیازی مساوات
\begin{align*}
\begin{vmatrix}
-3-\lambda&3\\
-0.8&0.4-\lambda
\end{vmatrix}=\lambda^2+2.6\lambda+1.2=0
\end{align*}
سے \عددی{\bM{A}} کے آئگنی قدر \عددی{\lambda_1=-0.6} اور \عددی{\lambda_2=-2} ملتے ہیں۔آئگنی قیمت \عددی{\lambda=\lambda_1=-0.6} کو مساوات \حوالہ{مساوات_نظام_آئگنی_ت} میں پر کرتے ہیں۔
\begin{gather*}
\begin{aligned}
(-3+0.6)x_1+3x_2&=0\\
-0.8x_1+(0.4+0.6)x_2&=0
\end{aligned}
\end{gather*}
ان دونوں مساوات کو \عددی{x_2=0.8x_1} لکھا جا سکتا ہے۔یوں اگر \عددی{x_1=1} چننا جائے تو \عددی{x_2=0.8} حاصل ہو گا لہٰذا، \عددی{\lambda_1=-0.6} کی نظیری، \عددی{\bM{A}} کا آئگنی سمتیہ 
\begin{align*}
\bM{x}^{(1)}=
\begin{bmatrix}
1\\
0.8
\end{bmatrix}
\end{align*}
ہو گا۔ اسی طرح \عددی{\lambda=\lambda_2=-2} کو مساوات \حوالہ{مساوات_نظام_آئگنی_ت} میں پر کرتے ہیں۔
\begin{gather*}
\begin{aligned}
(-3+2)x_1+3x_2&=0\\
-0.8x_1+(0.4+2)x_2&=0
\end{aligned}
\end{gather*}
ان دونوں مساوات کو \عددی{x_1=3x_2} لکھا جا سکتا ہے۔یوں اگر \عددی{x_2=1} چننا جائے تو \عددی{x_1=3} حاصل ہو گا لہٰذا، \عددی{\lambda_2=-2} کی نظیری، \عددی{\bM{A}} کا آئگنی سمتیہ 
\begin{align*}
\bM{x}^{(2)}=
\begin{bmatrix}
3\\
1
\end{bmatrix}
\end{align*}
ہو گا۔جیسا پہلے ذکر کیا گیا، آئگنی سمتیات کو کسی  بھی غیر صفر عدد سے ضرب دیا جا سکتا ہے۔
\انتہا{مثال}
%============================

\حصہ{سادہ تفرقی مساوات کے نظام بطور انجینئری مسائل کے نمونے}
